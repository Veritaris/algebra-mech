
\section{Вычислительная линейная алгебра}

\subsection{Системы линейных уравнений и элементарные преобразования}\label{subsection_linear_systems}
\literature{[F], гл. IV, \S~4, п. 5; [K1], гл. 1, \S~3, пп. 1, 2.}

Пусть $R$~--- ассоциативное коммутативное кольцо с единицей. Мы будем
называть \dfn{системой линейных уравнений}\index{система линейных
  уравнений} (над $R$) набор уравнений
вида
$$
\begin{array}{rcl}
a_{11}x_1+a_{12}x_2+\dots+a_{1n}x_n &=& b_1\\
a_{21}x_1+a_{22}x_2+\dots+a_{2n}x_n &=& b_2\\
\vdots & &\vdots\\
a_{m1}x_1+a_{m2}x_2+\dots+a_{mn}x_n &=& b_m,
\end{array}
$$
где $a_{ij}$ ($1\leq i\leq m$, $1\leq j\leq n$), $b_i$ ($1\leq i\leq
m$)~--- элементы $R$, а $x_1,\dots,x_n$~--- неизвестные.
\dfn{Решением}\index{решение системы линейных уравнений} этой системы линейных уравнений называется набор
$(c_1,\dots,c_n)$ элементов $R$, при подстановке которого в каждое из
$m$ уравнений системы получается верное равенство, то есть,
$\sum_{j=1}^n a_{ij}c_j=b_i$ для всех $i=1,\dots,m$.

В первом приближении линейная алгебра изучает свойства множеств
решений систем линейных уравнений. Наша ближайшая цель~--- указать
несколько преобразований, которые не меняют множество решений системы,
но, возможно, упрощают ее вид. Чтобы не писать каждый раз значки $+$ и
$=$, мы будем пользоваться {\it матричной формой записи} системы.
\dfn{Матрицей}\index{матрица!системы линейных уравнений} указанной
системы линейных уравнений называется таблица
$$
\begin{pmatrix}
a_{11} & a_{12} & \dots & a_{1n}\\
a_{21} & a_{22} & \dots & a_{2n} \\
\vdots & \vdots & \ddots & \vdots\\
a_{m1} & a_{m2} & \dots & a_{mn}
\end{pmatrix}.
$$
Заметим, однако, что матрица системы линейных уравнений содержит не
всю информацию о системе: мы нигде не использовали правые части этих
уравнений. \dfn{Расширенной матрицей}\index{матрица!расширенная} нашей
системы линейных уравнений
называется таблица
$$
\left(
\begin{array}{cccc|c}
a_{11} & a_{12} & \dots & a_{1n} & b_1\\
a_{21} & a_{22} & \dots & a_{2n} & b_2\\
\vdots & \vdots & \ddots & \vdots & \vdots\\
a_{m1} & a_{m2} & \dots & a_{mn} & b_m
\end{array}
\right)
$$
Вертикальная черта служит для визуального отделения коэффициентов
левой части и правой части системы; иногда мы опускаем ее.

Заметим, что в матрице линейной системы с $m$ уравнениями и $n$
неизвестными содержится $m$ строк и $n$ столбцов; на пересечении
строки с номером $i$ и столбца с номером $j$ стоит элемент $a_{ij}$. В
расширенной матрице такой системы $m$ строк и $n+1$ столбец.

Часто мы будем записывать матрицу так: $(a_{ij})_{\substack{1\leq
    i\leq m\\1\leq j\leq n}}$: в этой матрице $m$ строк, $n$ столбцов,
и на пересечении $i$-ой строки и $j$-го столбцы стоит элемент
$a_{ij}$. Если размер матрицы подразумевается известным, мы будем
сокращать эту запись до $(a_{ij})$.

Среди множества преобразований систем линейных уравнений выделяют три
несложных типа преобразований, играющих важную роль в нахождении
решений.

\begin{enumerate}
\item Элементарное преобразование первого типа: прибавить к $i$-му
  уравнению $j$-ое уравнение, умноженное на некоторый элемент
  $\lambda\in R$. Иными словами, $i$-ое уравнение
$$
a_{i1}x_1+a_{i2}x_2+\dots+a_{in}x_n=b_i
$$
заменяется при этом преобразовании на уравнение
$$
(a_{i1}+\lambda a_{j1})x_1+(a_{i2}+\lambda a_{j2})x_2+\dots
+ (a_{in}+\lambda a_{jn})x_n=b_i+\lambda b_j,
$$
а все остальные уравнения остаются неизменными.
\item Элементарное преобразование второго типа: поменять местами
  $i$-ое уравнение и $j$-ое уравнение. Остальные уравнения при этом
  остаются неизменными.
\item Элементарное преобразование третьего типа: домножить $i$-ое
  уравнение на обратимый элемент кольца $R$. Иными словами, для
  некоторого $\eps\in R^*$ уравнение под номером $i$
$$
a_{i1}x_1+a_{i2}x_2+\dots+a_{in}x_n=b_i
$$
заменяется на уравнение
$$
\eps a_{i1}x_1+\eps a_{i2}x_2+\dots+\eps a_{in}x_n=\eps b_i,
$$
а остальные уравнения не меняются.
\end{enumerate}
Несложно понять, как указанные преобразования меняют матрицу системы и
расширенную матрицу системы: элементарное преобразование первого типа
прибавляет к $i$-ой строке $j$-ую, умноженную на $\lambda\in R$;
второго типа~--- меняет местами строки с номерами $i$ и $j$; третьего
типа~--- домножает все элементы $i$-ой строки на $\eps\in R^*$.

Мы будем использовать следующие условные обозначения для элементарных
преобразований: преобразование первого типа, прибавляющее к $i$-ой
строке $j$-ую, умноженную на $\lambda$, обозначается через
$T_{ij}(\lambda)$ (здесь $1\leq i,j\leq m$, $i\neq j$, $\lambda\in
R$); преобразование второго типа, меняющее местами строки с номерами
$i$ и $j$, обозначается через $S_{ij}$ (здесь $1\leq i,j\leq m$,
$i\neq j$), а преобразование третьего
типа, домножающее $i$-ую строку на $\eps$, обозначается через
$D_i(\eps)$ (здесь $1\leq i\leq m$, $\eps\in R^*$). Через некоторое
время эти символы превратятся в обозначения совершенно конкретных
объектов, связанных с соответствующими преобразованиями.

Сразу же заметим, что каждое элементарное преобразование {\it
  обратимо}: это означает, что для каждого элементарного
преобразования найдется другое элементарное преобразование (называемое
{\it обратным} такое, что
применение двух этих преобразований подряд (в любом порядке) к системе
не меняет ее. Действительно, сразу видно, что для преобразования
третьего типа $D_i(\eps)$ обратным является $D_i(\eps^{-1})$, а для
преобразования второго типа $S_{ij}$ обратным является оно
само. Наконец, несложная выкладка показывает, что для преобразования
первого типа $T_{ij}(\lambda)$ обратным является преобразование
$T_{ij}(-\lambda)$: последовательное применение этих двух
преобразований сначала прибавляет к $i$-му уравнению исходной системы
$j$-ое, умноженное на $\lambda$, а потом прибавляет $j$-ое, умноженное
на $-\lambda$ (или наоборот), поэтому $i$-ое уравнение в итоге не
изменяется (а остальные~--- тем более).

\begin{lemma}\label{lem_elementary_transformations}
Элементарные преобразования не меняют множества (всех) решений
системы.
\end{lemma}
\begin{proof}
По замечанию выше, каждое элементарное преобразование обратимо;
поэтому достаточно доказать, что множество решений системы не
уменьшается: если набор $(c_1,\dots,c_n)$ является решением системы,
то он будет являться и решением системы, полученной из нее
элементарным преобразованием. Это очевидно для преобразований второго
и третьего типов, и несложно проверить для преобразований первого
типа.
\end{proof}

\subsection{Метод Гаусса}
\literature{[F], гл. IV, \S~4, п. 5; [K1], гл. 1, \S~3, п. 3.}

Сейчас мы опишем, как решать произвольную систему линейных
уравнений {\it над полем}. Основная идея состоит в том, чтобы сначала
привести систему
к удобному для решения виду~--- {\it ступенчатому}. Алгоритм
приведения произвольной системы к ступенчатому виду называется {\it
  методом Гаусса}. Мы дадим строгое определение ступенчатого вида
после того, как опишем этот алгоритм.

Как обычно, нам будет удобно работать не с системой линейных
уравнений, а с ее [расширенной] матрицей: метод Гаусса состоит в
последовательном применении к расширенной матрице системы элементарных
преобразований, после чего матрица становится {\it ступенчатой}, и
все решения соответствующей системы легко выписать; по
лемме~\ref{lem_elementary_transformations} полученное множество
решений будет и множеством решений исходной системы.

Итак, пусть $(a_{ij})$~--- матрица над полем $k$ размера $m\times n$.
Мы будем изучать ее столбцы
последовательно, слева направо. Возьмем первый столбец. Возможны два
варианта: либо он состоит из одних нулей, либо в нем найдется
ненулевой элемент. Если столбец состоит из одних нулей, мы пропускаем
его и переходим к следующему столбцу, пока не найдем какой-нибудь
столбец с ненулевым элементом. Пусть, наконец, в столбце с номером
$j_1$ нашелся ненулевой элемент (если такого столбца нет, то наша
матрица нулевая, и алгоритм завершен).

Для начала поставим этот ненулевой элемент на первое
место в столбце посредством элементарного преобразования второго
типа. Теперь мы сделаем все остальные элементы нашего столбца нулевыми
с помощью элементарных преобразований первого типа. Делается это так:
теперь мы считаем, что элемент $a_{1,j_1}$ не равен нулю; если
какой-нибудь элемент $a_{i,j_1}$ первого столбца также не равен нулю, то
прибавим к $i$-ой строчке первую, умноженную на
$-a_{i,j_1}/a_{1,j_1}$. Иными словами, проведем элементарное преобразование
$T_{i,j_1}(-a_{i,j_1}/a_{1,j_1})$. При этом изменится только $i$-ая строчка, и
ее первый элемент станет равным
$a_{i,j_1}+a_{1,j_1}\cdot(-a_{i,j_1}/a_{1,j_1})=0$. Проделаем это для всех
ненулевых элементов первого столбца. Заметим, что здесь мы
использовали тот факт, что ненулевой элемент $a_{1,j_1}$ обратим, то
есть, что $k$ является полем.

Теперь столбец с номером $j_1$ нашей матрицы содержит единственный
ненулевой элемент $a_{1,j_1}$ (а все столбцы, стоящие слева от него,
нулевые).
Мысленно забудем про первую строчку нашей матрицы и про все столбцы
вплоть до столбца с номером $j_1$ и повторим нашу операцию: теперь мы
берем столбец с номером $j_1+1$ и ищем в нем ненулевой элемент, не
принимая во внимание первую строчку. Если во всех позициях (кроме,
может быть, первой) этого столбцы стоят нули, мы двигаемся дальше
вправо, пока не находим, наконец, столбец с номером $j_2$, в котором
стоит какой-нибудь ненулевой элемент не в первой строчке. Посредством
элементарного преобразования второго типа можно поставить этот
ненулевой элемент на второе место, а затем, с помощью элементарных
преобразований первого типа, добиться того, что все элементы ниже его
станут нулями. Заметим, что первая строчка в этих преобразованиях уже
никак не участвует, поэтому про нее и можно забыть. Кроме того, в
столбцах с номерами $1,\dots,j_1$ стоят нули на тех позициях, которые
затрагиваются этими преобразованиями, поэтому они не изменяются. Итак,
в столбце с номером $j_2$ теперь стоит неизвестно что на первой
позиции, ненулевой элемент $a_{2,j_2}$ на второй позиции, и $0$ на
остальных позициях. Далее, конечно, мы продолжаем ту же процедуру,
забывая про первый две строчки и про столбцы с номерами
$1,\dots,j_2$. Заметим, что мы обязаны двигаться вправо: $1\leq
j_1<j_2<j_3<\dots$, поэтому этот процесс остановится.

Полученная матрица
$$
\left(
\begin{array}{ccccccccccccccccccc}
0&\dots&0&a_{1,j_1}&*& \dots & * & * & * & \dots &*&*&*&\dots&*&*&*&\dots&*\\
0 & \dots & 0 & 0 & 0 & \dots & 0 & a_{2,j_2} & * & \dots &*&*&*&\dots&*&*&*&\dots&*\\
0 & \dots & 0 & 0 & 0 & \dots & 0 & 0 & 0 & \dots & 0 & a_{3,j_3}&*&\dots&*&*&*&\dots&*\\ 
\vdots&\ddots&\vdots&\vdots&\vdots&\ddots&\vdots&\vdots&\vdots&\ddots&\vdots&\vdots&\vdots&\ddots&\vdots&\vdots&\vdots&\ddots&\vdots\\
0&\dots&0&0&0&\dots&0&0&0&\dots&0&0&0&\dots&0&a_{s,j_s}&*&\dots&*\\
0&\dots&0&0&0&\dots&0&0&0&\dots&0&0&0&\dots&0&0&0&\dots&0\\
\vdots&\ddots&\vdots&\vdots&\vdots&\ddots&\vdots&\vdots&\vdots&\ddots&\vdots&\vdots&\vdots&\ddots&\vdots&\vdots&\vdots&\ddots&\vdots\\
0&\dots&0&0&0&\dots&0&0&0&\dots&0&0&0&\dots&0&0&0&\dots&0\\
\end{array}\right)
$$
и называется ступенчатой; теперь мы готовы дать
формальное определение.

\begin{definition}
Матрица $(a_{ij})_{\substack{1\leq i\leq m\\1\leq j\leq n}}$
называется \dfn{ступенчатой}\index{матрица!ступенчатая}, если существует некоторая
последовательность индексов $1\leq j_1<j_2<\dots<j_s\leq n$ такая, что
\begin{itemize}
\item $a_{i,j_i}\neq 0$ для любого $i=1,\dots,s$;
\item $a_{i,j}=0$ при $j<j_i$;
\item $a_{i,j}=0$ для любого $j$ при $i>s$.
\end{itemize}
\end{definition}

% 10.12.2014

Иными словами, в ступенчатой матрице имеются строки $1,\dots,s$ такие,
что в строке с номером $i$ первый ненулевой элемент стоит в позиции
$(i,j_i)$, а все строки с номерами $s+1,\dots,m$~--- нулевые.

Ненулевые элементы $a_{1,j_1}, a_{2,j_2},\dots,a_{s,j_s}$ в
ступенчатой матрице $(a_{ij})$ мы будем
называть \dfn{ведущими}\index{ведущие элементы}.

Что же нам дает применение метода Гаусса к расширенной матрице системы
линейных уравнений? Напомним, что расширенная матрица системы состоит
из $m$ строк и $n+1$ столбца, где $m$~--- число уравнений, $n$~---
число неизвестных. Самый правый столбец расширенной матрицы несет
особый смысл~--- это правая часть системы. Поэтому сразу рассмотрим
особый случай: предположим, что ведущий элемент оказался в последнем
столбце. Очевидно, что это может быть только последний ведущий элемент
$a_{s,j_s}$. Тогда уравнение с номером $s$ выглядит так:
$0x_1+\dots+0x_n=a_{s,j_s}$, и $a_{s,j_s}\neq 0$. Очевидно, что это
уравнение не имеет решений, поэтому и вся система не имеет решений.

Теперь можно считать, что $j_s<n+1$, и всем ведущим элементам
соответствуют переменные $x_{j_1},\dots,x_{j_s}$. {\it Все остальные}
переменные мы будем называть \dfn{свободными}\index{свободные
  переменные}, а переменные
$x_{j_1},\dots,x_{j_s}$~--- \dfn{зависимыми}\index{зависимые
  переменные}. Теперь мы утверждаем,
что множество решений полученной системы выглядит так: свободные
переменные могут принимать произвольные значения, и, как только они
заданы, значения зависимых переменных определяются однозначным
образом.

Действительно, предположим, что мы задали произвольные значения
свободных переменных. Пойдем по уравнениям снизу вверх и начнем
выражать значения зависимых переменных. Заметим, что уравнения с
номерами $s+1,\dots,m$ фактически имеют вид $0=0$, поэтому не влияют
на множество решений системы, и их можно выбросить. Последнее
уравнение имеет вид $a_{s,j_s}x_{j_s}+\dots=b_s$, и значения всех
переменных в левой части, кроме $x_{j_s}$, уже заданы. Деля на
ненулевой элемент $a_{s,j_s}$ и перенося <<многоточие>> в правую
часть, получаем выражение для зависимой переменной $x_{j_s}$. Теперь
возьмем предпоследнее уравнение:
$a_{s-1,j_{s-1}}x_{j_{s-1}}+\dots=b_{s-1}$; мы уже знаем значения всех
переменных в левой части, кроме $x_{j_{s-1}}$, поэтому аналогичным
образом получаем выражение для следующей зависимой переменной,
$x_{j_{s-1}}$. Продолжая этот процесс, мы дойдем и до первой строчки,
выразив значение $x_{j_1}$.

Итак, если заданы значения свободных переменных, то значения свободных
переменных определяются однозначно. С другой стороны, значения
свободных переменных могут быть совершенно произвольными, и
приведенный алгоритм утверждает, что найдется решение с такими
значениями свободных переменных. Иными словами, мы установили
взаимно-однозначное соответствие между множеством решений нашей
системы и множеством произвольных наборов значений независимых
переменных.

\subsection{Операции над матрицами}
\literature{[F], гл. IV, \S~1; [K1], гл. 3, \S~3, пп. 1--3.}

\begin{definition}
\dfn{Матрицей}\index{матрица} над кольцом $R$ мы будем называть
прямоугольную
таблицу, составленную из элементов кольца $R$. Иными словами, задать
матрицу $A$~--- значит, задать набор элементов $a_{ij}\in R$ для всех
$i,j$ таких, что $1\leq i\leq m$, $1\leq j\leq n$. Эти элементы
называются \dfn{коэффициентами}\index{коэффициенты матрицы} матрицы
$A$ и мы пишем $A=(a_{ij})$.
При этом мы будем
изображать такую матрицу в виде таблицы из $m$ строк и $n$ столбцов, в
которой на пересечении $i$-й строки и $j$-го столбца стоит элемент
$a_{ij}$. Будем говорить, что $A$ является матрицей $m\times n$;
множество всех матриц $m\times n$ над кольцом $R$
обозначается через $M(m,n,R)$. Если
$m=n$ (число строк совпадает с числом столбцов), матрица называется
\dfn{квадратной}\index{матрица!квадратная}; мы будем писать $M(n,R)$
вместо $M(n,n,R)$. При этом $n$ называется
\dfn{порядком}\index{порядок!квадратной матрицы} квадратной матрицы
из $M(n,R)$.
\end{definition}

Элемент, стоящий в матрице $A$ на пересечении $i$-й строки и $j$-го
столбца мы часто будем обозначать через $A_{ij}$; будем говорить, что
в матрице $A$ элемент $A_{ij}$ \dfn{стоит на позиции
  $(i,j)$}\index{позиция элемента в матрице}.

Введем основные операции над матрицами. Если $A=(a_{ij})$,
$B=(b_{ij})$~--- две матрицы одинакового размера $m\times n$, определим их сумму
$A+B$ как матрицу, у которой на позиции $(i,j)$ стоит $a_{ij}+b_{ij}$.
Иными словами, $(A+B)_{ij}=A_{ij}+B_{ij}$ для всех $1\leq i\leq m$,
$\leq i\leq n$.
Таким образом, сложение матриц происходит {\it покомпонентно}.

Гораздо интереснее выглядит умножение матриц.
Пусть $A\in M(m,n,R)$, $B\in M(n,p,R)$~--- обратите внимание, что
число столбцов первой матрицы равно числу строк второй матрицы.
Тогда их произведением $AB$ называется матрица размера $m\times p$, у
которой на позиции $(i,k)$ стоит $\sum_{j=1}^nA_{ij}B_{jk}$. Иными
словами, $(AB)_{ik}=\sum_{j=1}^nA_{ij}B_{jk}$. Обратите внимание, что
при фиксированных $i$ и $k$ элементы $A_{ij}$ пробегают строку матрицы
$A$ с номером $i$, а элементы $B_{jk}$ пробегают столбец матрицы $B$ с
номером $k$. То есть, для того, чтобы получить элемент, стоящий в
матрице $AB$ на позиции $(i,k)$, нужно взять $i$-ю строку матрицы $A$,
$k$-й столбец матрицы $B$, и сформировать сумму произведений
соответствующих элементов этой строки и этого столбца; по условию на
размер матриц $A$ и $B$ они имеют одинаковую длину.

Определим также результат умножения скаляра (элемента кольца $R$) на
матрицу над $R$: пусть $\lambda\in R$, $A\in M(m,n,R)$. Рассмотрим
матрицу, в которой на позиции $(i,j)$ стоит $\lambda A_{ij}$; мы будем
обозначать ее через $\lambda A$. То есть, при умножении матрицы $A$ на
скаляр $\lambda$ каждый элемент матрицы $A$ умножается на $\lambda$
(здесь мы предполагаем, что кольцо $R$ коммутативно, поэтому неважно,
с какой стороны происходит умножение).

Наконец, еще одна важная операция~---
\dfn{транспонирование}\index{транспонирование}\index{матрица!транспонированная}
матрицы. Пусть $A\in M(m,n,R)$. Определим матрицу $A^T\in M(n,m,R)$
так: у нее в позиции $(j,i)$ стоит элемент $A_{ij}$. Такая матрица
называется матрицей, транспонированной к матрице $A$. Неформально
говоря, это матрица, полученная из матрицы $A$ <<симметрией>>
относительно главной диагонали. При этом строки с номерами
$1,2,\dots,m$ матрицы $A$ становятся столбцами с номерами
$1,2,\dots,m$ матрицы $A^T$; аналогично, столбцы матрицы $A$
превращаются в строки матрицы $A^T$.

Теперь сформулируем свойства введенных операций.

\begin{theorem}[Свойства операций над матрицами]\label{thm_matrix_operations_properties}
Следующие тождества выполняются для любых матриц $A,B,C$ над коммутативным
кольцом $R$ и для любых $\lambda,\mu\in R$,
если определены результаты всех входящих в них операций:
\begin{enumerate}
\item $A+(B+C)=(A+B)+C$ (ассоциативность сложения);
\item пусть $0$~--- матрица, все коэффициенты которой нулевые; тогда
  $A+0=0+A=A$ (нейтральный элемент относительно сложения);
\item для любой матрицы $A$ найдется матрица $-A$ такая, что
  $A+(-A)=(-A)+A=0$ (противоположный элемент);
\item $A+B=B+A$ (коммутативность сложения).
\item $(AB)C=A(BC)$ (ассоциативность умножения);
\item $A(B+C)=AB+AC$ (левая дистрибутивность);
\item $(B+C)A=BA+CA$ (правая дистрибутивность);
\item $\lambda(A+B)=\lambda A+\lambda B$ (левая дистрибутивность умножения
  на скаляр);
\item $(\lambda+\mu)A=\lambda A + \mu A$ (правая дистрибутивность
  умножения на скаляр);
\item $(\lambda A)B=\lambda (AB)=A(\lambda B)$;
\item $(\lambda\mu)A=\lambda(\mu A)$;
\item $(A+B)^T=A^T+B^T$;
\item\label{property_mult_transpose} $(AB)^T=B^TA^T$.
\end{enumerate}
\end{theorem}
Поясним формулировку <<\dots если определены результаты всех входящих
в них операций>>: мы можем сложить две матрицы только в том случае,
если они имеют одинаковый размер, и перемножить две матрицы только в
том случае, если количество столбцов первой матрицы совпадает с
количеством строк второй матрицы. Поэтому, скажем, тождество
$A+(B+C)=(A+B)+C$ выполняется для любых $A,B,C\in M(m,n,R)$, тождество
$(AB)C=A(BC)$~--- для любых $A\in M(m,n,R)$, $B\in M(n,p,R)$, $C\in
M(p,q,R)$, тождество $A(B+C)=AB+AC$~--- для любых $A\in M(m,n,R)$ и
$B,C\in M(n,p,R)$, и так далее.

\begin{proof}
Напоминаем, что через $A_{ij}$ мы обозначаем элемент матрицы $A$,
стоящий в позиции $(i,j)$. Для того, чтобы проверить равенство двух
матриц, достаточно проверить, что они имеют одинаковый размер и что
элементы, стоящие в соответствующих позициях этих матриц,
равны. Мы займемся именно проверкой поэлементного равенства, оставив
читателю [тривиальную] проверку равенства размеров.
\begin{enumerate}
\item
  $(A+(B+C))_{ij}=A_{ij}+(B+C)_{ij} = A_{ij}+(B_{ij}+C_{ij}) =
  (A_{ij}+B_{ij})+C_{ij} = (A+B)_{ij}+C_{ij}=((A+B)+C)_{ij}$; здесь мы
  воспользовались ассоциативностью сложения в кольце $R$.
\item $(A+0)_{ij} = A_{ij}+0_{ij} = A_{ij}+0 = A_{ij}=0+A_{ij} =
  0_{ij}+A_{ij} = (0+A)_{ij}$.
\item Составим матрицу $-A$ из элементов $-A_{ij}$, то есть, положим
  $(-A)_{ij} = -A_{ij}$. Тогда
  $(A+(-A))_{ij}=A_{ij}+(-A)_{ij}=A_{ij}-A_{ij}=0$, откуда $A+(-A)=0$;
  аналогично, $(-A)+A=0$.
\item $(A+B)_{ij} = A_{ij}+B_{ij} = B_{ij}+A_{ij} = (B+A)_{ij}$,
  поскольку сложение в $R$ коммутативно.
\item Пусть $A\in M(m,n,R)$, $B\in M(n,p,R)$, $C\in M(p,q,R)$. Тогда
  $$((AB)C)_{il} = \sum_{k=1}^p(AB)_{ik}C_{kl} =
  \sum_{k=1}^p\sum_{j=1}^nA_{ij}B_{jk}C_{kl};$$ с другой стороны,
  $$(A(BC))_{il} = \sum_{j=1}^nA_{ij}(BC)_{jl} =
  \sum_{j=1}^nA_{ij}\sum_{k=1}^pB_{jk}C_{kl} =
  \sum_{j=1}^n\sum_{k=1}^pA_{ij}B_{jk}C_{kl}.$$ Получившиеся суммы
  отличаются только изменением порядка суммирования.
\item Пусть $A\in M(m,n,R)$, $B\in M(n,p,R)$. Тогда
  $$(A(B+C))_{ik} = \sum_{j=1}^nA_{ij}(B+C)_{jk} =
  \sum_{j=1}^n(A_{ij}B_{jk}+A_{ij}C_{jk})$$ и
  $$(AB+AC)_{ik} = (AB)_{ik}+(AC)_{ik} = \sum_{j=1}^nA_{ij}B_{jk} +
  \sum_{j=1}^nA_{ij}C_{jk} = \sum_{j=1}^n(A_{ij}B_{jk}+A_{ij}C_{jk}).$$
\item Доказательство совершенно аналогично доказательству предыдущего
  пункта.
\item $(\lambda(A+B))_{ij} = \lambda(A+B)_{ij} =
  \lambda(A_{ij}+B_{ij}) = \lambda A_{ij}+\lambda B_{ij} =
  (\lambda A)_{ij}+(\lambda B)_{ij}=(\lambda A + \lambda B)_{ij}$.
\item $((\lambda+\mu)A)_{ij} = (\lambda+\mu)A_{ij} =
  \lambda A_{ij}+\mu A_{ij} = (\lambda A)_{ij} + (\mu A)_{ij} =
  (\lambda A + \mu A)_{ij}$.
\item Заметим, что $((\lambda A)B)_{ik} = \sum_{j}((\lambda A)_{ij}B_{jk}) =
  \sum_{j}(\lambda A_{ij}B_{jk})$; кроме того,
  $$(A(\lambda B))_{ik} = \sum_j(A_{ij}(\lambda B)_{jk}) =
  \sum_j(A_{ij}\lambda B_{jk}) = \sum_{j}(\lambda A_{ij}B_{jk})$$ и
  $$(\lambda (AB))_{ik} = \lambda (AB)_{ik} = \lambda\sum_j(A_{ij}B_{jk})
  = \sum_j(\lambda A_{ij}B_{jk}).$$
\item $((\lambda\mu)A)_{ij} = (\lambda\mu)A_{ij} = \lambda\mu A_{ij} =
  \lambda(\mu A_{ij}) = \lambda (\mu A)_{ij} = (\lambda(\mu A))_{ij}$.
\item $((A+B)^T)_{ij} = (A+B)_{ji} = A_{ji} + B_{ji} = (A^T)_{ij} +
  (B^T)_{ij} = (A^T + B^T)_{ij}$.
\item $((AB)^T)_{ik} = (AB)_{ki} = \sum_j(A_{kj}B_{ji}) =
  \sum_j((A^T)_{jk}(B^T)_{ij}) = \sum_j((B^T)_{ij}(A^T)_{jk}) = B^TA^T$.
\end{enumerate}
\end{proof}

\begin{definition}
Рассмотрим матрицу размера $n\times n$, у которой в позиции $(i,j)$
стоит $1$, если $i=j$, и $0$, если $i\neq j$. Такая матрица называется
\dfn{единичной матрицей}\index{матрица!единичная} и обозначается через $E_n$ (и часто мы будем
обозначать ее просто через $E$, если размер ясен из контекста). Эта
матрица действительно играет роль нейтрального элемента относительно
умножения, как показывает следующее утверждение.
\end{definition}

\begin{proposition}\label{prop_identity_matrix}
Пусть $A\in M(m,n,R)$. Тогда $E_m\cdot A = A\cdot E_n = A$.
\end{proposition}
\begin{proof}
Заметим, что $(E_m\cdot A)_{ik} = \sum_j (E_m)_{ij} A_{jk}$. В
получившейся сумме матричный элемент $(E_m)_{ij}$ равен $0$ для всех
$j$, кроме $j=i$. Поэтому от суммы остается одно слагаемое,
соответствующее случаю $j=i$, и равное $A_{ik}$. Это выполнено для
всех $i,k$, поэтому $E_m\cdot A = A$. Второе равенство доказывается
аналогично.
\end{proof}

\begin{remark}\label{rem:matrix_multiplication_properties}
Заметим, что для квадратных матриц фиксированного размера (то есть,
для элементов $M(n,R)$) свойства 1--7 из
теоремы~\ref{thm_matrix_operations_properties} и свойство единичных
матриц из предложения~\ref{prop_identity_matrix} означают, что эти
матрицы образуют ассоциативное кольцо с единицей. Это кольцо $M(n,R)$
называется \dfn{кольцом квадратных матриц}\index{кольцо!квадратных
  матриц} порядка $n$.
Отметим, что это кольцо не является коммутативным при $n\geq 2$:
$$
\begin{pmatrix}0 & 1\\0 & 0\end{pmatrix}\cdot
\begin{pmatrix}0 & 0\\1 & 0\end{pmatrix} = 
\begin{pmatrix}1 & 0\\0 & 0\end{pmatrix}\neq
\begin{pmatrix}0 & 0\\0 & 1\end{pmatrix} = 
\begin{pmatrix}0 & 0\\1 & 0\end{pmatrix}\cdot
\begin{pmatrix}0 & 1\\0 & 0\end{pmatrix}.
$$
Напомним, что элемент $a$ произвольного ассоциативного кольца $A$ с
единицей называется {\it обратимым}, если найдется элемент $b\in A$
такой, что $ab=ba=1$ в $A$. Такой элемент $b$ обозначается через
$a^{-1}$ и называется {\it обратным} к $a$. В полном соответствии с
этим, квадратная матрица $A\in M(n,R)$ называется
\dfn{обратимой}\index{матрица!обратимая},
если найдется матрица, обозначаемая через $A^{-1}\in M(n,R)$, такая,
что $A\cdot A^{-1} = A^{-1}\cdot A = E_n$. При этом, как и в
произвольном ассоциативном кольце с единицей, для обратимой матрицы
$A$ выполнено $(A^{-1})^{-1}=A$, а для набора обратимых матриц
$A_1,\dots,A_s$ выполнено $(A_1\cdot A_2\cdot\dots\cdot A_s)^{-1} =
A_s^{-1}\cdot\dots\cdot A_2^{-1}\cdot A_1^{-1}$.
\end{remark}

Упомянем еще одно важное свойство, связывающее обратимость и
транспонирование.

\begin{proposition}
Если матрица $A\in M(n,R)$ обратима, то и матрица $A^T$ обратима,
причем $(A^T)^{-1} = (A^{-1})^T$.
\end{proposition}
\begin{proof}
Пользуясь свойством~(\ref{property_mult_transpose}) из
теоремы~\ref{thm_matrix_operations_properties}, получаем
$A^T\cdot(A^{-1})^T = (A^{-1}\cdot A)^T = (E_n)^T$. Осталось заметить,
что $(E_n)^T=E_n$, поскольку из определения единичной матрицы легко
видеть, что $(E_n)_{ij}=(E_n)_{ji}$ для всех $i,j$. Равенство
$(A^{-1})^T\cdot A^T=E_n$ проверяется аналогично.
\end{proof}

\begin{remark}
Кольцо матриц $M(n,R)$ не является полем при $n\geq 2$, поскольку в
нем есть делители нуля. Например, пусть $A=\begin{pmatrix}0 & 1\\0 &
  0\end{pmatrix}\in M(2,R)$; тогда $A\cdot A=\begin{pmatrix}0 & 0\\0 &
  0\end{pmatrix}$. Поэтому матрица $A$ никак не может быть обратимой в
$M(2,R)$. Нетрудно придумать аналогичный пример в $M(n,R)$ для любого
$n\geq 2$.
\end{remark}

Удобно конструировать матрицы из маленьких кусочков: обозначим через
$e_{ij}$ матрицу из $M(m,n,R)$, у которой в позиции $(i,j)$ стоит $1$,
а во всех остальных позициях стоит $0$. Заметим, что $m$ и $n$ в наше
обозначение $e_{ij}$ не входят~--- мы подразумеваем, что всегда из
контекста ясно, какого размера матрицы рассматриваются (если это
вообще важно).
Любую матрицу $A=(a_{ij})\in M(m,n,R)$ тогда можно представить в виде
$A=\sum_{i,j}a_{ij}e_{ij}$. Например, для единичной матрицы имеем
$E_n=e_{11}+e_{22}+\dots+e_{nn}$.
Матрицы $e_{ij}$ называются \dfn{матричными единицами}\index{матричная
  единица} (не путать с
{\it единичными матрицами}!)

Как перемножаются матричные единицы? В произведении $e_{ij}\cdot
e_{kl}$ ненулевые элементы могут стоять только в $i$-ой строчке
(поскольку все строчки матрицы $e_{ij}$, кроме $i$-ой, нулевые), и
только в $l$-ом столбце (поскольку все столбцы матрицы $e_{kl}$, кроме
$l$-го, нулевые). Поэтому произведение $e_{ij}\cdot e_{kl}$ может
отличаться от нуля только в позиции $e_{il}$. Внимательное
рассмотрение произведения $i$-ой строчки матрицы $e_{ij}$ на $l$-й
столбец матрицы $e_{kl}$ показывает, что
$$e_{ij}\cdot e_{kl}=\begin{cases}e_{il}, &\text{если }j=k;\\ 0,
  &\text{если }j\neq k.\end{cases}$$

Наконец, докажем полезный критерий равенства двух матриц.
\begin{proposition}\label{prop:equal-matrices}
Пусть $A,B\in M(m,n,R)$. Следующие утверждения равносильны:
\begin{enumerate}
\item $A = B$;
\item $uA = uB$ для всех $u\in M(1,m,R)$;
\item $Av = Bv$ для всех $v\in M(n,1,R)$;
\item $uAv = uBv$ для всех $u\in M(1,m,R)$, $v\in M(n,1,R)$.
\end{enumerate}
\end{proposition}
\begin{proof}
Пусть $A = (a_{ij})$, $B = (b_{ij})$.
Очевидно, что из первого утверждения следуют остальные.
Докажем, что $(2)\Rightarrow (1)$.
Возьмем в качестве $u$ матрицу $e_{1,i}$. Тогда
$uA = \begin{pmatrix} a_{i1} & a_{i2} & \dots & a_{in} \end{pmatrix}$,
$uB = \begin{pmatrix} b_{i1} & b_{i2} & \dots & b_{in} \end{pmatrix}$,
и из их равенства следует равенство $i$-х строчек матриц $A$ и $B$.
Подставляя $i=1,\dots,m$, получаем, что $A=B$.

Совершенно аналогично доказывается, что $(3)\Rightarrow (1)$.
Наконец, покажем, что $(4)\Rightarrow (1)$.
Достаточно заметить, что если $u = e_{1,i}$ и $v = e_{j,1}$
то $uAv = a_{ij}$ и $uBv = b_{ij}$; подставляя всевозможные пары
$(i,j)$, получаем, что $A = B$.
\end{proof}

% 17.12.2014

\subsection{Матрицы элементарных преобразований}
\literature{[K1], гл. 1, \S~3, п. 6.}

В качестве первого применения операций над матрицами мы истолкуем
элементарные преобразования, введенные в
разделе~\ref{subsection_linear_systems}, как домножения на матрицы
определенного вида.

Для $i\neq j$ ($1\leq i,j\leq n$) и $\lambda\in R$ определим
$T_{ij}(\lambda) = E_n + \lambda e_{ij}$. Это матрица, которая
отличается от единичной матрицы лишь в одной позиции $(i,j)$, в
которой стоит $\lambda$.
Напомним, что по этим же данным $i,j,\lambda$ мы определили
элементарное преобразование первого типа как прибавление к $i$-й
строке матрицы ее $j$-ой строки, умноженной на $\lambda$. Оказывается,
проведение этого элементарного преобразования над матрицей $A\in
M(n,m,R)$ равносильно умножению матрицы $A$ слева на
$T_{ij}(\lambda)$.
Действительно, пусть $A=(a_{ij})\in M(n,m,R)$. Посмотрим на матрицу
$T_{ij}(\lambda)A$. Поскольку матрица $T_{ij}$ отличается от матрицы
$E_n$ только в $i$-й строке, произведение $T_{ij}(\lambda)A$
отличается от матрицы $A$ только в $i$-й строке. Значит, нам осталось
только перемножить $i$-ю строку матрицы $T_{ij}(\lambda)$ на $A$, и
записать результат в $i$-ю строку результата. В $i$-й строке матрицы
$T_{ij}(\lambda)$ лишь два элемента отличны от нуля: элемент в позиции
$i$ равен 1, а элемент в позиции $j$ равен $\lambda$. При умножении на
$k$-й столбец матрицы $A$, получаем следующее:
$$
\left(\begin{matrix}0 & \cdots & 1 & \cdots & \lambda & \cdots & 0\end{matrix}\right)\cdot
\left(\begin{matrix} a_{1k} \\ \vdots \\ a_{ik} \\ \vdots \\ a_{jk} \\
  \vdots \\ a_{nk}\end{matrix}\right) = a_{ik} + \lambda a_{jk}
$$
Это происходит в каждом столбце матрицы $A$; поэтому $i$-я строка
произведения $T_{ij}(\lambda)$ равна $(\begin{matrix}a_{i1}+\lambda
  a_{j1} & \cdots & a_{in}+\lambda a_{jn}\end{matrix})$, то есть,
равна сумме $i$-й строки матрицы $A$ и $j$-й строки матрицы $A$,
умноженной на $\lambda$.

Теперь разберемся с элементарными преобразованиями второго
типа. Для индексов $i\neq j$ рассмотрим матрицу $S_{ij}\in M(n,R)$, которая
отличается от единичной матрицы $E_n$ перестановкой строк с номерами
$i$ и $j$. Таким образом, $S_{ij}$ отличается от $E_n$ в четырех
позициях: в позициях $(i,i)$ и $(j,j)$ стоят $0$ (вместо $1$), а в позициях $(i,j)$
и $(j,i)$ стоят $1$ (вместо $0$). Иными словами,
$S_{ij}=E_n-e_{ii}-e_{jj}+e_{ij}+e_{ji}$.
Покажем, что умножение матрицы $A$ на $S_{ij}$ слева равносильно
элементарному преобразованию второго типа матрицы $A$~--- перестановке
$i$-ой и $j$-ой строчки.
Действительно, произведение $S_{ij}A$ отличается от матрицы $A$ только
в строчках с номерами $i$ и $j$: $i$-ая строчка равна произведению
строчки $(\begin{matrix} 0 & \cdots & 0 & 1 & 0 & \cdots &
  0\end{matrix})$ (где $1$ стоит на $j$-м месте) на матрицу $A$, то
есть, $j$-ой строчке матрицы $A$. Аналогично, $j$-ая строчка
произведения $S_{ij}A$ равна произведению строчки $(\begin{matrix} 0 &
  \cdot & 0 & 1 & 0 & \cdots & 0\end{matrix})$ (где $1$ стоит на $i$-м
месте) на матрицу $A$, то есть, $i$-ой строчке матрицы $A$.

Наконец, для индекса $i$ и обратимого элемента $\eps\in R^*$
рассмотрим матрицу $D_i(\eps)\in M(n,R)$, которая отличается от
единичной матрицы $E_n$ лишь в позиции $(i,i)$, где стоит $\eps$. То
есть, $D_i(\eps)=E_n+(\eps-1)e_{ii}$. Покажем, что умножение матрицы
$A$ на $D_i(\eps)$ слева равносильно элементарному преобразованию
третьего типа матрицы $A$~--- умножению $i$-ой строчки на
$\eps$. Действительно, матрица $D_i(\eps)\cdot A$ отличается от $A$
только в $i$-й строчке, и $i$-ая строчка матрицы $D_i(\eps)\cdot A$
равна произведению $(\begin{pmatrix}0 & \cdots & \eps & \cdots &
  0\end{pmatrix})\cdot A=\eps(\begin{pmatrix}0 & \cdots & 1 & \cdots
  & 0\end{pmatrix})\cdot A$, что равно произведению $\eps$ и $i$-ой
строчки матрицы $A$.

Таким образом, мы истолковали элементарные преобразования над строками
матрицы как домножения слева на несложные матрицы $T_{ij}(\lambda)$,
$S_{ij}$ и $D_i(\eps)$:
\begin{itemize}
\item умножение на $T_{ij}(\lambda)$ слева соответствует прибавлению к
  $i$-ой строчке $j$-ой строчки, умноженной на $\lambda$;
\item умножение на $S_{ij}$ слева соответствует перестановке $i$-ой и
  $j$-ой строчек;
\item умножение на $D_i(\eps)$ слева соответствует умножению $i$-ой
  строчки на $\eps$.
\end{itemize}
 Применяя транспонирование (с учетом свойства
$(AB)^T=B^TA^T$), получаем, что элементарные преобразования над {\it
  столбцами} матрицы соответствуют домножения {\it справа} на эти же
матрицы: действительно, при транспонировании строки матриц
превращаются в столбцы, и $(T_{ij}(\lambda))^T=T_{ji}(\lambda)$,
$(S_{ij})^T=S_{ij}$, $(D_i(\eps))^T=D_i(\eps)$. Поэтому
\begin{itemize}
\item умножение на $T_{ij}(\lambda)$ справа соответствует прибавлению к
  $j$-ому столбцу $i$-ого столбца, умноженного на $\lambda$;
\item умножение на $S_{ij}$ справа соответствует перестановке $i$-ого и
  $j$-ого столбцов;
\item умножение на $D_i(\eps)$ справа соответствует умножению $i$-ого
  столбца на $\eps$.
\end{itemize}
Заметим, что обратимость элементарных преобразований соответствует
тому факту, что любая матрица элементарного преобразования
обратима. Так, $(T_{ij}(\lambda))^{-1}=T_{ij}(-\lambda),$
$(S_{ij})^{-1}=S_{ij}$ и $(D_i(\eps))^{-1}=D_i(\eps^{-1}).$ Теперь это
можно проверить непосредственным матричным перемножением.

Теперь мы можем истолковать метод Гаусса как некоторый матричный
факт. Напомним, что метод Гаусса говорит, что с помощью элементарных
преобразований строк можно любую матрицу привести к ступенчатому
виду. В терминах матриц это означает, что для любой матрицы $A\in
M(m,n,k)$ над полем $k$ найдутся матрицы
элементарных преобразований $P_1,\dots,P_s\in M(m,k)$ такие, что
матрица $P_sP_{s-1}\dots P_1A$ является ступенчатой.

Проведем после этого некоторые элементарные преобразования над
{\it столбцами}.
Посмотрим на первую строчку ступенчатой матрицы $A=(a_{ij})$.
$$
\begin{pmatrix}
0 & \dots & 0 & 1 & * & \dots & * \\
0 & \dots & 0 & 0 & * & \dots & * \\
\vdots & \ddots & \vdots & \vdots & \vdots & \ddots & \vdots \\
0 & \dots & 0 & 0 & * & \dots & * 
\end{pmatrix}
$$
Здесь $1$ стоит в позиции $(1,j_1)$, и $a_{1,j}=0$ при
$j<j_1$. Для каждого $j>j_1$ прибавим к $j$-му столбцу столбец с
номером $j_1$, умноженный на $-a_{1,j}$. После этого в позиции $(1,j)$
окажется $a_{1,j}-a_{1,j}=0$. То есть, после таких прибавлений первая
строчка нашей матрицы будет иметь только один ненулевой элемент~---
$1$ в позиции $(1,j_1)$.
Продолжим эту операцию: посмотрим на вторую строчку нашей
матрицы. Если она отличается от нулевой, то там стоит $1$ в некоторой
позиции $(2,j_2)$. Прибавим к $j$-му столбцу столбец с номером $j_2$,
умноженный на $-a_{2,j}$. При этом первая строчка нашей матрицы уже
никак не изменится, а во второй останется лишь один ненулевой
элемент~--- $2$ в позиции $(2,j_2)$. Совершив аналогичное действие для
всех строк нашей матрицы, мы можем добиться того, что наша матрица
отличается от нулевой лишь в позициях $(1,j_1), (2,j_2), \dots
(r,j_r)$, где стоят единицы. После этого перестановкой столбцов можно
добиться того, что эти единицы будут стоять в позициях $(1,1), (2,2),
\dots (r,r)$. Полученная матрица называется \dfn{окаймленной
  единичной}\index{матрица!окаймленная единичная} матрицей. Можно изобразить ее в блочной форме следующим
образом:
$$
\left(\begin{matrix}
E_r & 0\\
0 & 0
\end{matrix}\right)
$$
(здесь $E_r$~--- единичная матрица размера $r\times r$, а нулевые
блоки имеют размеры $r\times (n-r)$, $(m-r)\times r$ и $(m-r)\times
(n-r)$). Конечно, возможно, что $r=0$ и наша матрица нулевая.

Сформулируем то, что было сделано, на матричном языке. Как мы знаем,
элементарные перестановки столбцов соответствуют домножениям нашей
матрицы на матрицы элементарных преобразований справа. Поэтому на
самом деле мы только что доказали следующую теорему:
\begin{theorem}\label{thm_pdq}
Для любой матрицы $A\in M(m,n,k)$ над полем $k$ найдутся матрицы
элементарных преобразований $P_1,\dots,P_t,Q_1,\dots,Q_s$ такие, что
$$
P_tP_{t-1}\dots P_1AQ_1\dots Q_{s-1}Q_s =
\begin{pmatrix}
E_r & 0\\
0 & 0
\end{pmatrix}
$$
для некоторого $r$.
\end{theorem}

\begin{corollary}\label{cor_pdq}
Для любой матрицы $A\in M(m,n,k)$ над полем $k$ существуют обратимые
матрицы $P\in M(m,k)$, $Q\in M(n,k)$ такие, что
$A=PDQ$, где $D=\begin{pmatrix}E_r&0\\0&0\end{pmatrix}\in
M(m,n,k)$~--- окаймленная единичная матрица. Более того, матрицы $P$ и
$Q$ являются произведениями матриц элементарных преобразований.
\end{corollary}
\begin{proof}
По теореме~\ref{thm_pdq} можно записать $P_tP_{t-1}\dots P_1AQ_1\dots
Q_{s-1}Q_s = \begin{pmatrix}E_r&0\\0&0\end{pmatrix}$. 
Обозначим правую часть через $D$~--- это окаймленная единичная матрица.
Все матрицы $P_i$,
$Q_j$ обратимы, поэтому можно последовательно домножить на обратные к
ним с соответствующих сторон и получить равенство
$A=P_1^{-1}\dots P_t^{-1}DQ_s^{-1}\dots Q_1^{-1}$. Положим
теперь $P=P_1^{-1}\dots P_t^{-1}$, $Q=Q_s^{-1}\dots Q_1^{-1}$; матрицы
$P$ и $Q$ обратимы, поскольку они являются произведениями обратимых
матриц. Получим $A=PDQ$, что и требовалось.
\end{proof}

Заметим, что набор матриц $P_1,\dots,P_s,Q_1,\dots,Q_t$ из теоремы не
является однозначно определенным. В то же время (хотя мы этого пока не
доказали) натуральное число $r$, полученной по матрице $A$, определено
однозначно: если взять другие матрицы элементарных преобразований,
после домножения на которые матрица $A$ превратится в окаймленную
единичную, то размер этой единичной матрицы все равно окажется равным
$r$. Это число $r$ является важной характеристикой матрицы $A$ и
называется ее {\it рангом}. Пока что отметим, что для квадратной
матрицы $A$ обратимость равносильна тому, что окаймленная единичная
матрица, к которой приводится матрица $A$, на самом деле является
единичной:
\begin{corollary}\label{cor_invertible_pdq}
Пусть квадратная матрица $A\in M(n,k)$ над полем $k$ представлена в
виде $A=P_sP_{s-1}\dots P_1\left(\begin{matrix}
E_r & 0\\
0 & 0\end{matrix}\right)Q_1\dots Q_{t-1}Q_t$, где $P_i,Q_i$~---
матрицы элементарных преобразований. Тогда обратимость матрицы $A$
равносильна тому, что $r=n$.

Иными словами, матрица $A$ обратима тогда и только тогда, когда ее
можно представить в виде произведения матриц элементарных
преобразований.
\end{corollary}
\begin{proof}
Если $r=n$, то в середине разложения $A$ стоит единичная матрица,
которую можно вычеркнуть, и получится, что $A$ является произведением
матриц элементарных преобразований. Каждая из матриц элементарных
преобразований обратима, а произведение обратимых элементов кольца
обратимо (лемма~\ref{lemma:product_of_invertibles}).

Обратно, предположим, что $A$ обратима. Из равенства
$$A=P_sP_{s-1}\dots P_1\left((\begin{matrix}
E_r & 0\\
0 & 0\end{matrix}\right)Q_1\dots Q_{t-1}Q_t$$ получаем, что
$$P_1^{-1}\dots P_{s-1}^{-1}P_s^{-1}AQ_t^{-1}Q_{t-1}^{-1}\dots
Q_1^{-1}=\left(\begin{matrix} E_r & 0 \\ 0 &
    0\end{matrix}\right).$$ Опять же, в левой части стоит произведение
обратимых матриц, поэтому и матрица в правой части должна быть
обратимой. Но матрица вида $\left(\begin{matrix} E_r & 0 \\
0 & 0\end{matrix}\right)$ может быть обратимой только при
$r=n$. Действительно, если $r<n$, то у нее последняя строка равна
нулю, и в любом произведении этой матрицы на другую последняя строка
также нулевая; поэтому это произведение не может быть единичной
матрицей.
\end{proof}

\subsection{Блочные матрицы}

При работе с большими матрицами часто удобно разбивать их на
кусочки поменьше. Мы видели это в теореме~\ref{thm_pdq}:
окаймленная единичная матрица размера $m\times n$ и ранга $r$
имеет вид
$\begin{pmatrix}
E_r & 0\\ 0 & 0
\end{pmatrix}$.
Вообще, пусть $m = m_1 + \dots + m_s$, $n = n_1 + \dots + n_t$~---
разбиения чисел $m$ и $n$ в сумму $s$ и $t$ слагаемых, соответственно.
Тогда матрица $A\in M(m,n,R)$ разбивается
на $st$ матриц с размерами $m_i\times n_j$: мы группируем
первые $m_1$ строк, следующие $m_2$ строк, и так далее;
а также первые $n_1$ столбцов, следующие $n_2$, и так далее.
Обозначим эти блоки через $x_{ij}\in M(m_i,n_j,R)$ для
$i=1,\dots,s$, $j=1,\dots,t$.
Матрица с выбранными разбиениями множеств строк и столбцов
называется \dfn{блочной матрицей}\index{блочная матрица}
указание разбиений строк и столбцов
называется \dfn{блочной структурой}\index{блочная структура}.
Например, в приведенном выше примере окаймленная
единичная матрица имеет вид
$\begin{pmatrix}
E_r & 0\\ 0 & 0
\end{pmatrix}$.
в соответствии с разбиениями $m = r + (m-r)$, $n = r + (n-r)$.

Пусть теперь $B\in M(m,n,R)$~--- еще одна матрица того же размера,
что и $A$, и пусть для $B$ выбраны те же разбиения
$m = m_1 + \dots + m_s$, $n = n_1 + \dots + n_t$; таким образом,
у матрицы $B$ есть блоки $y_{ij}\in M(m_i,n_j,R)$.
Посмотрим на сумму $A+B$. Это снова матрица из $M(m,n,R)$.
Можно и ее разбить на блоки тем же образом и
получить блоки $z_{ij}\in M(m_i,n_r,R)$.
Нетрудно понять, что $z_{ij} = x_{ij} + y_{ij}$ для всех $i=1,\dots,s$,
$j=1,\dots,t$. Иными словами,
блочные матрицы с одной и той же блочной структурой
складываются <<поблочно>>.

Посмотрим теперь, как перемножаются блочные матрицы.
Пусть $A\in M(m,n,R)$, $B\in M(n,p,R)$, и пусть выбраны разбиения
чисел $m,n,p$: $m = m_1 + \dots + m_s$, $n = n_1 + \dots + n_t$,
$p = p_1 + \dots + p_u$.
Тогда $A$ является блочной матрицей с блоками, скажем,
$x_{ij}\in M(m_i,n_j,R)$, а $B$~--- блочной матрицей с блоками
$y_{jk}\in M(n_j,p_k,R)$.
Их произведение $AB$ лежит в $M(m,p,R$), и его можно рассмотреть
как блочную матрицу в соответствии с указанными разбиениями
чисел $m$ и $p$.
Блоки матрицы $AB$ обозначим через $z_{ik}\in M(m_i,p_k,R)$.
Как блок $z_{ik}$ связан с блоками матриц $A$ и $B$?
Оказывается
$$
z_{ik} = x_{i1}y_{1k} + \dots + x_{it}y_{tk}
= \sum_{j=1}^t x_{ij}y_{jk}.
$$
Таким образом, блочные матрицы можно перемножать <<поблочно>>,
и формула для каждого блока в произведении выглядит точно так же,
как формула для элемента в произведении матриц.
Обратите внимание, однако, что теперь в этом произведении
элементы $x_{ij}$ и $y_{jk}$ являются матрицами, так что
мы должны следить за порядком, в котором они перемножаются.

%%% коллоквиум

%%% 2015

\subsection{Перестановки}\label{subsect:permutations}
\literature{[F], гл. IV, \S~2, п. 2.}

Нам необходимо на время отвлечься от линейной алгебры, чтобы
ввести важное понятие {\it группы перестановок}.
Пусть $X$~--- некоторое
множество. \dfn{Перестановкой}\index{перестановка} на множестве
$X$ называется биекция $X\to X$. Заметим, что любая биекция обратима:
если $\pi\colon X\to X$~--- биекция, то существует и обратное
отображение $\pi^{-1}\colon X\to X$, также являющееся биекцией, такое,
что $\pi\circ\pi^{-1}$ и $\pi^{-1}\circ\pi$ тождественны. Напомним
также, что композиция отображений ассоциативна.

\begin{definition}\label{def_group}
Множество $G$ с бинарной операцией $\circ\colon G\to G$ называется
\dfn{группой}\index{группа}, если выполняются следующие свойства:
\begin{itemize}
\item $a\circ (b\circ c)=(a\circ b)\circ c$ для всех $a,b,c\in G$;
  (\dfn{ассоциативность}\index{ассоциативность!в группе});
\item существует элемент $e\in G$ (\dfn{единичный
    элемент}\index{единичный элемент!в группе}) такой, что
  для любого $a\in G$
  выполнено $a\circ e=e\circ a=a$;
\item для любого $a\in G$ найдется элемент $a^{-1}\in G$ (называемый
  \dfn{обратным}\index{обратный элемент!в группе} к $a$) такой, что
  $a\circ a^{-1}=a^{-1}\circ a=e$.
\end{itemize}
\end{definition}

\begin{definition}\label{def:symmetric_group}
Множество всех биекций из $X$ в $X$ обозначается через $S(X)$ и
называется \dfn{группой перестановок}\index{группа!перестановок}
множества $X$. Тождественное
отображение $\id_X\colon X\to X$ называется \dfn{тождественной
  перестановкой}\index{тождественная перестановка}.
\end{definition}
Как мы заметили выше, $S(X)$ действительно является группой в смысле
определения~\ref{def_group} относительно операции композиции, которая
еще называется \dfn{умножением}\index{умножение перестановок} перестановок.

Зачастую нам не важна природа элементов множества $X$, а важно лишь их
количество, особенно если $X$ конечно. Поэтому для каждого
натурального $n$ можно рассматривать
группу перестановок какого-нибудь выделенного множества из $n$
элементов, например, множества $\{1,\dots,n\}$. Эта группа
обозначается через $S_n$: $S(\{1,\dots,n\})=S_n$.
Элемент $\pi$ группы $S_n$ можно записывать в виде таблицы из двух
строк, в первой строке которой стоят числа $1,\dots,n$ (как правило, в
порядке возрастания), а под каждым
из них стоит его образ $\pi(1),\dots,\pi(n)$:
$$
\pi=\begin{pmatrix} 1 & 2 & \dots & n\\
\pi(1) & \pi(2) & \dots & \pi(n)\end{pmatrix}.
$$
Понятно, что по такой записи однозначно восстанавливается элемент
$\pi$, и обратно, если есть таблица, в первой строке которой стоят
числа $1,\dots,n$, а во второй~--- те же самые числа в каком-то
порядке, то она задает некоторый элемент $S_n$. Такая запись
называется \dfn{табличной записью}\index{табличная запись
  перестановки} перестановки.
Например, группа $S_1$ состоит из одного (тождественного) элемента
$\left(\begin{matrix} 1 \\ 1\end{matrix}\right)$. Группа $S_2$ состоит
из двух элементов: один из них тождественный,
$\begin{pmatrix} 1 & 2\\ 1 & 2\end{pmatrix}$,
а другой переставляет местами $1$ и $2$:
$\begin{pmatrix} 1 & 2\\ 2 & 1\end{pmatrix}$. Группа $S_3$
состоит из шести элементов:
$$
S_3=\left\{\begin{pmatrix} 1 & 2 & 3\\ 1 & 2 & 3\end{pmatrix},
\begin{pmatrix} 1 & 2 & 3\\ 1 & 3 & 2\end{pmatrix},
\begin{pmatrix} 1 & 2 & 3\\ 2 & 1 & 3\end{pmatrix},
\begin{pmatrix} 1 & 2 & 3\\ 2 & 3 & 1\end{pmatrix},
\begin{pmatrix} 1 & 2 & 3\\ 3 & 1 & 2\end{pmatrix},
\begin{pmatrix} 1 & 2 & 3\\ 3 & 2 & 1\end{pmatrix}\right\}.
$$
Несложное комбинаторное рассуждение показывает, что количество
элементов в $S_n$ равно $n!$. Действительно, образом элемента $1$
может быть любой из $n$ элементов множества $\{1,\dots,n\}$, образом
элемента $2$~--- любой из оставшихся $n-1$, и так далее; всего
получаем $n\cdot (n-1)\cdot\dots\cdot 1=n!$ различных вариантов.

Табличная запись позволяет визуализировать перемножение перестановок:
для того, чтобы перемножить перестановки $\pi$ и $\rho$, нужно
записать друг под другом табличные записи $\pi$ и $\rho$, переставить
столбцы в таблице $\rho$ так, чтобы в первой строке оказалась {\it
  вторая} строка таблицы $\pi$, и сформировать ответ из первой строки
верхней таблицы и второй строки нижней таблицы~--- это будет табличной
записью перестановки $\rho\circ\pi$. Обратите внимание на порядок!
Напомним, что мы записываем композицию отображений {\it справа
  налево}: запись $\rho\circ\pi$ означает, что мы сначала применяем
отображение $\pi$, а затем~--- отображение $\rho$.
Это важно, поскольку при $n\geq 3$ умножение в группе $S_n$
некоммутативно. Действительно, рассмотрим перестановки
$\pi=\begin{pmatrix}1 & 2 & 3 \\ 1 & 3 & 2\end{pmatrix}$ и
$\rho=\begin{pmatrix}1 & 2 & 3 \\ 2 & 3 & 1\end{pmatrix}$.
Перемножим их по описанному выше способу:
$$
\rho\circ\pi\colon
\begin{matrix}
\begin{pmatrix}1 & 2 & 3 \\ 1 & 3 & 2\end{pmatrix}
\\
\begin{pmatrix}1 & 2 & 3 \\ 2 & 3 & 1\end{pmatrix}
\end{matrix}
\to
\begin{matrix}
\begin{pmatrix}1 & 2 & 3 \\ 1 & 3 & 2\end{pmatrix}
\\
\begin{pmatrix}1 & 3 & 2 \\ 2 & 1 & 3\end{pmatrix}
\end{matrix}
\to
\begin{pmatrix}1 & 2 & 3 \\ 2 & 1 & 3\end{pmatrix}
$$
$$
\pi\circ\rho\colon
\begin{matrix}
\begin{pmatrix}1 & 2 & 3 \\ 2 & 3 & 1\end{pmatrix}
\\
\begin{pmatrix}1 & 2 & 3 \\ 1 & 3 & 2\end{pmatrix}
\end{matrix}
\to
\begin{matrix}
\begin{pmatrix}1 & 2 & 3 \\ 2 & 3 & 1\end{pmatrix}
\\
\begin{pmatrix}2 & 3 & 1 \\ 3 & 2 & 1\end{pmatrix}
\end{matrix}
\to
\begin{pmatrix}1 & 2 & 3 \\ 3 & 2 & 1\end{pmatrix}
$$
Мы получили, что $\rho\circ\pi=\begin{pmatrix}1 & 2 & 3 \\ 2 & 1 &
  3\end{pmatrix}$,
$\pi\circ\rho=\begin{pmatrix}1 & 2 & 3 \\ 3 & 2 & 1\end{pmatrix}$, и
видно, что это разные перестановки: $\rho\circ\pi\neq\pi\circ\rho$.

% 27.02.2013

Сейчас мы покажем, что любая перестановка представляется в виде
произведения перестановок простейшего вида. Интуитивно ясно, что
простейшей [нетождественной] перестановкой является та, которая лишь
меняется местами два элемента, а остальные оставляет на своих местах.

\begin{definition}
Пусть $1\leq i,j\leq n$ и $i\neq j$. Обозначим через $\tau_{ij}$
следующую перестановку:
$$
\begin{cases}
\tau_{ij}(i)&=j,\\
\tau_{ij}(j)&=i,\\
\tau_{ij}(k)&=k\text{ при $k\neq i,j$}.
\end{cases}
$$
Ее табличная запись выглядит так:
$$
\begin{pmatrix}
\dots & i & \dots & j & \dots\\
\dots & j & \dots & i & \dots.
\end{pmatrix}
$$
(подразумевается, что все столбики с многоточиями отвечают {\it
  неподвижным} элементам).
Такая перестановка называется \dfn{транспозицией}\index{транспозиция}. Перестановка вида
$\tau_{i,i+1}$ (при $1\leq i\leq n-1$) называется \dfn{элементарной
  транспозицией}\index{транспозиция!элементарная}.
\end{definition}
Очевидно, что любая транспозиция $\tau_{ij}$ совпадает с $\tau_{ji}$ и
является обратной к себе самой: $\tau_{ij}=\tau_{ji}$,
$\tau_{ij}\circ\tau_{ij}=\id$.
Посмотрим, что происходит при умножении перестановки на транспозицию:
сравним табличные записи перестановок $\pi$ и
$\pi\circ\tau_{ij}$. Нетрудно видеть, что они различаются только в
столбцах с номерами $i$ и $j$ (поскольку $\tau_{ij}$ совпадает с
тождественной в остальных точках). А именно,
$$
\pi=\begin{pmatrix}\dots & i & \dots & j & \dots\\
\dots & \pi(i) & \dots & \pi(j) & \dots\end{pmatrix},\quad
\pi\circ\tau_{ij}=\begin{pmatrix}\dots & i & \dots & j & \dots\\
\dots & \pi(j) & \dots & \pi(i) & \dots\end{pmatrix}.
$$
Иными словами, домножение на $\tau_{ij}$ справа соответствует
перестановке $i$-ой и $j$-ой позиций в нижней строке табличной записи
перестановки.

\begin{proposition}\label{prop:product_of_transpositions}
Любая перестановка является произведением транспозиций.
\end{proposition}
\begin{proof}
Пусть $\pi\in S_n$.
Начнем с тождественной перестановки $\id$ и покажем, что
последовательным домножением на транспозиции справа можно получить
перестановку $\pi$. Сначала добьемся того, чтобы на первом месте в
нижней строке табличной записи нашей перестановки стояло то, что
нужно~--- то есть, $\pi(1)$. Для этого нужно переставить местами
первый столбик с тем, в котором стоит $\pi(1)$ (Конечно, если
$\pi(1)=1$, ничего переставлять и не нужно). После этого поставим
на второе место в нижней строке $\pi(2)$: так как $\pi$ является
перестановкой, то $\pi(1)\neq\pi(2)$, поэтому где-то справа от первого
столбца есть столбец с $\pi(2)$. Поменяем его со вторым. И так далее:
на $k$-шаге мы добиваемся того, что первые $k$ чисел в нижней строке
нашей перестановки выглядели так: $\pi(1),\pi(2),\dots,\pi(k)$. В
конце концов (дойдя до $k=n$) мы получим перестановку $\pi$ путем
домножения $\id$ на транспозиции, что и требовалось.
\end{proof}
\begin{proposition}\label{prop_odd_number_of_elementary_transpositions}
Любая транспозиция является произведением нечетного числа элементарных
транспозиций.
\end{proposition}
\begin{proof}
Неформально задача выглядит так: нам разрешено менять местами любые
два соседних элемента в строке, а хочется поменять местами два
элемента, стоящих далеко друг от друга. Как этого добиться? Очень
просто: сначала «продвинуть» последовательно левый из этих элементов
направо до второго, поменять их там местами, а потом второй элемент
«отогнать» обратно на место левого. При этом наши элементы поменяются
местами, а все остальные элементы останутся на своих местах: любой
элемент между нашими мы затронем ровно два раза: на пути «туда» и на
пути «обратно»; сначала он сдвинется на шаг влево, а потом~--- на шаг
вправо. Ну, а любой элемент, стоящий не между нашими, и подавно
останется на своем месте. Аккуратный подсчет показывает, что мы
совершили нечетное число операций.

Формально же это рассуждение выражается в виде формулы
$$
\tau_{ij}=\tau_{i,i+1}\circ\tau_{i+1,i+2}\circ\dots
\circ\tau_{j-2,j-1}\circ\tau_{j-1,j}\circ\tau_{j-2,j-1}\circ\dots
\tau_{i+1,i+2}\circ\tau_{i,i+1}
$$
(здесь мы считаем, что $i<j$).
Это равенство несложно проверить напрямую, и оно представляет
транспозицию $\tau_{ij}$ в виде произведения $2(j-i)-1$ элементарных
транспозиций.
\end{proof}

\begin{definition}
Пусть $\pi\in S_n$. Говорят, что пара индексов $(i,j)$ образует
\dfn{инверсию}\index{инверсия} для перестановки $\pi$, если $i<j$ и
$\pi(i)>\pi(j)$. Количество пар индексов от $1$ до $n$, образующих
инверсию для $\pi$, называется \dfn{числом инверсий}\index{число
  инверсий перестановки} перестановки
$\pi$ и обозначается через $\inv(\pi)$.
\end{definition}
Неформально говоря, число инверсий измеряет «отклонение» перестановки
от тождественной: если $\pi=\id$, то для $i<j$ всегда выполнено
$\pi(i)=i<j=\pi(j)$, поэтому $\inv(\id)=0$. Число инверсий~--- это
количество пар элементов, стоящих в «неправильном» порядке.
Важнейшей характеристикой перестановки является {\it четность} ее
числа инверсий, которая называется {\it знаком}:
\begin{definition}\label{def:permutation_sign}
Пусть $\pi\in S_n$. Число $(-1)^{\inv(\pi)}$ называется
\dfn{знаком}\index{знак перестановки}
перестановки $\pi$ и обозначается через $\sgn(\pi)$. Иными словами,
$\sgn(\pi)=1$, если $\inv(\pi)$ четно, и $\sgn(\pi)=-1$, если
$\inv(\pi)$ нечетно. Перестановка называется \dfn{четной}\index{четная
  перестановка}, если
$\sgn(\pi)=1$, и \dfn{нечетной}\index{нечетная перестановка}, если $\sgn(\pi)=-1$.
\end{definition}
\begin{example}
Единственный элемент в $S_1$ является четной перестановкой.
Одна из двух перестановок в $S_2$ (тождественная) является четной, а
другая~--- нечетной. Среди шести перестановок в $S_3$ имеется три
четных и три нечетных: четными являются $\id$,
$\begin{pmatrix}1&2&3\\2&3&1\end{pmatrix}$ и
$\begin{pmatrix}1&2&3\\3&1&2\end{pmatrix}$, а нечетными~---
транспозиции $\tau_{12}$, $\tau_{13}$ и $\tau_{23}$.
\end{example}
Оказывается, если перестановка представлена в виде произведения
транспозиций, то четность числа этих транспозиций всегда совпадает с
четностью перестановки (хотя понятно, что у перестановки может быть
много различных представлений в виде произведения транспозиций).
Для доказательства этого нам необходимо посмотреть на то, что
происходит со знаком при домножении перестановки на
транспозицию.
\begin{proposition}\label{prop_transposition_changes_sign}
Пусть $\pi\in S_n$, $\tau_{ij}\in S_n$~--- транспозиция. Тогда
$\sgn(\pi)=-\sgn(\pi\circ\tau_{ij})$.
\end{proposition}
\begin{proof}
Посмотрим, как меняется число инверсий перестановки при домножении на
{\it элементарную транспозицию}. Сравним перестановки
$$
\pi=\begin{pmatrix}\dots&i&i+1&\dots\\
\dots&\pi(i)&\pi(i+1)&\dots\end{pmatrix}\text{ и }
\pi\circ\tau_{i,i+1}=\begin{pmatrix}\dots&i&i+1&\dots\\
\dots&\pi(i+1)&\pi(i)&\dots\end{pmatrix}.
$$
Заметим, что вне столбцов с номерами $i$ и $i+1$ эти перестановки
совпадают, поэтому число инверсий для индексов вне множества
$\{i,i+1\}$, у них одинаковое. Далее, если для некоторого
$j\notin\{i,i+1\}$ индексы $i$ и $j$ образуют
инверсию для $\pi$ (например, мы имели $j<i$ и $\pi(j)>\pi(i)$), то
$i+1$ и $j$ образуют инверсию для $\pi\circ\tau_{i,i+1}$,
(поскольку
$(\pi\circ\tau_{i,i+1})(i+1)=\pi(i)<\pi(j)=(\pi\circ\tau_{i,i+1})(j)$
и $j<i+1$), и наоборот. Аналогично, если $i+1$ и $j$ образуют
инверсию для $\pi$, то $i$ и $j$ образуют инверсию для
$\pi\circ\tau_{i,i+1}$, и наоборот. Поэтому среди всех пар индексов,
кроме пары $(i,j)$, количество инверсий у $\pi$ и
$\pi\circ\tau_{i,i+q}$ одинаковое. Но если $(i,i+1)$ является
инверсией для $\pi$, то $(i,i+1)$ не является инверсией для
$\pi\circ\tau_{i,i+1}$, поскольку значения $\pi$ и
$\pi\circ\tau_{i,i+1}$ на $i$ и $i+1$ поменялись местами. Обратно,
если пара $(i,i+1)$ не была инверсией для $\pi$, она станет инверсией
для $\pi\circ\tau_{i,i+1}$. Значит, число инверсий
$\pi\circ\tau_{i,i+1}$ отличается от числа инверсий $\tau_{i,i+1}$
ровно на единицу: $\inv(\pi\circ\tau_{i,i+1})=\inv(\pi)\pm 1$. Поэтому
эти числа имеют разную четность.

Это означает, что при домножении на элементарную транспозицию
перестановка меняет знак. По
предложению~\ref{prop_odd_number_of_elementary_transpositions} любую
транспозицию можно записать как произведение нечетного числа
элементарных, поэтому при домножении на любую транспозицию
перестановка меняет знак нечетное число раз~--- то есть, меняет знак.
\end{proof}

\begin{corollary}\label{cor_sign_and_number_of_transpositions}
Пусть $\pi=\tau_1\circ\dots\circ\tau_s$, где $\tau_1,\dots,\tau_s$~---
транспозиции. Тогда $\sgn(\pi)=(-1)^s$.
\end{corollary}
\begin{proof}
Запишем $\pi=\id\circ\tau_1\circ\dots\circ\tau_s$ и посмотрим на это
произведение так: мы начали с тождественной перестановки и $s$ раз
домножили на транспозиции справа. Тождественная перестановка является
четной, и при каждом домножении знак меняется на противоположный,
поэтому итоговый знак равен $(-1)^s$.
\end{proof}

\begin{corollary}\label{cor_odd_and_even}
При $n\geq 2$ в группе $S_n$ поровну (по $n!/2$) четных и нечетных перестановок.
\end{corollary}
\begin{proof}
Рассмотрим отображение $f\colon S_n\to S_n$, $\pi\mapsto
\pi\circ\tau_{12}$. Нетрудно видеть, что это биекция (обратным к этому
отображению является оно само: $(f\circ
f)(\pi)=f(f(\pi))=(\pi\circ\tau_{12})\circ\tau_{12}=\pi$, поэтому
$f\circ f=\id_{S_n}$). При этом по
предложению~\ref{prop_transposition_changes_sign} $f$ переводит четные
перестановки в нечетные, а нечетные~--- в четные. Поэтому $f$
устанавливает биекцию между подмножеством четных перестановок и
подмножеством нечетных перестановок в $S_n$. Всего перестановок $n!$,
поэтому и четных, и нечетных по $n!/2$.
\end{proof}

Теперь несложно показать, что знак ведет себя мультипликативно:

\begin{theorem}\label{thm:permutation_sign_product}
Пусть $\pi,\rho\in S_n$; тогда
$\sgn(\pi\circ\rho)=\sgn(\pi)\cdot\sgn(\rho)$.
\end{theorem}
\begin{proof}
Представим $\pi$ и $\rho$ в виде произведения транспозиций:
$\pi=\sigma_1\circ\dots\circ\sigma_s$,
$\rho=\tau_1\circ\dots\circ\tau_t$. По
следствию~\ref{cor_sign_and_number_of_transpositions} имеем
$\sgn(\pi)=(-1)^s$ и $\sgn(\rho)=(-1)^t$. При этом
$\pi\circ\rho=\sigma_1\circ\dots\circ\sigma_s\circ\tau_1\circ\dots\circ\tau_t$
есть произведение $s+t$ транспозиций, поэтому $\sgn(\pi\circ\rho)=(-1)^{s+t}=(-1)^s\cdot(-1)^t=\sgn(\pi)\cdot\sgn(\rho)$.
\end{proof}

\begin{corollary}\label{cor:permutation_sign_inverse}
Пусть $\pi\in S_n$; тогда $\sgn(\pi^{-1})=\sgn(\pi)$.
\end{corollary}
\begin{proof}
Заметим, что $\pi\circ\pi^{-1}=\id$, поэтому
$\sgn(\pi)\cdot\sgn(\pi^{-1})=\sgn(\id)=1$.
\end{proof}

\subsection{Определитель}\label{ssect:det}
\literature{[F], гл. IV, \S~2, пп. 1, 3, 4; [K1], гл. 3, \S~1;  [vdW], гл. 4, \S~25.}

Теперь все готово, чтобы ввести интересный инвариант квадратной
матрицы.
\begin{definition}
Пусть $A=(a_{ij})\in M(n,k)$~--- квадратная матрица над полем $k$. Ее
\dfn{определителем}\index{определитель} (или \dfn{детерминантом}\index{детерминант}) называется следующий
элемент поля $k$:
$$
\det(A)=\sum_{\pi\in S_n}\sgn(\pi)\cdot a_{1,\pi(1)}\cdot
a_{2,\pi(2)}\cdot\dots\cdot a_{n,\pi(n)}=\sum_{\pi\in S_n}\sgn(\pi)\prod_{i=1}^na_{i,\pi(i)}.
$$
Мы будем также использовать обозначение $|A|=\det(A)$.
\end{definition}

\begin{examples}
\begin{itemize}
\item Определитель матрицы $1\times 1$: в этом случае в сумме из
  определения
  $\det(A)$ всего одно слагаемое, и знак тождественной перестановки
  равен $1$, поэтому
  $\det(\begin{pmatrix}a_{11}\end{pmatrix})=a_{11}$.
\item Определитель матрицы $2\times 2$: $S_2=\{\id,\tau_{12}\}$,
  причем $\sgn(\id)=1$, $\sgn(\tau_{12})=-1$, поэтому
  $$\left|\begin{matrix}a_{11}&a_{12}\\a_{21}&a_{22}\end{matrix}\right|=a_{11}a_{22}-a_{12}a_{21}.$$
\item Определитель матрицы $3\times 3$:
$$
\left|\begin{matrix}a_{11}&a_{12}&a_{13}\\a_{21}&a_{22}&a_{23}\\
a_{31}&a_{32}&a_{33}\end{matrix}\right| = a_{11}a_{22}a_{33} +
a_{12}a_{23}a_{31} + a_{13}a_{21}a_{32} - a_{12}a_{21}a_{33} -
a_{13}a_{31}a_{22} - a_{11}a_{23}a_{32}.
$$
\end{itemize}
\end{examples}

Выясним простейшие свойства определителя.
\begin{proposition}
Пусть $A\in M(n,k)$; тогда $\det(A^T)=\det(A)$.
\end{proposition}
\begin{proof}
Посмотрим на формулу для определителя матрицы $A=(a_{ij})$. В
слагаемом, соответствующем
перестановке $\pi$, перемножаются элементы вида $a_{i,\pi(i)}$, то
есть, элементы вида $a_{ij}$ для $j=\pi(i)$. Заметим, что $j=\pi(i)$
тогда и только тогда, когда $\pi^{-1}(j)=i$. Иными словами, в
рассматриваемом слагаемом перемножаются элементы вида
$a_{\pi^{-1}(j),j}$ для всех $j=1,\dots,n$.
Поэтому мы можем записать
$$
\det(A)=\sum_{\pi\in S_n}\sgn(\pi)\prod_{i=1}^n a_{i,\pi(i)}
=\sum_{\pi\in S_n}\sgn(\pi)\prod_{j=1}^n a_{\pi^{-1}(j),j}
=\sum_{\pi\in S_n}\sgn(\pi)\prod_{j=1}^n a_{\pi(j),j}.
$$
В последнем равенстве мы воспользовались тем фактом, что если $\pi$
пробегает всю группу $S_n$, то и $\pi^{-1}$ пробегает всю $S_n$; кроме
того, $\sgn(\pi)=\sgn(\pi^{-1})$, поэтому можно заменить суммирование
по всем $\pi$ на суммирование по всем $\pi^{-1}$.
Но последнее выражение совпадает с формулой для $\det(A^T)$: элемент
матрицы $A$, стоящий в позиции $(\pi(j),j)$~--- это в точности элемент
матрицы $A^T$, стоящий в позиции $(j,\pi(j))$.
\end{proof}

Следующие свойства определителя касаются его зависимость от различных
операций над строками.
Пусть $A=(a_{ij})\in M(n,k)$~--- квадратная
матрица, $(a'_{i1},a'_{i2},\dots,a'_{in})$~--- некоторая
строка. Рассмотрим матрицу $A'$, полученную заменой $i$-ой строки
матрицы $A$ на строку $(a'_{i1},a'_{i2},\dots,a'_{in})$, и матрицу
$A''$, полученную заменой $i$-ой строки матрицы $A$ на строку
$(a_{i1}+a'_{i1}, a_{i2}+a'_{i2},\dots, a_{in}+a'_{in})$. Схематично
мы будем изображать это так:
$$
\begin{array}{c}
A=\begin{pmatrix}\vdots & \vdots & \ddots & \vdots\\
a_{i1} & a_{i2} & \dots & a_{in}\\
\vdots & \vdots & \ddots & \vdots\end{pmatrix},
A'=\begin{pmatrix}\vdots & \vdots & \ddots & \vdots\\
a'_{i1} & a'_{i2} & \dots & a'_{in}\\
\vdots & \vdots & \ddots & \vdots\end{pmatrix},\\
A''=\begin{pmatrix}\vdots & \vdots & \ddots & \vdots\\
a_{i1}+a'_{i1} & a_{i2}+a'_{i2} & \dots & a_{in}+a'_{in}\\
\vdots & \vdots & \ddots & \vdots\end{pmatrix}.
\end{array}
$$
Здесь многоточия символизируют тот факт, что все три матрицы $A, A',
A''$ совпадают за пределами $i$-й строки.
Оказывается, что определитель ведет себя
\dfn{аддитивно}\index{аддитивность!определителя} по отношению
к строкам матрицы: $\det(A'')=\det(A)+\det(A')$. Иными словами, если
представить какую-нибудь строку матрицы в виде суммы двух строк, то
определитель исходной матрицы будет равен сумме определителей матриц,
в которых эта строка заменена на строки-слагаемые.
Нам будет удобнее записывать это следующим образом: обозначим
$u=(a_{i1},a_{i2},\dots,a_{in})$,
$v=(a'_{i1},a'_{i2},\dots,a'_{in})$ (таким образом, $u,v\in
M(1,n,k)$~--- две строки длины $n$). Тогда
$$
\left|\begin{matrix}\vdots \\ u+v \\ \vdots\end{matrix}\right|=
\left|\begin{matrix}\vdots \\ u \\ \vdots\end{matrix}\right|+
\left|\begin{matrix}\vdots \\ v \\ \vdots\end{matrix}\right|
$$
 (здесь $u+v$ обозначает [покомпонентную] сумму строк $u$ и $v$, и
снова подразумевается, что в остальных позициях эти три матрицы
совпадают).

Посмотрим на формулу для определителя матрицы
$A''$:
$$
\det(A'')=\sum_{\pi\in S_n} \sgn(\pi) a_{1,\pi(1)} \dots
(a_{i,\pi(i)}+a'_{i,\pi(i)}) \dots a_{n,\pi(n)}
$$
(здесь мы воспользовались тем, что в $i$-ой строке матрицы $A''$ стоят
суммы соответствующих элементов $i$-х строк матриц $A$ и $A'$). Каждое
слагаемое выписанной суммы в силу дистрибутивности распадается на два
слагаемых, в одно из которых входит $a_{i,\pi(i)}$, а в другое~---
$a'_{i,\pi(i)}$:
\begin{align*}
\det(A'')&=\sum_{\pi\in S_n}\left(\sgn(\pi) a_{1,\pi(1)} \dots
a_{i,\pi(i)} \dots a_{n,\pi(n)} + \sgn(\pi)a_{1,\pi(1)} \dots
a'_{i,\pi(i)}) \dots a_{n,\pi(n)}\right)\\
 &= \sum_{\pi\in S_n}\left(\sgn(\pi) a_{1,\pi(1)} \dots
a_{i,\pi(i)} \dots a_{n,\pi(n)}\right)
 + \sum_{\pi\in S_n}\left(\sgn(\pi) a_{1,\pi(1)} \dots
a'_{i,\pi(i)} \dots a_{n,\pi(n)}\right).
\end{align*}
Первое из полученных слагаемых в точности равно $\det(A)$, а второе
равно $\det(A')$, поэтому $\det(A'')=\det(A)+\det(A')$, что и
требовалось.

Кроме того, если все элементы некоторой строки умножить на $\lambda\in
k$, то и определитель матрицы умножится на $\lambda$. Точнее,
рассмотрим матрицу $A=(a_{ij})\in M(n,k)$ и заменим в ней $i$-ю строку
$(a_{i1},a_{i2},\dots,a_{in})$ на строку $(\lambda a_{i1}, \lambda
a_{i2}, \dots, \lambda a_{in})$. Обозначим полученную матрицу через
$A'$. Тогда $\det(A')=\lambda\det(A)$. Действительно, определитель
матрицы $A'$ равен
$$
\det(A') = \sum_{\pi\in S_n}\left(\sgn(\pi) a_{1,\pi(1)} \dots
(\lambda a_{i,\pi(i)}) \dots a_{n,\pi(n)}\right).
$$
В каждом слагаемом полученной суммы присутствует множитель
$\lambda$. После вынесения его за скобки получаем
$$
\det(A') = \lambda\left(\sum_{\pi\in S_n}\sgn(\pi) a_{1,\pi(1)} \dots
a_{i,\pi(i)} \dots a_{n,\pi(n)}\right) = \lambda\det(A).
$$

% 6.03.2013

Доказанные два свойства в совокупности называют \dfn{линейностью}\index{линейность!определителя}
определителя по строкам. Кроме того, определитель обладает
\dfn{кососимметричностью}\index{кососимметричность определителя} по
строкам:
если две строки матрицы $A=(a_{ij})\in M(n,k)$ совпадают, то ее
определитель равен
нулю. То есть, если найдутся такие индексы $i\neq j$, что
$a_{il}=a_{jl}$ для всех $l=1,\dots,n$, то $\det(A)=0$. Конечно,
кососимметричность имеет смысл только при $n\geq 2$.

Для доказательства кососимметричности заметим сначала, что отображение
$f\colon S_n\to S_n$, $\pi\mapsto f\circ\tau_{ij}$ является биекцией и
меняет четность перестановок. Мы уже видели такое отображение в
доказательстве следствия~\ref{cor_odd_and_even} для частного случая
$\{i,j\}=\{1,2\}$. Значит, ограничив должным образом отображение $f$,
мы получаем биекцию между множеством всех четных и множеством всех
нечетных перестановок. Обозначим множество всех четных перестановок из
$S_n$ через $A_n$, и для краткости будем писать $\tau$ вместо
$\tau_{ij}$. Получаем биекцию $A_n\to S_n\setminus A_n$,
$\pi\mapsto f\circ\tau$, которую мы обозначим также через $f$.
Теперь вернемся к нашей матрице $A=(a_{ij})\in M(n,k)$, в которой
$i$-ая строка совпадает с $j$-ой. Запишем определитель матрицы $A$:
$$
\det(A)=\sum_{\pi\in S_n}\sgn(\pi)a_{1,\pi(1)}\dots a_{i,\pi(i)}\dots
a_{j,\pi(j)}\dots a_{n,\pi(n)}.
$$
Теперь при помощи биекции $f$ разобьем все слагаемые на пары, поставив
в одну пару слагаемые, соответствующие перестановкам $\pi\in A_n$ и
$f(\pi)=\pi\circ\tau\in S_n\setminus A_n$:
\begin{align*}
\det(A)=\sum_{\pi\in A_n} & \big(\sgn(\pi)a_{1,\pi(1)}\dots
  a_{i,\pi(i)}\dots a_{n,\pi(n)} +\\
  & \sgn(\pi\circ\tau)a_{1,(\pi\circ\tau)(1)}\dots
  a_{i,(\pi\circ\tau)(i)}\dots a_{j,(\pi\circ\tau)(j)}\dots
  a_{n,(\pi\circ\tau)(n)} \big).\\
\end{align*}
Осталось заметить, что $\sgn(\pi\circ\tau)=-\sgn(\pi)$,
$a_{i,(\pi\circ\tau)(i)}=a_{i,\pi(j)}=a_{j,\pi(j)}$,
$a_{j,(\pi\circ\tau)(j)}=a_{j,\pi(i)}=a_{i,\pi(i)}$ и
$a_{k,(\pi\circ\tau)(k)}=a_{k,\pi(k)}$ для всех $k\neq i,j$. Поэтому
сумма двух слагаемых в каждой паре равна $0$, а с ней и весь
$\det(A)$.

Стало быть, нами доказана следующая теорема.
\begin{theorem}
Определитель линейно и кососимметрично зависит от строк матрицы. Иными
словами,
$$
\left|\begin{matrix}\vdots \\ u+v \\ \vdots\end{matrix}\right|=
\left|\begin{matrix}\vdots \\ u \\ \vdots\end{matrix}\right|+
\left|\begin{matrix}\vdots \\ v \\ \vdots\end{matrix}\right|,\quad
\left|\begin{matrix}\vdots \\ \lambda u \\ \vdots\end{matrix}\right|=
\lambda\left|\begin{matrix}\vdots \\ u \\ \vdots\end{matrix}\right|,\quad
\left|\begin{matrix}\vdots \\ u \\ \vdots \\ u \\
    \vdots\end{matrix}\right| = 0.
$$
Кроме того, определитель линейно и кососимметрично зависит от столбцов
матрицы.
\end{theorem}
\begin{proof}
Утверждение для строк доказано выше; утверждение для столбцов
получается транспонированием матрицы.
\end{proof}

Теперь нетрудно понять, как меняется определитель при элементарных
преобразованиях строк и столбцов.
\begin{theorem}\label{thm_det_under_elementary}
Определитель матрицы не меняется при элементарном преобразовании
(строк или столбцов) первого типа, меняет знак при элементарном
преобразовании второго типа, и умножается на $\eps$ при элементарном
преобразовании $D_i(\eps)$ третьего типа. На матричном языке:
$$
|T_{ij}(\lambda)A|=|AT_{ij}(\lambda)|=|A|,\quad
|S_{ij}A|=|AS_{ij}|=-|A|,\quad
|D_i(\eps)A|=|AD_i(\eps)|=\eps|A|.
$$
\end{theorem}
\begin{proof}
Как всегда, мы проведем доказательство только для элементарных
преобразований строк. Рассмотрим элементарное преобразование первого
типа и воспользуемся линейностью:
$$
\left|\begin{matrix}\vdots \\ u+\lambda v \\ \vdots \\ v \\
    \vdots\end{matrix}\right|=
\left|\begin{matrix}\vdots \\ u \\ \vdots \\ v \\
    \vdots\end{matrix}\right|+
\lambda\left|\begin{matrix}\vdots \\ v \\ \vdots \\ v \\
    \vdots\end{matrix}\right|.
$$
Заметим, что первое слагаемое результата~--- это определитель исходной
матрицы, а второе слагаемое равно нулю в силу кососимметричности.

Посмотрим на элементарные преобразования второго типа. Для любых строк
$u,v$ длины $n$ выполнено
$$
0 = \left|\begin{matrix}\vdots \\ u+v \\ \vdots \\ u+v \\
    \vdots \end{matrix}\right| =
\left|\begin{matrix}\vdots \\ u \\ \vdots \\ u \\
    \vdots\end{matrix}\right|+
\left|\begin{matrix}\vdots \\ u \\ \vdots \\ v \\
    \vdots\end{matrix}\right|+
\left|\begin{matrix}\vdots \\ v \\ \vdots \\ u \\
    \vdots\end{matrix}\right|+
\left|\begin{matrix}\vdots \\ v \\ \vdots \\ v \\
    \vdots\end{matrix}\right| = 
\left|\begin{matrix}\vdots \\ u \\ \vdots \\ v \\
    \vdots\end{matrix}\right|+
\left|\begin{matrix}\vdots \\ v \\ \vdots \\ u \\
    \vdots\end{matrix}\right|,
$$
откуда 
$$
\left|\begin{matrix}\vdots \\ u \\ \vdots \\ v \\
    \vdots\end{matrix}\right| = -
\left|\begin{matrix}\vdots \\ v \\ \vdots \\ u \\
    \vdots\end{matrix}\right|.
$$
Это и означает, что элементарное преобразование второго типа меняет
знак определителя. Наконец, для элементарных преобразований третьего
типа утверждение теоремы напрямую следует из линейности определителя.
\end{proof}

\subsection{Дальнейшие свойства определителя}
\literature{[K1], гл. 3, \S~2, п. 2; [vdW], гл. 4, \S~19.}

\begin{theorem}[Определитель блочной верхнетреугольной матрицы]\label{thm_det_block_ut}
Пусть матрица $A\in M(n,k)$ имеет вид
$A=\begin{pmatrix}B & X\\0 & C\end{pmatrix}$, где
$B\in M(m,k)$, $C\in M(n-m,k)$, $X\in M(m,n-m,k)$. Тогда $|A|=|B|\cdot
|C|$.
\end{theorem}
\begin{proof}
Мы знаем, что $\det(A)=\sum_{\pi\in S_n}\sgn(\pi)a_{1,\pi(1)}\dots a_{m,\pi(m)}
a_{m+1,\pi(m+1)} \dots a_{n,\pi(n)}$.
По предположению, $a_{ij}=0$, если $i>m$ и $j\leq m$. Поэтому
некоторые слагаемые в этой сумме равны $0$. Покажем, что ненулевое
слагаемое не может содержать и множителей из блока $X$, то есть, не
может включать в себя множитель $a_{ij}$ для $i\leq m$, $j>m$.
Действительно, посмотрим на некоторое ненулевое слагаемое
$a_{1,\pi(1)}\dots a_{m,\pi(m)} a_{m+1,\pi(m+1)}\dots a_{n,\pi(n)}$,
соответствующее перестановке $\pi$.
Среди чисел $\pi(1),\dots,\pi(n)$ должны встречаться по разу числа
$1,\dots,m$. Если некоторое число $j\leq m$ равно $\pi(i)$, то
обязательно должно быть $i\leq m$, поскольку, по предположению,
$a_{ij}=0$ при $i>m$ и $j\leq m$. Значит, все числа $1,\dots,m$
встречаются среди чисел $\pi(1),\dots,\pi(m)$. Но тех и других
поровну, значит, $\pi(i)\leq m$ для любого $i\leq m$. Стало быть,
$\pi(i)>m$ для любого $i>m$. Мы получили, что наше слагаемое содержит
лишь множители вида $a_{ij}$, где либо $i,j\leq m$, либо $i,j>m$. В
частности, матричных элементов из блока $X$ среди них не встречается.

Таким образом, на самом деле суммирование в $\det(A)$ производится по
тем перестановкам $\pi$, которые действуют <<отдельно>> на наборах
$1,\dots,m$ и $m+1,\dots,n$, не переставляя числа из разных
наборов. Поэтому каждая такая перестановка однозначно определяет две
перестановки: на числах $1,\dots,m$ и на числах
$m+1,\dots,n$. Обозначим первую из них через $\rho$, а вторую сдвинем
на $m$ влево (чтобы получить перестановку чисел $1,\dots,n-m$, то
есть, элемент из $S_{n-m}$) и обозначим через $\sigma$. По
перестановке $\pi$ мы построили пару перестановок $\rho\in S_m$,
$\sigma\in S_{n-m}$.

Посмотрим теперь на произведение $\det(B)\cdot\det(C)$. Это
$$
\left(\sum_{\rho\in S_m}\sgn(\rho)a_{1,\rho(1)}\dots a_{m,\rho(m)}\right)\cdot
\left(\sum_{\sigma\in S_{n-m}}\sgn(\sigma)a_{m+1,m+\sigma(1)}\dots a_{n,m+\sigma(n-m)}\right).
$$
При раскрытии скобок в этом произведении получим сумму слагаемых вида
$$\sgn(\rho)\sgn(\sigma)a_{1,\rho(1)}\dots
a_{m,\rho(m)}a_{m+1,m+\sigma(1)}\dots a_{n,m+\sigma(n-m)}$$ для всех пар
перестановок $\rho\in S_m$, $\sigma\in S_{n-m}$. По каждой такой паре
перестановок построим перестановку $\pi\in S_n$, подействовав
перестановкой $\rho$ на числах $1,\dots,m$ и перестановкой $\sigma$
(сдвинутой на $m$ вправо) на числах $m+1,\dots,n$.

Теперь видно, что в формулах для $\det(A)$ и $\det(B)\cdot\det(C)$
происходит суммирование по всем парам перестановок $(\rho,\sigma)\in
S_m\times S_{n-m}$ слагаемых одинакового вида. Осталось лишь проверить
совпадение знаков: в первой формуле мы видим $\sgn(\pi)$, а во
второй~--- произведение $\sgn(\rho)\cdot\sgn(\sigma)$. Но нетрудно
видеть, что число инверсий в перестановке $\pi$ равно сумме чисел
инверсий в соответствующих им перестановках $\rho$ и $\sigma$: нет
никаких инверсий между числами из набора $1,\dots,m$ и числами из
набора $m+1,\dots,n$.
\end{proof}

\begin{corollary}\label{cor_ut_det}
Определитель верхнетреугольной матрицы равен произведению ее
диагональных элементов:
$$
\left|
\begin{pmatrix}
a_1 & *   & *   & \dots & *\\
0   & a_2 & *   & \dots & *\\
0   & 0   & a_3 & \dots & *\\
\vdots & \vdots & \vdots & \ddots & \vdots\\
0 & 0 & 0 & \dots & a_n
\end{pmatrix}
\right| = a_1a_2\dots a_n.
$$
В частности, определитель единичной матрицы $E_n$ равен $1$.
\end{corollary}
\begin{proof}
Это несложно получить из предыдущей теоремы индукцией по размеру
матрицы. Можно и напрямую заметить, что в сумме из определения
$\det(A)$ для верхнетреугольной матрицы $A$ лишь одно слагаемое
отлично от нуля~--- то, которое отвечает тождественной перестановке.
\end{proof}

\begin{proposition}\label{prop_det_zero_row}
Если в матрице присутствует нулевой столбец или нулевая строка, то ее
определитель равен нулю.
\end{proposition}
\begin{proof}
Пусть $i$-ая строка матрицы $A$ равна нулю.
В каждое слагаемое из определения $\det(A)$ входит элемент вида
$a_{i,\pi(i)}$, равный нулю, поэтому каждое слагаемое равно
нулю. Доказательство для нулевого столбца получается
транспонированием.
\end{proof}

\begin{proposition}\label{prop_det_of_elementary}
Определители матриц элементарных преобразований:
$|T_{ij}(\lambda)|=1$, $|S_{ij}|=-1$, $|D_i(\eps)|=\eps$.
Определитель окаймленной единичной матрицы размера $n\times n$:
$\left|\begin{matrix}E_r & 0 \\ 0 & 0\end{matrix}\right|=\begin{cases}0,
  &\text{если }r<n;\\1, &\text{если }r=n\end{cases}$.
\end{proposition}
\begin{proof}
Матрица элементарных преобразований приводится к единичной одним
элементарным преобразованием, и мы знаем, как при этом меняется ее
определитель, поэтому первая часть~--- тривиальное вычисление.
Окаймленная единичная матрица является верхнетреугольной, поэтому
вторая часть сразу следует из следствия~\ref{cor_ut_det}.
\end{proof}

\begin{theorem}[Мультипликативность определителя]\label{thm:determinant_product}
Определитель произведения матриц равен произведению их
определителей:
$$\det(AB)=\det(A)\det(B)\quad\text{ для любых }A,B\in M(n,k).$$
\end{theorem}
\begin{proof}
Заметим, что для любой матрицы $C\in M(n,k)$ выполнены равенства
\begin{align*}
\det(T_{ij}(\lambda)C) &= \det(T_{ij}(\lambda))\det(C),\\
\det(S_{ij}C) &= \det(S_{ij})\det(C),\\
\det(D_i(\eps)C) &= \det(D_i(\eps))\det(C),\\
\det(\begin{pmatrix}E_r & 0\\0 & 0\end{pmatrix}C) &=
\det(\begin{pmatrix}E_r & 0\\0 & 0\end{pmatrix})\det(C).
\end{align*}
Действительно, первые три равенства следуют из
теоремы~\ref{thm_det_under_elementary} и
предложения~\ref{prop_det_of_elementary}. При $r<n$ матрица
$\begin{pmatrix}E_r & 0\\0 & 0\end{pmatrix}C$ имеет нулевую строку,
поэтому ее определитель равен нулю
(предложение~\ref{prop_det_zero_row}), как и произведение
определителей сомножителей (в силу
предложения~\ref{prop_det_of_elementary}. При $r=n$ указанная матрица
является единичной, поэтому результат следует из
следствия~\ref{cor_ut_det}.

По следствию~\ref{cor_pdq} мы можем записать
$$A=P_t\dots P_1\begin{pmatrix}E_r & 0\\0 & 0\end{pmatrix}Q_1\dots
Q_s,$$
где $P_1,\dots,P_t,Q_1,\dots,Q_s$~--- матрицы элементарных
преобразований. Тогда
$$\det(AB)=\det(P_t\dots P_1\begin{pmatrix}E_r & 0\\0 &
  0\end{pmatrix}Q_1\dots Q_sB).$$ Применяя замечание из предыдущего
абзаца несколько раз, получаем, что
$$\det(AB)=\det(P_t)\dots\det(P_1)\det(\begin{pmatrix}E_r & 0\\0 &
  0\end{pmatrix})\det(Q_1)\dots\det(Q_s)\det(B).$$
С другой стороны,
$$\det(A)=\det(P_t\dots P_1\begin{pmatrix}E_r & 0\\0 &
  0\end{pmatrix}Q_1\dots Q_s),$$ и, снова применяя замечание выше,
получаем
$$\det(A)=\det(P_t)\dots\det(P_1)\det(\begin{pmatrix}E_r & 0\\0 &
  0\end{pmatrix})\det(Q_1)\dots\det(Q_s).$$ Сопоставляя полученные
равенства, получаем, что $\det(AB)=\det(A)\det(B)$.
\end{proof}

\subsection{Разложение определителя по строке}
\literature{[F], гл. IV, \S~2, п. 5; [K1], гл. 3, \S~2.}

Посмотрим на матрицу $A\in M(n,k)$. Вычеркнем из нее строку с номером
$i$ и столбец с номером $j$ для некоторых $1\leq i,j\leq
n$. Обозначим полученную матрицу через $M_{ij}\in M(n-1,k)$.
Определитель матрицы $M_{ij}$ (а иногда сама эта матрица) называется
\dfn{(дополнительным) минором}\index{минор!дополнительный}.

Теперь посмотрим на строку с номером $i$ исходной матрицы $A$ и
воспользуемся линейностью определителя:
$$
|A| = 
\left|\begin{matrix}\vdots & \vdots & \ddots & \vdots\\
a_{i1} & a_{i2} & \dots & a_{in}\\
\vdots & \vdots & \ddots & \vdots\end{matrix}\right|
= 
\left|\begin{matrix}\vdots & \vdots & \ddots & \vdots\\
a_{i1} & 0 & \dots & 0\\
\vdots & \vdots & \ddots & \vdots\end{matrix}\right| + 
\left|\begin{matrix}\vdots & \vdots & \ddots & \vdots\\
0 & a_{i2} & \dots & 0\\
\vdots & \vdots & \ddots & \vdots\end{matrix}\right| + \dots +
\left|\begin{matrix}\vdots & \vdots & \ddots & \vdots\\
0 & 0 & \dots & a_{in}\\
\vdots & \vdots & \ddots & \vdots\end{matrix}\right|.
$$
Посчитаем отдельно определитель каждого слагаемого в правой части.
Слагаемое с номером $j$ имеет вид
$$
\left|\begin{matrix}\ddots & \vdots & \vdots & \vdots & \ddots\\
\dots & 0 & a_{ij} & 0 & \dots\\
\ddots & \vdots & \vdots & \vdots & \ddots\end{matrix}\right|:
$$
все элементы в $i$-ой строчке равны нулю, кроме $a_{ij}$.
Теперь аккуратно переставим строчки и столбцы так, чтобы элемент
$a_{ij}$ оказался в левом верхнем углу нашей матрицы; для этого
нужно сдвинуть по циклу строки с номерами от $1$ до $i$ и столбцы с
номерами от $1$ до $j$. То есть, сначала поменяем местами строки $i$ и
$i-1$, затем строки $i-1$ и $i-2$, и так далее, пока не поменяем
строки $1$ и $2$. Нетрудно видеть, что мы совершили ровно $i-1$
элементарное преобразоване второго типа. При этом определитель нашей
матрицы умножился на $(-1)^{i-1}$. После этого сделаем то же самое со
столбцами, и определитель умножится на $(-1)^{j-1}$. В итоге он
умножится на $(-1)^{i-1+j-1}=(-1)^{i+j-2}=(-1)^{i+j}$. После таких
операций наша матрица будет иметь следующий блочный вид:
$$
\begin{pmatrix}a_{ij} & 0\\
* & M_{ij}
\end{pmatrix}.
$$
По теореме~\ref{thm_det_block_ut} (напомним, что определитель не
меняется при транспонировании) ее определитель равен произведению
$a_{ij}$ на дополнительный минор $|M_{ij}|$. Значит, $j$-е слагаемое в
разложении $\det(A)$, с которого мы начали, равно
$(-1)^{i+j}a_{ij}|M_{ij}|$.

Произведение $(-1)^{i+j}|M_{ij}|$ называется
\dfn{алгебраическим дополнением}\index{алгебраическое дополнение}
элемента $a_{ij}$ и обозначается
через $\widetilde{A}_{ij}$.
Мы получили \dfn{разложение определителя по строке:}\index{разложение
  определителя!по строке}
$\det(A)=a_{i1}\widetilde{A}_{i1} + a_{i2}\widetilde{A}_{i2} + \dots +
a_{in}\widetilde{A}_{in}$.
Транспонируя полученный результат, мы получаем
\dfn{разложение определителя по столбцу:}\index{разложение
  определителя!по столбцу}
$\det(A)=a_{1i}\widetilde{A}_{1i} + a_{2i}\widetilde{A}_{2i} + \dots +
a_{ni}\widetilde{A}_{ni}$.

Сформулируем чуть более общий результат.

\begin{theorem}[Соотношения ортогональности]\index{соотношения
    ортогональности}
Пусть $A\in M(n,k)$ и $1\leq i\leq n$. Тогда
$$
a_{i1}\widetilde{A}_{j1} + a_{i2}\widetilde{A}_{j2} + \dots +
a_{in}\widetilde{A}_{jn} =
\begin{cases}
\det(A),&\text{если }i=j;\\
0,&\text{если }i\neq j.
\end{cases}.
$$
\end{theorem}
\begin{proof}
При $i=j$ это в точности разложение определителя по строке. Если же
$i\neq j$, рассмотрим матрицу $A'$, которая совпадает с матрицей $A$
везде, кроме строчки с номером $j$, а в ее строчке с номером $j$ стоит
строчка с номером $i$ матрицы $A$. Таким образом, строки матрицы $A'$
с номерами $i$ и $j$ совпадают, поэтому ее определитель равен нулю. С
другой стороны, раскладывая этот определитель по строке с номером $j$,
мы получим в
точности сумму $a_{i1}\widetilde{A}_{j1} + a_{i2}\widetilde{A}_{j2} + \dots +
a_{in}\widetilde{A}_{jn}$, поскольку в строке с номером $j$ стоят
элементы $a_{i1},a_{i2},\dots,a_{in}$, а их дополнения совпадают с
дополнениями элементов $j$-ой строки матрицы $A$, поскольку
алгебраические дополнения элементов $j$-ой строки не зависят от того,
что именно стоит в $j$-ой строке.
\end{proof}
Конечно, несложно сформулировать аналогичные соотношения, исходя из
разложения определителя по столбцу.

Эту теорему можно записать в более компактной форме. Для этого
рассмотрим матрицу
$\adj(A)$, в которой на позиции $(i,j)$ стоит алгебраическое
дополнение $\widetilde{A}_{ji}$ (обратите внимание на то, что индексы
поменялись местами). Она называется
\dfn{присоединенной}\index{матрица!присоединенная}
(или \dfn{взаимной}\index{матрица!взаимная}) к матрице
$A$. Соотношения ортогональности (для
строк и столбцов) тогда
переписываются следующим образом.
\begin{corollary}\label{cor_orthogonality_relations}
Для матрицы $A\in M(n,k)$ выполнено
$$
A\cdot\adj(A)=\det(A)\cdot E = \adj(A)\cdot A
$$
\end{corollary}
Теперь нетрудно доказать критерий обратимости квадратной матрицы.
\begin{corollary}\label{cor_matrix_invertible_det}
Матрица $A\in M(n,k)$ обратима тогда и только тогда, когда
$\det(A)\neq 0$; в этом случае $A^{-1}=(\det(A))^{-1}\adj(A)$.
\end{corollary}
\begin{proof}
Если $A$ обратима, то найдется $A^{-1}$ такая, что $A\cdot A^{-1}=E$;
тогда $$\det(A)\det(A^{-1})=\det(A\cdot A^{-1})=\det(E)=1$$ в силу
мультипликативности определителя.
Обратно, если $\det(A)\neq 0$, то, разделив соотношение
ортогональности на скаляр $\det(A)$, получаем, что
$$A\cdot(\det(A))^{-1}\adj(A)=E=(\det(A))^{-1}\adj(A)\cdot A,$$
что и требовалось.
\end{proof}

% 13.03.2013

В частности, для матрицы $2\times 2$ это следствие означает,
что
$$
\begin{pmatrix}a & b\\c & d\end{pmatrix}
= \frac{1}{ad-bc}\begin{pmatrix}d & -b\\-c & a\end{pmatrix}
$$
(если, конечно, $ad-bc\neq 0$).

 Применим теперь полученные результаты к решению системы линейных
уравнений с невырожденной матрицей.
Рассмотрим систему линейных уравнений $AX=B$ с квадратной матрицей
$A=(a_{ij})\in M(n,k)$, где
$X=\begin{pmatrix}x_1\\x_2\\\vdots\\x_n\end{pmatrix}$~--- столбец
неизвестных,
$B=\begin{pmatrix}b_1\\b_2\\\vdots\\b_n\end{pmatrix}\in M(n,1,k)$~---
столбец правой части. Напомним, что {\it решить систему}~--- значит,
найти все столбцы $X\in M(n,1,k)$, для которых выполнено $AX=B$.
Если матрица $A$ невырождена, то есть, существует обратная матрица
$A^{-1}$, после домножения обеих частей уравнения на $A^{-1}$ получаем
$A^{-1}AX=A^{-1}B$, что равносильно равенству $X=A^{-1}B$. Таким
образом, система уравнений с невырожденной квадратной матрицей всегда
имеет единственное решение.

Более того, для нахождения этого решения нетрудно написать чуть более
явные формулы, называемые \dfn{формулами Крамера}\index{формулы
  Крамера}.
Действительно,
\begin{align*}
X = A^{-1}B = \frac{1}{\det(A)}\adj(A)B &= 
\frac{1}{\det(A)}
\begin{pmatrix}
\widetilde{A}_{11} & \widetilde{A}_{21} & \dots & \widetilde{A}_{n1}\\
\widetilde{A}_{12} & \widetilde{A}_{22} & \dots & \widetilde{A}_{n2}\\
\vdots & \vdots & \ddots & \vdots\\
\widetilde{A}_{1n} & \widetilde{A}_{2n} & \dots & \widetilde{A}_{nn}
\end{pmatrix}\cdot
\begin{pmatrix}
b_1 \\ b_2 \\ \vdots \\ b_n
\end{pmatrix}\\
&=
\frac{1}{\det(A)}
\begin{pmatrix}
b_1\widetilde{A}_{11} + b_2\widetilde{A}_{21} + \dots +
b_n\widetilde{A}_{n1}\\
b_1\widetilde{A}_{12} + b_2\widetilde{A}_{22} + \dots +
b_n\widetilde{A}_{n2}\\
\vdots\\
b_1\widetilde{A}_{1n} + b_2\widetilde{A}_{2n} + \dots +
b_n\widetilde{A}_{nn}
\end{pmatrix}.
\end{align*}
Итоговые выражения очень похожи на разложения определителя по строке.
И действительно, заменим в матрице $A$ столбец под номером $i$ на
столбец $B$. Обозначим полученную матрицу через~$A'_i$.
Посчитаем определитель этой матрицы, разложив его по $i$-ому столбцу:
для этого нужно перемножать элементы ее $i$-го столбца (то есть,
элементы столбца $B$) на их алгебраические дополнения, которые
совпадают с соответствующими алгебраическими дополнениями элементов
матрицы $A$. Мы получим в точности $b_1\widetilde{A}_{1i} +
b_2\widetilde{A}_{2i} + \dots + b_n\widetilde{A}_{ni}$~--- то, что
стоит в столбце $X$ на позиции $i$ (с точностью до множителя
$1/\det(A)$. Сформулируем полученный результат в виде теоремы.

\begin{theorem}[Формулы Крамера]
Пусть $A\in M(n,k)$~--- невырожденная матрица, $B\in M(n,1,k)$~---
некоторый столбец. Обозначим через $A'_i$ матрицу, полученную
подстановкой столбца $B$ вместо $i$-го столбца матрицы $A$.
Тогда решение $X=\begin{pmatrix}x_1\\x_2\\\vdots\\x_n\end{pmatrix}$
системы линейных уравнений $AX=B$ единственно и задается формулами
$$
x_i=\frac{\det(A'_i)}{\det(A)}.
$$
\end{theorem}

Посмотрим теперь на множество решений произвольной однородной системы
линейных уравнений $AX=0$ с матрицей $A\in M(m,n,k)$; здесь
$X=\begin{pmatrix}x_1\\x_2\\\vdots\\x_n\end{pmatrix}$~--- столбец
неизвестных, а в правой части стоит нулевая матрица $0\in M(m,1,k)$.

\begin{proposition}[Свойства решений однородной системы линейных
  уравнений]
Если $X, X'\in M(n,1,k)$~--- решения системы $AX=0$, то сумма
  $X+X'$ также является решением этой системы.
Если $X\in M(n,1,k)$~--- решение системы $AX=0$, $\lambda\in k$,
  то $\lambda X\in M(n,1,k)$ также является решением этой системы.
\end{proposition}
\begin{proof}
Если $AX=0$ и $AX'=0$, то $A(X+X')=AX+AX'=0+0=0$ и
$A(\lambda X)=\lambda(AX)=\lambda\cdot 0=0$.
\end{proof}

Теперь посмотрим на произвольную систему линейных уравнений $AX=B$
(мы сохраняем предыдущие обозначения; кроме того, $B\in M(m,1,k)$~---
некоторый столбец правой части).
\begin{proposition}[Свойства решений неоднородной системы линейных
  уравнений]\label{prop_structure_of_solutions_linear_system}
Пусть $X_0$~--- некоторое фиксированное решение системы $AX=B$
Тогда любое решение этой системы
имеет вид $X = X_0 + Y$, где $Y$~--- некоторое решение соответствующей
однородной системы $AX=0$. Обратно, для любого решения $Y$ однородной
системы $AX=0$ сумма $X = X_0+Y$ является решением системы $AX=B$.
\end{proposition}
\begin{proof}
Если $AX_0=B$ и $AY=0$, то $A(X_0+Y)=AX_0+AY=B+0=B$. Обратно, если
$AX_0=B$ и, кроме того, $AX=B$, то $A(X-X_0)=AX-AX_0=B-B=0$, поэтому
$X-X_0$ является решением соответствующей однородной системы.
\end{proof}

Поэтому поиск решений произвольной системы линейных уравнений $AX=B$
сводится к нахождению {\em частного решения} $X_0$ этой системы (если
оно вообще существует), и к
нахождению всех решений соответствующей однородной системы $AX=0$.
В главе~\ref{section_vector_spaces} мы построим общую теорию для
изучения свойств решений однородных систем, а в главе 7 сформулируем
в рамках этой теории и вопрос о существовании частного решения
неоднородной
системы.
