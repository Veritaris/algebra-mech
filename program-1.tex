\documentclass[12pt]{article}
\usepackage[T2A]{fontenc}
\usepackage[utf8]{inputenc}
\usepackage[russian]{babel}
\usepackage{mdwlist}
\usepackage{amsfonts}
\usepackage{amssymb}
\usepackage{amsmath}
\usepackage{amsthm}
\usepackage{paralist}
\usepackage{ccfonts,eulervm,euler}
\renewcommand{\bfdefault}{sbc}
%\usepackage{ulem}
\usepackage[margin=0.7in,bmargin=0.7in]{geometry}


\pagestyle{empty}

\DeclareMathOperator{\Hom}{Hom}

\newcommand\glava[1]{{\medskip\bf\hfill #1}}

\begin{document}

\begin{center}
{\Large Вопросы экзамена по алгебре}
\par\medskip
Группы 151, 153 (лектор Александр Лузгарев)
\par\medskip
Первый семестр, осень 2016
\end{center}

\glava{Глава 3. Комплексные числа}
\begin{compactenum}
\item Комплексные числа: определение, алгебраическая форма записи.
\item Комплексное сопряжение и модуль. Деление комплексных чисел.
\item Неравенство треугольника. Тригонометрическая форма записи
  комплексного числа.
\item Перемножение комплексных чисел в тригонометрической
  форме. Формула Муавра.
\item Корни $n$-ой степени из комплексного числа. Свойства корней из
  единицы.
\item Первообразные корни из единицы, их количество.
\item Экспоненциальная форма записи комплексного числа и логарифм.
\suspend{compactenum}
\glava{Глава 4. Кольцо многочленов}
\resume{compactenum}
\item Кольцо многочленов над кольцом.
\item Теорема о степени произведения многочленов
  над областью целостности и ее следствия.
\item Делимость в кольце многочленов. Теорема о делении с остатком.
\item Многочлен как функция. Лемма Безу.
\item Выделение линейных множителей и число различных корней
  многочлена над полем. Формальное и функциональное равенство
  многочленов.
\item Алгебраическая замкнутость. Разложение многочленов над полями
  комплексных и вещественных чисел.
\item Определение и свойства производной. Связь между корнями
  многочлена и его производной.
\item Характеристика поля. Поведение кратности корня при взятии
  производной над полем характеристики $0$.
\item Интерполяционная задача, единственность ее решения.
  Интерполяционные формулы Лагранжа и Ньютона.
\item Наибольший общий делитель многочленов: существование и линейное
  представление.
\item Алгорифм Эвклида для многочленов. Оценка на степень
  коэффициентов в линейном представлении НОД.
\item Неприводимые многочлены. Основная теорема арифметики в кольце
  многочленов.
\item Конструкция поля частных области целостности: эквивалентность
  дробей, введение операций и проверка аксиом поля.
\item Поле рациональных функций. Правильные дроби, их
  свойства. Выделение многочлена из дроби.
\item Простейшие дроби: две леммы о разложении знаменателя.
\item Теорема о представлении правильной дроби в виде суммы
  простейших.
\item Простейшие дроби над полями комплексных и вещественных
  чисел. Нахождение коэффициентов в случае простых корней.
\suspend{compactenum}
\glava{Глава 5. Вычислительная линейная алгебра}
\resume{compactenum}
\item Системы линейных уравнений и матрицы. Элементарные
  преобразования и связь с множеством решений.
\item Метод Гаусса решения систем линейных уравнений.
\item Матрицы, свойства сложения и умножения.
\item Транспонирование и его свойства. Матричные единицы.
\item Матрицы элементарных преобразований. Элементарные преобразования
  строк и столбцов как умножения на матрицы.
\item Приведение матрицы к окаймленному единичному виду элементарными
  преобразованиями. Связь с обратимостью.
\item Блочные матрицы и операции над ними.
\item Группа перестановок. Табличная запись перестановки.
\item Разложение перестановки в произведение [элементарных]
  транспозиций.
\item Число инверсий и знак перестановки. Изменение знака при
  домножении на транспозицию.
\item Знак как число транспозиций в разложении. Мультипликативность
  знака.
\suspend{compactenum}
\glava{Глава 6. Векторные пространства}
\resume{compactenum}
  \item Определение векторного пространства, простейшие свойства,
  примеры.
\item Подпространства: определение и примеры. Пересечение
  и сумма подпространств.
\item Прямая сумма подпространств. Критерии разложения пространства
  в прямую сумму подпространств.
\item Системы образующих и линейно независимые системы. Примеры.
\item Лемма о линейной зависимости. Следствие о добавлении вектора
к линейно независимой системе.
\item Количество элементов в линейно независимой и порождающей
системах.
\item Конечномерность подпространства конечномерного пространства.
\item Базис, определение и эквивалентная переформулировка.
Любая система образующих содержит базис.
\item Любая линейно независимая система содержится в базисе.
Дополнение пространства до прямой суммы.
\item Равномощность всех базисов. Размерность подпространства.
\item Теорема Грассмана о размерности суммы подпространств.
\item Если длина порождающей системы равна размерности, то это
базис; то же для линейно независимой системы.
Размерность прямой суммы подпространств. Критерий разложения
в прямую сумму, использующий размерность.
\end{compactenum}

\end{document}

