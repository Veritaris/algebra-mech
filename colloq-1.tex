\documentclass[12pt]{article}
\usepackage[T2A]{fontenc}
\usepackage[utf8]{inputenc}
\usepackage[russian]{babel}
\usepackage{mdwlist}
\usepackage{amsfonts}
\usepackage{amssymb}
\usepackage{amsmath}
\usepackage{amsthm}
\usepackage{paralist}
\usepackage{ccfonts,eulervm,euler}
\renewcommand{\bfdefault}{sbc}
%\usepackage{ulem}
\usepackage[margin=0.7in,bmargin=0.7in]{geometry}


\pagestyle{empty}

\DeclareMathOperator{\Hom}{Hom}

\newcommand\glava[1]{{\bf\hfill #1}}

\begin{document}

\begin{center}
{\Large Вопросы коллоквиума по алгебре}
\par\medskip
Группы 151, 153 (лектор Александр Лузгарев)
\par\medskip
Первый семестр, осень 2016
\end{center}

\glava{Наивная теория множеств}
\begin{compactenum}
\item Множества, подмножества, основные операции над множествами.
\item Отображения: образ, прообраз, инъекция, сюръекция, биекция.
\item Композиция отображений, ее ассоциативность, тождественное
  отображение.
\item Левая/правая обратимость и инъективность/сюръективность.
\item График отображения, бинарные отношения и отношения эквивалентности.
\item Теорема о разбиении на классы эквивалентности. Фактор-множество.
\item Метод математической индукции. Бинарные операции.
\item Нейтральные элементы и обратимость.
\item Теорема об обобщенной ассоциативности.
\suspend{compactenum}
\glava{Элементарная теория чисел}
\resume{compactenum}
\item Делимость: определения и простейшие свойства. Ассоциированность.
\item Теорема о делении с остатком.
\item Наибольший общий делитель; его существование и
единственность. Линейное представление НОД.
\item Алгорифм Эвклида.
\item Свойства НОД. Взаимная простота, свойства взаимно простых чисел.
\item Линейные диофантовы уравнения. Полное описание множества решений
уравнения с двумя неизвестными.
\item НОД нескольких чисел и критерий разрешимости линейного диофантова
уравнения с несколькими неизвестными.
\item Простые числа, их свойства.
\item Основная теорема арифметики.
\item Каноническое разложение. Приложения: НОД, число делителей.
\item Сравнения по модулю. Свойства.
\item Классы вычетов, действия над ними.
\item Определение кольца. Кольцо классов вычетов.
\item Нулевое кольцо. Делители нуля, области целостности, поля.
\item Критерий обратимости элемента кольца классов вычетов. Когда
  кольцо классов вычетов является полем?
\item Китайская теорема об остатках.
\item Теорема Вильсона.
\item Функция Эйлера. Переформулировка китайской теоремы об остатках в
терминах колец классов вычетов.
\item Мультипликативность функции Эйлера. Формула для функции Эйлера.
\item Теорема Эйлера и малая теорема Ферма.
\end{compactenum}

\end{document}

