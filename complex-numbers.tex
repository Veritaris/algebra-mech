\section{Комплексные числа}

\subsection{Определение комплексных чисел}

\literature{[F], гл. II, \S~1, пп. 1--5; [K1], гл. 5, \S~1, пп. 1--2.}

Комплексные числа представляют собой расширение поля вещественных
чисел, обладающее гораздо более приятными алгебраическими
свойствами. Наш подход к определению комплексных чисел
аксиоматический~--- мы сначала описываем некоторое множество с
операциями, которое оказывается полем, а потом показываем, что оно
содержит вещественные числа и задумываемся о мотивации.

\begin{definition}\label{def_complex}
Рассмотрим множество $\mb R\times\mb R$ пар вещественных чисел.
Введем на нем операции сложения и умножения:
\begin{align*}
&(a,b)+(c,d)=(a+c,b+d),\\
&(a,b)\cdot (c,d)=(ac-bd,ad+bc).
\end{align*}
\end{definition}

\begin{theorem}\label{complex_ring}
Множество с операциями, определенное в~\ref{def_complex}, является
ассоциативным коммутативным кольцом с единицей.
\end{theorem}
\begin{proof}
Необходимо проверить восемь аксиом из определения~\ref{def:ring}.
\begin{enumerate}
\item $((a,b)+(c,d))+(e,f)=(a+c,b+d)+(e,f)=((a+c)+e,(b+d)+f)$,
  $(a,b)+((c,d)+(e,f))=(a,b)+(c+e,d+f)=(a+(b+c),d+(e+f))$. Полученные
  выражения равны, поскольку сложение вещественных чисел ассоциативно.
\item Нейтральным элементом по сложению является пара
  $(0,0)$. Действительно, $(a,b)+(0,0)=(a+0,b+0)=(a,b)$, и по
  коммутативности сложения (аксиома 4) то же верно, если складывать в
  другом порядке.
\item Противоположным элементом к паре $(a,b)$ является пара
  $(-a,-b)$. Действительно, $(a,b)+(-a,-b)=(a+(-a),b+(-b))=(0,0)$.
\item $(a,b)+(c,d)=(a+c,b+d)=(c+a,d+b)=(c,a)+(d,b)$.
\item $((a,b)\cdot(c,d))\cdot(e,f)=(ac-bd,ad+bc)\cdot(e,f)
  =((ac-bd)e-(ad+bc)f,(ac-bd)f+(ad+bc)e)$. С другой стороны,
  $(a,b)\cdot((c,d)\cdot(e,f))=(a,b)\cdot(ce-df,cf+de)
  =(a(ce-df)-b(cf+de),a(cf+de)+b(ce-df))$. Раскрытие скобок
  показывает, что полученные выражения равны.
\item Нейтральным элементом по умножению является пара
  $(1,0)$. Действительно, $(a,b)\cdot(1,0)=(a\cdot-b\cdot 0,a\cdot
  0+b\cdot 1=(a,b)$, и этого достаточно в силу коммутативности
  умножения (аксиома 7).
\item $(a,b)\cdot (c,d)=(ac-bd,ad+bc)$ и $(c,d)\cdot
  (a,b)=(ca-db,cb+da)$.
\item $(a,b)\cdot ((c,d)+(e,f))=(a,b)\cdot
  (c+e,d+f)=(a(c+e)-b(d+f),a(d+f)-b(c+e))$. С другой стороны,
  $(a,b)\cdot (c,d) + (a,b)\cdot (e,f)=(ac-bd,ad+bc)+(ae-bf,af+be)
  =(ac-bd+ae-bf,ad+bc+af+be)$. Раскрытие скобок показывает, что
  полученные выражения равны; и этого достаточно в силу
  коммутативности умножения (аксиома 7).
\end{enumerate}
\end{proof}

\begin{definition}
Множество таких пар вещественных чисел с определенными
в~\ref{def_complex} операциями
обозначается через $\mb C$; его элементы называются \dfn{комплексными
  числами}\index{комплексное число}.
\end{definition}

\begin{remark}
Множество вещественных чисел можно считать
подмножеством множества комплексных чисел: число $a\in\mb R$ можно
рассматривать как комплексное число $(a,0)$. При этом введенные нами
операции на парах превращаются в обычные операции над комплексными
числами: действительно, $(a,0)+(b,0)=(a+b,0)$ и $(a,0)\cdot
(b,0)=(ab,0)$; единица $(1,0)$ и нуль $(0,0)$ в множестве комплексных
чисел являются вещественными числами $1$ и $0$. Заметим также, что
$a\cdot (c,d)=(a,0)\cdot (c,d)=(ac,ad)$.
\end{remark}

\begin{definition}
Пусть $z=(a,b)$~--- комплексное число; запишем
$z=(a,b)=(a,0)+(0,b)=a+b\cdot(0,1)$. Комплексное число $(0,1)$
обозначается через $i$ и называется \dfn{мнимой единицей}\index{мнимая
  единица}; основанием
этому служит тому, что $i^2=-1$. Запись
$z=a+bi$ называется \dfn{алгебраической формой записи комплексного
  числа}\index{комплексное число!алгебраическая форма записи},
вещественные числа $a$ и $b$~--- \dfn{вещественной
  частью}\index{вещественная часть} и
\dfn{мнимой частью}\index{мнимая часть} комплексного числа $z$
соответственно. Обозначения: $a=\Ree(z)$, $b=\Img(z)$.
\end{definition}

\begin{remark}
Теперь мы можем забыть про интерпретацию комплексного числа как пары
вещественных чисел и считать, что комплексное число~--- это выражение
вида $a+bi$ с вещественными $a,b$. При этом введенные нами
в~\ref{def_complex} операцию переписываются в алгебраической форме
следующим образом:
\begin{align*}
(a+bi)+(c+di)&=(a+c)+(b+d)i,\\
(a+bi)\cdot (c+di)&=(ac-bd)+(ad+bc)i.
\end{align*}
Иными словами, комплексные числа~--- это выражения вида $a+bi$,
которые складываются и перемножаются согласно обычным правилам
обращения с числами с учетом равенства $i^2=-1$.
\end{remark}

\subsection{Комплексное сопряжение и модуль}

\literature{[F], гл. II, \S~1, пп. 3--5, \S~2, пп. 1--4; [K1], гл. 5, \S~1, п. 3.}

\begin{definition}
Сопоставим комплексному числу $z=a+bi$ комплексное число
$\overline{z}=a-bi$. Полученное отображение $\mb C\to\mb C$ называется
\dfn{сопряжением}\index{сопряжение}, а число $\overline{z}$~--- \dfn{сопряженным} к
числу $z$.
\end{definition}

\begin{proposition}[Свойства сопряжения]
Для любых комплексных чисел $z,w\in\mb C$ выполняются следующие свойства:
\begin{enumerate}
\item $\overline{z+w}=\overline{z}+\overline{w}$;
\item $\overline{z\cdot w}=\overline{z}\cdot\overline{w}$;
\item $\overline{\overline{z}}=z$;
\item $z=\overline{z}$ тогда и только тогда, когда $z\in\mb R$;
\item $\overline{z}\cdot z=z\cdot\overline{z}$~--- неотрицательное
  вещественное число; оно равно нулю тогда и только тогда, когда
  $z=0$.
\end{enumerate}
\end{proposition}
\begin{proof}
Пусть $z=a+bi$, $w=c+di$.
\begin{enumerate}
\item $\ol{(a+bi)+(c+di)}=\ol{(a+c)+(b+d)i}=(a+c)-(b+d)i$,
  $\ol{a+bi}+\ol{c+di}=(a-bi)+(c-di)=(a+c)-(b+d)i$.
\item $\ol{(a+bi)(c+di)}=\ol{(ac-bd)+(ad+bc)i}=(ac-bd)-(ad+bc)i$,
  $\ol{a+bi}\cdot\ol{c+di}=(a-bi)(c-di)=(ac-bd)-(ad+bc)i$.
\item $\ol{\ol{z}}=\ol{a-bi}=a+bi$.
\item Если $z\in\mb R$, то $z=a+0i$ и $\ol{z}=a-0i=z$. Обратно, если
  $a+bi=a-bi$, то $b=-b$, откуда $b=0$ и $z=a\in\mb R$.
\item $z\cdot\ol{z}=(a+bi)(a-bi)=(a^2+b^2)+(-ab+ba)i=a^2+b^2\geq 0$, и
  $a^2+b^2=0$ тогда и только тогда, когда $a=b=0$, то есть, когда $z=0$.
\end{enumerate}
\end{proof}

\begin{definition}\label{dfn:absolute_value_complex}
Поскольку $z\cdot\overline{z}$~--- неотрицательное вещественное число,
из него можно извлечь (также неотрицательный) квадратный корень. Этот
корень называется \dfn{модулем}\index{модуль} комплексного числа $z$ и
обозначается
через $|z|$; таким образом, $z\cdot\overline{z}=|z|^2$. Если
$z=a+bi$~--- алгебраическая форма записи комплексного числа, то
$|z|=\sqrt{a^2+b^2}$.
\end{definition}

\begin{proposition}
Множество $\mb C$ комплексных чисел является полем.
\end{proposition}
\begin{proof}
После доказательства теоремы~\ref{complex_ring} остается проверить
наличие обратного по умножению у каждого ненулевого элемента. Пусть
$z\in\mb C$, $z\neq 0$. Тогда $|z|\neq 0$. Рассмотрим число
$z'=\frac{1}{|z|^2}\overline{z}$; легко видеть, что $z\cdot z'=z'\cdot
z=1$.
\end{proof}

\begin{remark}
Таким образом, в множестве комплексных чисел можно делить на ненулевые
элементы: $z/w=zw^{-1}$. Также определена операция возведения в целую
степень: если $n>0$, то $z^n=\underbrace{z\cdot\dots\cdot z}_{n}$,
если $n<0$ (и $z\neq 0$), то $z^n=\underbrace{z^{-1}\cdot\dots\cdot z^{-1}}_{-n}$,
если же $n=0$, то $z^0=1$. Нетрудно видеть, что эта операция
удовлетворяет обычным свойствам возведения в степень, типа
$z^{m+n}=z^m\cdot z^n$ и $(zw)^n=z^nw^n$.
\end{remark}

\begin{proposition}[Свойства модуля комплексных
  чисел]\label{prop_abs_properties}
\hspace{1em}
\begin{enumerate}
\item $|z|\cdot |w|=|z\cdot w|$;
\item если $w\neq 0$, то $|z|/|w|=|z/w|$.
\end{enumerate}
\end{proposition}
\begin{proof}
\begin{enumerate}
\item $|zw|=\sqrt{(zw)(\ol{zw})}
=\sqrt{z\cdot w\cdot\ol{z}\cdot\ol{w}}
=\sqrt{z\ol{z}\cdot w\ol{w}}=\sqrt{z\ol{z}}\sqrt{w\ol{w}}
=|z|\cdot|w|$.
\item Домножая на $|w|$, получаем, что нужно доказать $|z|=|z/w|\cdot
  |w|$, что следует из первой части.
\end{enumerate}
\end{proof}

\begin{remark}
Комплексные числа удобно изображать в виде точек плоскости. Рассмотрим
декартову систему координат на плоскости и сопоставим комплексному
числу $a+bi$ вектор с координатами $(a,b)$ (то есть, радиус-вектор
точки $(a,b)$). Сложение векторов (как и комплексных чисел) происходит
покоординатно, поэтому сумма векторов изображает сумму комплексных
чисел. Модуль комплексного числа в силу теоремы Пифагора равен длине
соответствующего вектора.
\end{remark}

\begin{proposition}[Неравенство треугольника]
Для любых комплексных чисел $z_1,z_2,z_3$ выполнено неравенство
$|z_1-z_2|+|z_2-z_3|\geq |z_3-z_1|$.
\end{proposition}
\begin{proof}
Обозначим $z=z_1-z_2$, $w=z_2-z_3$; нужно доказать, что $|z|+|w|\geq
|z+w|$. Заметим, что если $z+w=0$, неравенство очевидно.
Запишем $1=\frac{z}{z+w}+\frac{w}{z+w}$. Согласно правилу сложения
комплексных чисел,
$\Ree{1}=\Ree(\frac{z}{z+w})+\Ree(\frac{w}{z+w})$. Заметим, что
$\Ree(z)\leq |z|$ для любого комплексного числа $z$, поэтому
$\Ree{1}\leq |\frac{z}{z+w}|+|\frac{w}{z+w}|$. Домножая на
знаменатель, получаем необходимое неравенство.
\end{proof}

% 29.10.2014

\subsection{Тригонометрическая форма записи комплексного числа}

\literature{[F], гл. II, \S~2, пп. 1--6; [K1], гл. 5, \S~1, п. 4.}

\begin{definition}\label{dfn:trigonometric_form}
Пусть $z=a+bi\in\mb C$~--- ненулевое комплексное число. Обозначим
через $r=\sqrt{a^2+b^2}$ модуль числа $z$. Вещественные
числа $a/r$ и
$b/r$ таковы, что сумма их квадратов равна $1$. Поэтому
найдется такой угол $\ph$, что $a/r=\cos(\ph)$,
$b/r=\sin(\ph)$. Такой угол $\ph$ называется
\dfn{аргументом}\index{аргумент}
комплексного числа $z$. Заметим, что при этом
$$
z=|z|\cdot z/|z|=|z|(\frac{a}{r}+\frac{b}{r}i)=|z|(\cos(\ph)+i\sin(\ph)).
$$
Выражение $z=r(\cos(\ph)+i\sin(\ph))$ называется
\dfn{тригонометрической формой записи комплексного
  числа}\index{комплексное число!тригонометрическая
  форма}. Обозначение: $\ph=\arg(z)$. Как обычно,
можно считать, что аргумент (как и любой угол) записывается
вещественным числом с точностью до $2\pi k$, $k\in\mb Z$. Если выбрать
представитель в полуинтервале $[0,2\pi)$, получим то, что называется
\dfn{главным значением аргумента}\index{аргумент!главное значение}, оно обозначается через $\Arg(z)$
Обратно, по
модулю $r$ и аргументу $\ph$ комплексное число $z$ однозначно
восстанавливается: $z=a+bi$, $a=r\cos(\ph)$, $b=r\sin(\ph)$.
\end{definition}

{\small
Обратите внимание на необходимость осторожного обращения с понятием
угол. Аргумент комплексного числа $z$, вообще говоря, является не
вещественным числом, а углом (позднее мы придадим этому точный смысл:
$\arg(z)$~--- элемент {\it группы углов},
см.~пример~\ref{examples:group}(\ref{item:group_of_angles})). Этот угол можно
записать вещественным числом, но не однозначным образом: некоторые
вещественные числа записывают одинаковые углы. Например, числа $0$,
$2\pi$, $-2\pi$, $4\pi$, $-4\pi$,\dots ~--- это разные формы записи
одного и того же угла. При этом два вещественных числа $\alpha$ и
$\beta$ записывают один и тот же угол если и только если они
отличаются на целое кратное $2\pi$: $\alpha-\beta = 2\pi k$ для
некоторого $k\in\mb Z$. Это похоже на делимость целых чисел: $\alpha$
и $\beta$ задают один угол, если их разность <<делится>> на
$2\pi$. Это наводит на мысль, что углы~--- это классы эквивалентности
по описанному отношению <<сравнимости по модулю $2\pi$>>.
}

\begin{proposition}[Единственность тригонометрической формы записи]\label{prop_trig_unique}
Пусть $r,r'$~--- положительные вещественные числа, $\ph,\ph'$~---
углы, $z=r(\cos(\ph)+i\sin(\ph))$, $z'=r'(\cos(\ph')+i\sin(\ph'))$
Равенство комплексных чисел
$z=z'$ выполнено тогда и
только тогда, когда $r=r'$ и $\ph=\ph'$.
\end{proposition}
\begin{proof}
Модуль комплексного числа $z$ равен
\begin{align*}
\sqrt{(r\cos(\ph))^2+(r\sin(\ph))^2}&=\sqrt{(r^2((\cos(\ph))^2+(\sin(\ph))^2))}\\
&=r;
\end{align*}
аналогично, модуль комплексного числа $z'$ равен $r'$. Если $z=z'$, то
$r=r'$, откуда $z/r=z'/r'$. Значит,
$\cos(\ph)+i\sin(\ph)=\cos(\ph')+i\sin(\ph')$, откуда
$\cos(\ph)=\cos(\ph')$ и $\sin(\ph)=\sin(\ph')$. Но если у двух углов
совпадают синусы и совпадают косинусы, то они равны. Поэтому и
$\ph=\ph'$.
Обратно, если $r=r'$ и $\ph=\ph'$, то очевидно, что $z=z'$.
\end{proof}

\begin{remark}
Таким образом, $z$ можно задавать не парой вещественных чисел, а парой
$(|z|,\arg(z))$, состоящей из положительного вещественного числа и
угла. Единственное исключение~--- случай $z=0$: у нуля модуль равен
нулю, а аргумент вообще не определен. Чем полезно такое задание? В
алгебраической форме записи комплексные числа легко складывать:
вещественные части складываются и мнимые части
складываются. Оказывается, в тригонометрической форме записи
комплексные числа легко перемножать.
\end{remark}

\begin{theorem}\label{thm_complex_mult}
При перемножении комплексных чисел их модули перемножаются, а
аргументы складываются. Иными словами, если $z,w\in\mb C^*$, то
$|zw|=|z|\cdot |w|$ и $\arg(zw)=\arg(z)+\arg(w)$.
\end{theorem}
\begin{proof}
Первое утверждение было доказано в
предложении~\ref{prop_abs_properties}. Обозначим $\ph=\arg(z)$,
$\psi=\arg(w)$. Заметим, что
\begin{align*}
zw&=|z|(\cos(\ph)+i\sin(\ph))|w|(\cos(\psi)+i\sin(\psi))\\
&=|z|\cdot |w|(\cos(\ph)\cos(\psi)-\sin(\ph)\sin(\psi)+i(\cos(\ph)\sin(\psi)+\sin(\ph)\cos(\ph)))\\
&=|z|\cdot |w|(\cos(\ph+\psi)+i\sin(\ph+\psi)).
\end{align*}
С другой стороны, $zw=|zw|\cdot (\cos(\arg(zw))+i\sin(\arg(zw)))$.
По предложению~\ref{prop_trig_unique} из этого следует, что
$|zw|=|z|\cdot |w|$ (что мы знали и раньше) и
$\arg(zw)=\ph+\psi=\arg(z)+\arg(w)$, что и требовалось.
\end{proof}

\begin{corollary}\label{cor_complex_inverse}
Для любого ненулевого комплексного числа $z=r(\cos(\ph)+i\sin(\ph))$ имеем
$z^{-1}=r^{-1}(\cos(-\ph)+i\sin(-\ph))$.
\end{corollary}

\begin{corollary}
При делении комплексных чисел их модули делятся, а аргументы вычитаются.
\end{corollary}

\begin{corollary}[Формула де Муавра]\label{thm_de_moivre}
Для любого ненулевого комплексного числа $z=r(\cos(\ph)+i\sin(\ph))$
и любого целого $n$ имеет место равенство $z^n=r^n(\cos(n\ph)+i\sin(n\ph))$.
\end{corollary}
\begin{proof}
Для $n=0$ равенство очевидно; для $n>0$ следует из
теоремы~\ref{thm_complex_mult} по индукции, а случай отрицательного
$n$ сводится к случаю положительного при помощи равенства
$z^n=(z^{-1})^{-n}$ и следствия~\ref{cor_complex_inverse}.
\end{proof}

\subsection{Корни из комплексных чисел}

\literature{[F], гл. II, \S~3, пп. 1--2; [K1], гл. 5, \S~1, п. 4.}

Пусть $n$~--- положительное натуральное число, $w\in\mb C$. Посмотрим
на решения уравнения $z^n=w$. Во-первых, заметим, что если $w=0$, то
и $z=0$ (иначе из равенства $z^n=0$ делением на $z^n$ получаем
$1=0$). Пусть теперь $w\neq 0$. Запишем $w$ и $z$ в тригонометрической
форме: $w=r(\cos(\ph)+i\sin(\ph))$,
$z=|z|\cdot(\cos(\arg(z))+i\sin(\arg(z)))$.
По формуле де Муавра (\ref{thm_de_moivre})
$z^n=|z|^n\cdot(\cos(n\arg(z))+i\sin(n\arg(z)))$. Приравнивая $z^n$ к
$w$ и пользуясь единственностью тригонометрической записи
(\ref{prop_trig_unique}), получаем, что $|z|^n=r$ и
$n\arg(z)=\ph$. Отсюда следует, что $|z|=r^{1/n}$. Кроме того,
равенство углов $n\arg(z)=\ph$ означает равенство $n\psi=\ph+2\pi k$,
где $\psi$~--- некоторый числовой представитель угла $\arg(z)$, а
$k$~--- целое число.
Значит, $\psi=(\ph+2\pi k)/n$.

\begin{theorem}\label{thm_roots_of_complex_number}
Пусть $w=r(\cos(\ph)+i\sin(\ph))\in\mb C^*$, $n$~--- положительное натуральное
число. Существует ровно $n$ комплексных чисел $z$ таких, что $z^n=w$;
можно записать их так:
$$
z=r^{1/n}\left(\cos\left(\frac{\ph+2\pi k}{n}\right) +
  i\sin\left(\frac{\ph+2\pi k}{n}\right)\right),
$$
где $k=0,1,\dots,n-1$.
\end{theorem}
\begin{proof}
Выше мы проверили, что решения уравнения $z^n=w$ имеют вид
$$
z_k=r^{1/n}\left(\cos\left(\frac{\ph+2\pi k}{n}\right) +
  i\sin\left(\frac{\ph+2\pi k}{n}\right)\right).
$$
Осталось разобраться с их количеством и устранить неоднозначность:
дело в том, что при различных целых $k$ эта формула часто дает
одинаковые значения $z$. А именно, $z_k=z_l$ тогда и только тогда,
когда углы $(\ph+2\pi k)/n$ и $(\ph+2\pi l)/n$ совпадают. А это
происходит тогда, когда их числовые значения отличаются на целое
кратное $2\pi$: $(\ph+2\pi k)/n=(\ph+2\pi l)/n+2\pi t$, откуда
$\ph+2\pi k=\ph+2\pi l+2\pi tn$ и $k-l=tn$, то есть, $k\equiv
l\pmod{n}$. Значит различных значений $z$ столько же, сколько классов
вычетов по модулю $n$, и можно выбрать $z_k$, соответствующие
различным представителям $k$ этих классов вычетов
(см.~\ref{rem_cong_representatives}), например, $k=0,1,\dots,n-1$.
\end{proof}

\subsection{Корни из единицы}

\literature{[F], гл. II, \S~4, пп. 1--4.}

Пусть $n$~--- положительное натуральное число. Посмотрим на решения
уравнения $z^n=1$ в комплексных числах.

\begin{definition}
Пусть $n\in\mb N$, $n\geq 1$. Комплексное число $z\in\mb C$ называется
\dfn{корнем $n$-ой степени из $1$}\index{корень!степени $n$}, если $z^n=1$. Множество всех корней
степени $n$ из $1$ обозначается через $\mu_n$.
\end{definition}

\begin{proposition}[Свойства корней $n$-ой степени из 1]
Для каждого натурального $n\geq 1$ существуют ровно $n$ корней степени $n$
из $1$; это числа
$\eps_0^{(n)},\eps_1^{(n)},\dots,\eps_{n-1}^{(n)}$, где
$$
\eps_k^{(n)}=\cos(\frac{2\pi k}{n})+i\sin(\frac{2\pi k}{n}).
$$
При этом произведение двух корней степени $n$ из $1$ является корнем
степени $n$ из $1$; обратный к корню степени $n$ из $1$ является
корнем степени $n$ из $1$.
\end{proposition}
\begin{proof}
Формула для $\eps_k^{(n)}$ немедленно следует из
теоремы~\ref{thm_roots_of_complex_number} (с учетом того, что $|1|=1$
и $\arg(1)=0$.
Если $z,w\in\mu_n$, то $z^n=1$,
$w^n=1$, откуда $(zw)^n=z^n\cdot w^n=1$, поэтому и $zw\in\mu_n$. Кроме
того, $(z^{-1})^n=(z^n)^{-1}=1$, поэтому и $z^{-1}\in\mu_n$.
\end{proof}

\begin{remark}[Геометрическая интерпретация корней из единицы]\label{rem:roots_of_unity_geometry}
Из формулы для $\eps_k^{(n)}$ видно, что модули всех корней степени
$n$ из $1$ равны единице, а аргументы равны
$0,2\pi/n,4\pi/n,\dots,2(n-1)\pi/n$, то есть, образуют арифметическую
прогрессию с разностью $2\pi/n$. Значит, на комплексной плоскости
точки $\eps_k^{(n)}$ лежат на окружности с центром в $0$ и радиусом 1,
и углы $\angle AOB$ для двух соседних точек $A$, $B$, равны
$2\pi/n$. Из этого следует, что точки $\eps_k^{(n)}$ лежат в вершинах
правильного $n$-угольника с центром в $0$. Кроме того, так как
$\eps_0^{(n)}=1$, число $1$ является одной из вершин этого $n$-угольника.
\end{remark}

\begin{remark}
Вернемся к уравнению $z^n=w$ для комплексного числа $w\neq 0$. Пусть
$z_0$~--- некоторое решение этого уравнения; тогда $z_0^n=w$ и,
разделив первоначальное уравнение на это равенство, получаем
$z^n/z_0^n=w/w=1$, откуда $(z/z_0)^n=1$, то есть, $z/z_0$ является
корнем степени $n$ из $1$. Поэтому $z/z_0=\eps_k^{(n)}$ для некоторого
$k$, и $z=z_0\eps_k^{(n)}$. Таким образом, любое решение уравнения
$z^n=w$ отличается от некоторого фиксированного решения $z_0$
домножением на корень степени $n$ из $1$.
\end{remark}

\begin{definition}
Корень $n$-ой степени из $1$ называется
\dfn{первообразным}\index{корень!первообразный}, если он
не является корнем из $1$ никакой меньшей, чем $n$, степени. Иными
словами, $z$ называется первообразным корнем степени $n$ из $1$, если
$z^n=1$ и $z^m\neq 1$ при $0<m<n$.
\end{definition}

\begin{remark}
Заметим, что $\eps_1^{(n)}=\cos(2\pi/n)+i\sin(2\pi/n)$ является
первообразным корнем степени $n$ из $1$. Действительно, если
$(\cos(2\pi/n)+i\sin(2\pi/n))^m=1$ для некоторого $0<m<n$, то
по формуле Муавра $\cos(2\pi m/n)+i\sin(2\pi m/n)=1$, откуда $2\pi
m/n=2\pi k$ для некоторого целого $k$. Получаем $m=kn$, то есть, $m$
делится на $n$, что невозможно.
\end{remark}

\begin{proposition}
Пусть $\eps$~--- корень степени $n$ из $1$. Равносильны:
\begin{enumerate}
\item $\eps$~--- первообразный корень;
\item все числа $1=\eps^0, \eps^1, \eps^2,\dots,\eps^{n-1}$ различны.
\end{enumerate}
\end{proposition}
\begin{proof}
$(2)\Leftrightarrow (1)$: если $\eps^m=1$ для некоторого $0<m<n$, то
среди указанных чисел есть совпадающие.
$(1)\Leftrightarrow (2)$: если $\eps^k=\eps^m$ для некоторых $k,m$, то
можно считать, что $k>m$; тогда $\eps^k/\eps^m=\eps^{k-m}=1$. Из
определения первообразного корня следует, что $k=m$.
\end{proof}

% 05.11.2014

\begin{proposition}\label{prop_primitive_root_criteria}
Пусть $n\geq 1$~--- натуральное число, $0\geq k\geq n-1$.
Корень $\eps_k^{(n)}$ степени $n$ из $1$ является первообразным тогда
и только тогда, когда $\gcd(k,n)=1$.
\end{proposition}
\begin{proof}
Обозначим $\eps=\eps_1^{(n)}$. Нетрудно видеть, что $\eps_k^{(n)}=\eps^k$.
Если $\gcd(k,n)=d>1$, то
$(\eps_k^{(n)})^{n/d}=(\eps^k)^{n/d}=\eps^{kn/d}=(\eps^n)^{k/d}=1^{k/d}=1$
(здесь важно, что $k/d$~--- целое число). Это значит, что
$\eps_k^{(n)}$ является корнем степени $n/d$ из $1$, и, поскольку $n/d<n$, не
является первообразным корнем степени $n$ из $1$.

Обратно, если $\gcd(k,n)=1$, покажем, что $\eps_k^{(n)}=\eps^k$~---
первообразный корень степени $n$ из $1$.
Действительно, предположим,
что $(\eps^k)^m=\eps^{km}=1$, где $0<m<n$. Но
$\eps^{km}=(\cos(2\pi/n)+i\sin(2\pi/n))^{km}= (\cos(2\pi
km/n)+i\sin(2\pi km/n))=1$, откуда $2\pi km/n=2\pi t$ для некоторого
целого $t$. Это означает, что $km=nt$, то есть, $n\divides km$. Но
$k$ и $n$ взаимно просты; по свойству~\ref{coprime_prop3} взаимной
простоты (\ref{prop_properties_of_coprime}) теперь
$n\divides m$~--- противоречие с предположением $0<m<n$.
\end{proof}

\begin{corollary}
Количество первообразных корней степени $n$ из $1$ равно $\ph(n)$.
\end{corollary}
\begin{proof}
Следует из предложения~\ref{prop_primitive_root_criteria} и
определения функции Эйлера (\ref{def_euler_function}).
\end{proof}

\subsection{Экспоненциальная форма записи комплексного числа}

\literature{[F], гл. II, \S~5, пп. 1--3.}

Мы видели, что аргумент комплексного числа ведет себя подобно
логарифму: аргумент произведения равен сумме аргументов. Это
оправдывает следующее определение.
\begin{definition}
Пусть $z=a+bi$~--- комплексное число. Положим
$e^z=e^a(\cos(b)+i\sin(b))$.
\end{definition}

Заметим, что основное свойство экспоненты выполняется при таком
определении.
\begin{proposition}
$e^{z_1+z_2}=e^{z_1}\cdot e^{z_2}$.
\end{proposition}
\begin{proof}
Пусть $z_1=a_1+b_1i$, $z_2=a_2+b_2i$, тогда
$z_1+z_2=(a_1+a_2)+(b_1+b_2)i$ и
\begin{align*}
e^{z_1}\cdot e^{z_2} &=
e^{a_1}(\cos(b_1)+i\sin(b_1)e^{a_2}(\cos(b_2)+i\sin(b_2))\\
&=e^{a_1+a_2}(\cos(b_1+b_2)+i\sin(b_1+b_2)\\
&=e^{z_1+z_2}.
\end{align*}
\end{proof}

При этом $e^{i\ph}=\cos(\ph)+i\sin(\ph)$; в частности, $e^{i\pi}=-1$.
Теперь для любого ненулевого комплексного числа
$z=r(\cos(\ph)+i\sin(\ph))$ можно записать
$z=re^{i\ph}=e^{\logn(r)+i\ph}$. Эта запись называется
\dfn{экспоненциальной формой записи комплексного
  числа}\index{комплексное число!экспоненциальная форма}.

Попытаемся теперь определить обратную функцию~--- логарифм. Основное
свойство логарифма должно сохраниться: логарифм должен быть обратной
функцией к экспоненте. Заметим, что экспонента переводит сумму в
произведение: $e^{a+b} = e^a\cdot e^b$. Поэтому логарифм должен
переводить произведение в сумму: $\ln(ab) = \ln(a) + \ln(b)$.
Таким образом, если определить логарифм вообще возможно,
то для комплексного числа
$z=r(\cos(\ph)+i\sin(\ph)) = r\cdot e^{i\ph}$ должно
выполняться $\logn(z)=\logn(r)+\logn(e^{i\ph})=\logn(r)+i\ph$.
Проблема состоит в том, что аргумент $\ph$ комплексного числа $z$
определен не вполне однозначно, а с точностью до прибавления целого
кратного числа $2\pi$. Поэтому и логарифм должен быть определен не
однозначно, а с точностью до целого кратного числа $2\pi i$.
Часто через $\Logn(z)$ обозначают все множество значений, то есть,
$\Logn(r(\cos(\ph)+i\sin(\ph)))=\{\logn(r)+i\ph+2\pi i k\mid k\in\mb Z\}$.
Под записью $\logn(z)$ мы будем понимать {\it какое-нибудь} значение
логарифма, то есть, какой-то элемент множества $\Logn(z)$. При этом из
основного свойства экспоненты немедленно следует основное свойство
логарифма: $\logn(z_1z_2)=\logn(z_1)+\logn(z_2)$. Понимать это равенство,
конечно, следует с точностью до слагаемого вида $2\pi ik$; например,
$\logn(1)=0$ и $\logn(-1)=\pi i$, но в то же время
$\logn(1)=\logn((-1)\cdot(-1))=\logn(-1)+\logn(-1)
=\pi i+\pi i = 2\pi i$.
