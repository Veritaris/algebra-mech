\section{Полилинейная алгебра}

\subsection{Полилинейные отображения}

\literature{[KM], ч. 2, \S~2, п. 1; ч. 4, \S~1, пп. 1--2.}

Пусть $k$~--- поле, $V_1, \dots, V_m, U$~--- векторные пространства
над $k$. Отображение
$f\colon V_1\times\dots\times V_m\to U$ называется
\dfn{полилинейным}\index{полилинейное отображение}, если оно линейно
по каждому аргументу при фиксированных значениях остальных. Иными
словами, $f$ \dfn{аддитивно}\index{аддитивное отображение} по каждому
аргументу:
$$
f(\dots,v'_i+v''_i,\dots) =
f(\dots,v'_i,\dots) + f(\dots,v''_i,\dots).
$$
Кроме того, отображение $f$
\dfn{однородно степени 1}\index{однородное отображение} по каждому
аргументу (также при фиксированных остальных):
$$
f(\dots,\lambda v_i,\dots) = \lambda f(\dots,v_i,\dots).
$$

Приведем примеры полилинейных отображений, которые мы
встречали раньше:
\begin{itemize}
\item Скалярное произведение: билинейная форма
  $B\colon V\times V\to R$ является полилинейным отображением по самому
  определению (см. определение~\ref{def:bilinear_form}).
\item Определитель: пусть $V = k^n$~--- пространство столбцов высоты
  $n$. Можно рассмотреть отображение
  $$
  \det\colon k^n\times\dots\times k^n\to k,\quad
  (v_1,\dots,v_n)\mapsto\det(v_1,\dots,v_n),
  $$
  сопоставляющий набору столбцов определитель матрицы, составленной из
  этих столбцов. Это отображение полилинейно
  (см. раздел~\ref{ssect:det}).
\end{itemize}

Оказывается, что полилинейные отображения из $V_1\times\dots\times V_m$ в
$U$ в точности соответствуют {\em линейными} отображениям из
некоторого нового объекта (тензорного произведения пространств
$V_1,\dots,V_m$) в $U$.

\subsection{Тензорное произведение двух пространств}

\literature{[F], гл. XIV, \S~4, пп. 1, 2; [K2], гл. 6, \S~1, п. 5; [KM], ч. 4, \S~1, пп. 2--5.}

\begin{definition}\label{def:tensor_product_2}
Пусть $V,W$~--- векторные пространства над полем $k$. 
\dfn{Тензорным произведением}\index{тензорное произведение}
пространств $V$ и $W$ называется векторное пространство $V\otimes W$
вместе с билинейным отображением $\ph\colon V\times W\to V\otimes W$,
удовлетворяющие следующему {\em универсальному свойству}:
\begin{itemize}
\item для любого векторного пространства $U$ и любого билинейного
  отображения $\psi\colon V\times W\to U$ существует единственное
  линейное отображение $\tld\psi\colon V\otimes W\to U$ такое, что
  $\psi = \tld\psi\circ\ph$.
\end{itemize}
\end{definition}
Универсальное свойство можно изобразить следующей диаграммой:
$$
\begin{tikzcd}
V\times W\arrow{rr}{\ph}\arrow{rd}[swap]{\psi} &
& V\otimes W\arrow[dashed]{dl}{\tld\psi} \\
& U
\end{tikzcd}
$$
\begin{theorem}\label{thm:tensor_product}
Тензорное произведение любых векторных пространств $V,W$ над полем $k$
существует и единственно с точностью до канонического
изоморфизма. Последнее означает, что если $\ol\ph\colon V\times W\to
V\ol\otimes W$~--- еще одно тензорное произведение в смысле
определения~\ref{def:tensor_product_2}, то существует единственный
изоморфизм векторных пространств $\alpha\colon V\otimes W\to
V\ol\otimes W$ такой, что $\ol\ph = \alpha\circ\ph$:
$$
\begin{tikzcd}
V\times W \arrow{rr}{\ph} \arrow{dr}[swap]{\ol\ph}
& & V\otimes W \arrow{dl}{\alpha} \\
& V\ol\otimes W
\end{tikzcd}
$$
\end{theorem}
\begin{proof}
Сначала докажем единственность. Итак, пусть $\ph\colon V\times W\to
V\otimes W$ и $\ol\ph\colon V\times W\to V\ol\otimes W$~--- два
тензорных произведения пространств $V$ и $W$. Рассмотрим следующую
диаграмму:
$$
\begin{tikzcd}
V\times W\arrow{rr}{\ph} \arrow{rd}[swap]{\ol\ph} & &
V\otimes W \\
& V\ol\otimes W
\end{tikzcd}
$$
Поскольку $V\otimes W$ является тензорным произведением $V$ и $W$,
можно подставить в универсальное свойство $U = V\ol\otimes W$ и $\psi
= \ol\ph$. Значит, существует единственное линейное отображение
$\alpha\colon V\otimes W\to V\ol\otimes W$, для которого $\ol\ph =
\alpha\circ\ph$. Осталось доказать, что $\alpha$ является
изоморфизмом. Для этого мы построим отображение, обратное к
$\alpha$. Рассмотрим диаграмму
$$
\begin{tikzcd}
V\times W \arrow{rr}{\ol\ph} \arrow{rd}[swap]{\ph} & &
V\ol\otimes W \\
& V\otimes W
\end{tikzcd}
$$
Поскольку $V\ol\otimes W$ также является тензорным произведением $V$ и
$W$, можно подставить в универсальное свойство $U = V\otimes W$ и
$\psi = \ph$. Значит, существует единственное линейное отображение
$\beta\colon V\ol\otimes W\to V\otimes W$ такое, что
$\ph = \beta\circ\ol\ph$. Покажем, что $\beta$ является обратным к
$\alpha$.
Рассмотрим диаграмму
$$
\begin{tikzcd}
V\times W \arrow{rr}{\ph} \arrow{rd}[swap]{\ph} & & V\otimes W\\
& V\otimes W
\end{tikzcd}
$$
Из универсального свойства для $V\otimes W$ следует, что существует
единственное линейное отображение $V\otimes W\to V\otimes W$,
композиция которого с $\ph$ равна $\ph$. Но мы знаем два таких
отображения: одно из них тождественное, $\id_{V\otimes W}$, а другое
равно композиции $\beta\circ\alpha$. Действительно,
$(\beta\circ\alpha)\circ\ph = \beta\circ\ol\ph = \ph$.
Из единственности в универсальном свойстве следует, что эти
отображения должны совпадать. Поэтому $\beta\circ\alpha =
\id_{V\otimes W}$. Аналогичное соображение для $V\ol\otimes W$
показывает, что $\alpha\circ\beta = \id_{V\ol\otimes W}$.

Для доказательства существования тензорного произведения мы приведем
явную конструкцию.
Рассмотрим вспомогательное векторное пространство $L$, базис
которого состоит из всевозможных выражений вида <<$v\otimes w$>> для
всех векторов $v\in V$, $w\in W$. Иными словами, $L$~--- это множество
всех [конечных] формальных линейных комбинаций выражений вида
<<$v\otimes w$>> (с коэффициентами из $k$) с очевидными операциями
суммы и умножения на скаляры.

Несложно определить отображение $f\colon V\times W\to L$: положим
$f(v,w) = \mbox{<<}v\otimes w\mbox{>>}$. Однако, это отображение не
является билинейным: например, $f(v_1+v_2,w) =
\mbox{<<}(v_1+v_2)\otimes w\mbox{>>}$, в то время как
$f(v_1,w) + f(v_2,w) = \mbox{<<}v_1\otimes w\mbox{>>} +
\mbox{<<}v_2\otimes w\mbox{>>}$.
В нашем пространстве $\mbox{<<}(v_1+v_2)\otimes w\mbox{>>}\neq
\mbox{<<}v_1\otimes w\mbox{>>} + 
\mbox{<<}v_2\otimes w\mbox{>>}$, поскольку равенство означало бы
наличие линейной комбинации между базисными элементами.
Кроме того,
$f(\lambda v,w) = \mbox{<<}(\lambda v)\otimes w\mbox{>>}$, но
$\lambda f(v,w) = \lambda\mbox{<<}v\otimes w\mbox{>>}$.
Для того, чтобы исправить это, мы профакторизуем по всем таким
соотношениям, и в полученном фактор-пространстве нужные выражения
совпадут.
А именно, обозначим через $R$ линейную оболочку в $L$ следующих векторов:
\begin{align*}
& \mbox{<<}(v_1+v_2)\otimes w\mbox{>>} - \mbox{<<}v_1\otimes w\mbox{>>} - 
\mbox{<<}v_2\otimes w\mbox{>>},\\
& \mbox{<<}(\lambda v)\otimes w\mbox{>>} - \lambda\mbox{<<}v\otimes w\mbox{>>},\\
& \mbox{<<}v\otimes (w_1+w_2)\mbox{>>} - \mbox{<<}v\otimes w_1\mbox{>>} -
\mbox{<<}v\otimes w_2\mbox{>>},\\
& \mbox{<<}v\otimes (\lambda w)\mbox{>>} - \lambda\mbox{<<}v\otimes w\mbox{>>}
\end{align*}
для всех $v_1,v_2,v,w_1,w_2,w\in V$ и $\lambda\in k$.
Рассмотрим фактор-пространство $L/R$ и покажем, что
оно удовлетворяет определению тензорного произведения $V$
и $W$. Нам еще нужно построить билинейное отображение
$\ph\colon V\times W\to L/R$; для этого рассмотрим композицию $f$ и
канонической проекции $\pi\colon L\to L/R$. Проверим, что $\ph$
билинейно. Например, $\ph(v_1+v_2,w)-\ph(v_1,w)-\ph(v_2,w) = 
\pi(\mbox{<<}(v_1+v_2)\otimes w\mbox{>>}) -
\pi(\mbox{<<}v_1\otimes w\mbox{>>}) -
\pi(\mbox{<<}v_2\otimes w\mbox{>>})
= \pi(\mbox{<<}(v_1+v_2)\otimes w\mbox{>>}-
\mbox{<<}v_1\otimes w\mbox{>>} -
\mbox{<<}v_2\otimes w\mbox{>>}) = 0$, поскольку выражение в скобках
лежит в $R$. Аналогично проверяется однородность и линейность по
второму аргументу.

Наконец, проверим универсальное свойство.
Пусть $\psi\colon V\times W\to U$~--- билинейное отображение.
По универсальному свойству базиса
(теорема~\ref{thm:universal-basis-property}) существует единственное
линейное отображение $\psi'\colon L\to U$ такое, что $\psi=\psi'\circ
f$. Для того, чтобы это отображение <<пропустить>> через
фактор-пространство
$L/R$, достаточно проверить, что отображение $\psi'$ переводит каждый
элемент $R$ в $0$ (в этом случае отображение $L/R\to U$,
$x+R\mapsto \psi'(x)$ корректно определено).
Но для этого достаточно проверить, что $\psi'$ переводит каждый
элемент из нашей системы, порождающей пространство $R$, в $0$.
Это очевидно в силу билинейности $\psi$; например,
\begin{align*}
\psi'(\mbox{<<}(v_1+v_2)\otimes w\mbox{>>} -
\mbox{<<}v_1\otimes w\mbox{>>} -
\mbox{<<}v_2\otimes w\mbox{>>})
&= \psi'(f(v_1+v_2,w)-f(v_1,w)-f(v_2,w)) \\
&= \psi'(f(v_1+v_2,w))-\psi'(f(v_1,w))-\psi'(f(v_2,w))\\
&= \psi(v_1+v_2,w) - \psi(v_1,w) - \psi(v_2,w)\\
&= 0.
\end{align*}
Таким образом, мы построили отображение
$\tld\psi\colon L/R = V\otimes W\to U$, для которого $\tld\psi\circ\ph
= \psi$. Для доказательства единственности осталось заметить, что
элементы вида $\ph(v,w)$ для $u\in V$, $w\in W$ являются образами в
$L/R$ базисных элементов пространства $L$. Поэтому такие элементы
порождают $U\otimes V$. Значит, линейное отображение $\tld\psi\colon
V\otimes W\to U$ полностью определяется своими значениями на таких
элементах: $\tld\psi(\ph(v,w)) = \psi(v,w)$.
\end{proof}

Итак, мы построили векторное пространство $V\otimes W$ вместе с
билинейным отображением $\ph\colon V\times W\to V\otimes W$. Слово
<<универсальность>> в названии универсального свойства означает, что
билинейное отображение $\ph$ универсально среди всех билинейных
отображений из $V\times W$ в следующем смысле: любое билинейное
отображение из $V\times W$ пропускается через $\ph$ (является
композицией $\ph$ и некоторого линейного отображения).

Элементы пространства $V\otimes W$ называются
\dfn{тензорами}\index{тензор}.
Образ пары $(v,w)$ под действием $\ph$ мы будем обозначать через
$v\otimes w\in V\otimes W$ и называть
\dfn{разложимым тензором}\index{тензор!разложимый}. Из определения
немедленно следует,
что $(v_1+v_2)\otimes w = v_1\otimes w + v_2\otimes w$,
$v\otimes(w_1+w_2) = v\otimes w_1 + v\otimes w_2$,
$(\lambda v)\otimes w = \lambda (v\otimes w) = u\otimes (\lambda v)$.
Заметим, однако, что (как правило) не любой тензор является
разложимым. В то же время, множество всех разложимых тензоров является
системой образующих пространства $V\otimes W$, поскольку это образы
базисных элементов пространства $L$ в нашей конструкции. В частности,
любой тензор является {\it суммой} конечного числа
разложимых. Поэтому, например, для задания линейного отображения из
$V\otimes W$ достаточно задать его на разложимых тензорах (на самом
деле, это еще одна переформулировка универсального свойства). Точнее,
если мы сопоставили каждому разложимому тензору $v\otimes w\in
V\otimes W$ некоторый элемент пространства $U$ {\em билинейным
  образом}, то однозначно определено линейное отображение $V\otimes
W\to U$.

Отметим, что приведенная в доказательстве
теоремы~\ref{thm:tensor_product} конструкция совершенно чудовищна:
даже если пространства $V$ и $W$ конечномерны, по пути к $V\otimes W$
мы строим пространство $L$, которое, как правило, бесконечномерно:
даже если $\dim(V)=\dim(W)=1$ и $k=\mb R$, базис пространства $L$
имеет мощность континуума. На самом деле, тензорное произведение
конечномерных пространств конечномерно; если в пространствах $V$ и $W$
выбраны базисы, то и в $V\otimes W$ естественным образом возникает
базис.

\begin{proposition}\label{prop:tensor_product_basis}
Пусть $V,W$~--- векторные пространства над полем $k$, и пусть
$\mc B=\{e_1,\dots,e_m\}$~--- базис $V$,
$\mc C=\{f_1,\dots,f_n\}$~--- базис $W$.
Тогда элементы вида $e_i\otimes f_j$, $1\leq i\leq m$, $1\leq j\leq
n$, образуют базис пространства $V\otimes W$.
\end{proposition}
\begin{proof}
Рассмотрим пространство $X$ размерности $mn$, базис которого состоит
из элементов вида $e_i\otimes f_j$. Сейчас мы определим билинейное
отображение $V\times W\to X$ и проверим, что $X$ вместе с этим
отображением удовлетворяет универсальному свойству тензорного
произведения.

Для определения $\ph$ сначала положим $\ph(e_i,f_j) = e_i\otimes f_j$.
Для двух произвольных векторов $v = \sum_i\lambda_i e_i\in V$
и $w = \sum_j\mu_j f_j\in W$ теперь определим $\ph(v,w)$ так,
чтобы $\ph$ было билинейным. Раскрывая скобки, получаем, что
$\ph(v,w) = \sum_{i,j}\lambda_i\mu_j e_i\otimes f_j$.
Очевидно, что построенное отображение $\ph\colon V\times W\to X$
билинейно.

Пусть теперь $U$~--- еще одно векторное пространство над $k$, и пусть
$\psi\colon V\times W\to U$~--- билинейное отображение. Так как
векторы $e_i\otimes f_j$ образуют базис пространства $X$, для
определения линейного отображения $\tld\psi\colon X\to U$ мы можем
задать его значения на этих векторых произвольным образом; полученное
линейное отображение определяется этим однозначно
(теорема~\ref{thm:universal-basis-property}).
Поэтому положим $\tld\psi(e_i\otimes f_j) = \psi(e_i,f_j)$ и продолжим
$\tld\psi$ до линейного отображения $X\to U$. Композиция
$\tld\psi\circ\ph$ билинейна и совпадает с $\psi$ на парах $(e_i,f_j)$,
поэтому $\tld\psi\circ\ph = \psi$. Вместе с тем, любое отображение,
композиция которого с $\ph$ равна $\psi$, должно на базисных векторах
$\ph(e_i,f_j)$ принимать значения $\psi(e_i,f_j)$, поэтому такое
отображение единственно.
\end{proof}

\begin{definition}\label{dfn:tensor_basis}
Базис из предложения~\ref{prop:tensor_product_basis} называется
\dfn{тензорным базисом}\index{тензорный базис} пространства $V\otimes
W$. Обычно мы
упорядочиваем его следующим ({\em лексикографическим}) образом:
$e_1\otimes f_1$, $e_1\otimes f_2$, \dots, $e_1\otimes f_n$, \dots,
$e_m\otimes f_1$, $e_m\otimes f_2$, \dots, $e_m\otimes f_n$.
\end{definition}

\begin{corollary}
Если пространства $V,W$ над полем $k$ конечномерны, то $V\otimes W$
конечномерно и $\dim(V\otimes W) = \dim(V)\cdot\dim(W)$.
\end{corollary}

\begin{remark}
Сравните формулу для размерности тензорного произведения с формулой
для прямой суммы: $\dim(V\oplus W) = \dim(V) + \dim(W)$. Это
свидетельство того, что тензорное произведение и прямая сумма~---
аналоги умножения и сложения для векторных пространств.
\end{remark}

\subsection{Тензорное произведение нескольких пространств}

\literature{[F], гл. XIV, \S~4, п. 3; [KM], ч. 4, \S~1, пп. 2--5;
  \S~2, пп. 1--3.}

Мы можем теперь попытаться определить тензорное произведение
{\it трех} пространств $U,V,W$ формулой $U\otimes V\otimes W =
(U\otimes V)\otimes W$. Однако, такое определение нарушает симметрию
между $U$, $V$ и $W$ (почему не $U\otimes (V\otimes W)$?). Поэтому мы
просто повторим универсальное определение тензорного произведения,
изменив его соответствующим образом.

Пусть $V_1,\dots,V_s$~--- векторные пространства над полем $k$. Тогда
их \dfn{тензорным
произведением}\index{тензорное произведение!нескольких пространств}
называется векторное пространство $V_1\otimes\dots\otimes V_s$ над $k$
вместе с полилинейным отображением
$\ph\colon V_1\times\dots\times V_s\to V_1\otimes\dots\otimes V_s$
таким, что для любого полилинейного отображения
$\psi\colon V_1\times\dots\times V_s\to U$ в некоторое векторное
пространство $U$ существует единственное линейное отображение
$\tld\psi\colon V_1\otimes\dots\otimes V_s\to U$ такое,
что $\psi = \tld\psi\circ\ph$:
$$
\begin{tikzcd}
V_1\times\dots\times V_s \arrow{rr}{\ph} \arrow{rd}[swap]{\psi}
& & V_1 \otimes\dots\otimes V_s \arrow[dashed]{ld}{\tld\psi} \\
& U
\end{tikzcd}
$$

\begin{theorem}
Тензорное произведение любого конечного числа векторных пространств
$V_1,\dots,V_s$ существует и единственно с точностью до канонического
изоморфизма.
\end{theorem}
\begin{proof}
Доказательство этой теоремы совершенно такое же, как в случае двух
пространств (теорема~\ref{thm:tensor_product}).
А именно, рассмотрим векторное пространство $L$ с
базисом, состоящим из элементов
$\mbox{<<}v_1\otimes\dots\otimes v_s\mbox{>>}$, где $v_1,\dots,v_s$
пробегают всевозможные наборы элементов пространств $V_1,\dots,V_s$,
соответственно. Имеется естественное отображение множеств
$V_1\times\dots\times V_s\to L$, переводящее набор
$(v_1,\dots,v_s)$ в базисный элемент
$\mbox{<<}v_1\otimes\dots\otimes v_s\mbox{>>}$. Чтобы сделать это
отображение полилинейным, профакторизуем $L$ по линейной оболочке $R$
следующих элементов:
\begin{align*}
&\mbox{<<}\dots\otimes v_i+v'_i\otimes\dots\mbox{>>} - 
\mbox{<<}\dots\otimes v_i\otimes\dots\mbox{>>} - 
\mbox{<<}\dots\otimes v'_i\otimes\dots\mbox{>>};\\
&\mbox{<<}\dots\otimes \lambda v_i\otimes\dots\mbox{>>} - 
\lambda\mbox{<<}\dots\otimes v_i\otimes\dots\mbox{>>}.
\end{align*}
Теперь сквозное отображение $\ph\colon V_1\times\dots\times V_s\to
L\to L/R$ полилинейно. Проверим, что оно универсально:
пусть $\psi\colon V_1\times\dots\times V_s\to U$~--- некоторое
полилинейное отображение.
Сопоставление $\mbox{<<}v_1\otimes\dots\otimes v_s\mbox{>>} \mapsto
\psi(v_1,\dots,v_s)$ задает линейное отображение $L\to U$, и элементы,
порождающие $R$, переходят в $0$ в силу полилинейности $\psi$. Поэтому
оно пропускается через фактор-пространство и мы получаем линейное
отображение $L/R\to U$. Таким образом, мы можем положить
$V_1\otimes\dots\otimes V_s = L/R$. Единственность тензорного
произведения доказывается буквально так же, как и в случае двух
пространств.
\end{proof}

\begin{remark}
Как и в случае двух пространств, образ набора $(v_1,\dots,v_s)\in
V_1\times\dots\times V_s$ в пространстве $V_1\otimes\dots\otimes V_s$
обозначается через $v_1\otimes\dots\otimes v_s$ и называется
\dfn{разложимым тензором}\index{тензор!разложимый};
 для задания линейного отображения из
$V_1\otimes\dots\otimes V_s$ в $U$ достаточно определить его на
разложимых тензорах билинейным образом. Проиллюстрируем это на примере
доказательства следующей теоремы.
\end{remark}

\begin{proposition}\label{prop:tensor_assoc_and_comm}
Тензорное произведение векторных пространств ассоциативно и
коммутативно с точностью
до канонических изоморфизмов: а именно, для любых трех векторных
пространств $U,V,W$ имеют место канонические изоморфизмы
$(U\otimes V)\otimes W \isom U\otimes V\otimes W \isom U\otimes
(V\otimes W)$ и $U\otimes V \isom V\otimes U$.
\end{proposition}
\begin{proof}
Определим отображение
$U\otimes V\otimes W\to (U\otimes V)\otimes W$
на разложимых тензорах формулой
$u\otimes v\otimes w\mapsto (u\otimes v)\otimes w$.
Эта формула задает линейные отображения, и той же формулой,
прочитанной справа налево, задается отображение в обратную
сторону. Очевидно, что композиция этих отображений
$U\otimes V\otimes W\to (U\otimes V)\otimes W\to
U\otimes V\otimes W$ тождественна на
разложимых тензорах, и потому тождественна на всем пространстве.
Аналогично доказывается изоморфизм
$U\otimes V\otimes W\isom U\otimes (V\otimes W)$.
Для задания отображения $U\otimes V\to V\otimes U$ отправим
$u\otimes v$ в $v\otimes u$; доказательство завершается так же.
\end{proof}

\begin{proposition}
Пусть $V_1,\dots,V_s$~--- векторные пространства над полем $k$
размерностей $n_1,\dots,n_s$;
$\mc B_j=\{e^j_1,\dots,e^j_{n_j}\}$~--- базис $V_j$ для каждого
$j=1,\dots,s$.
Тогда элементы вида $e^1_{i_1}\otimes\dots\otimes e^s_{i_s}$, где
$1\leq i_k\leq n_k$ для всех $k=1,\dots,s$, образуют базис
пространства $V_1\otimes\dots\otimes V_s$.
\end{proposition}
\begin{proof}
Мы можем повторить доказательство
предложения~\ref{prop:tensor_product_basis}. А именно, рассмотрим
векторное пространство $W$ над $k$, базисом которого являются формальные
символы вида $e^1_{i_1}\otimes\dots\otimes e^s_{i_s}$. Определим
полилинейное отображение $\ph\colon V_1\times\dots\times V_s\to W$
следующим образом: набор базисных векторов
$(e^1_{i_1},\dots,e^s_{i_s})\in V_1\times\dots\times V_s$
отправим в базисный элемент $e^1_{i_1}\otimes\dots\otimes e^s_{i_s}$,
а дальше продолжим по полилинейности.
А именно,
если $(v_1,\dots,v_s)\in V_1\times\dots\times V_s$~--- набор
векторов, разложим каждый $v_j$ по базису $\mc B_j$. Получим равенства
вида $v_j = \sum_{i_j=1}^{n_j} e^j_{i_j} a_{i_j,j}$.
Положим
\begin{align*}
\ph(v_1,\dots,v_s) &= \ph(\sum_{i_1=1}^{n_1} e^1_{i_1}a_{i_1,1},
\dots,\sum_{i_s=1}^{n_s} e^s_{i_s}a_{i_s,s}) \\
&= \sum_{i_1=1}^{n_1}\dots\sum_{i_s=1}^{n_s}a_{i_1,1}\dots
a_{i_s,s}\ph(e^1_{i_1},\dots,e^s_{i_s}) \\
& = \sum_{i_1=1}^{n_1}\dots\sum_{i_s=1}^{n_s}a_{i_1,1}\dots
a_{i_s,s} e^1_{i_1}\otimes\dots\otimes e^s_{i_s}.
\end{align*}
Очевидно, что это отображение полилинейно; покажем, что пространство
$W$ вместе с $\ph$ удовлетворяет универсальному свойству из
определения тензорного произведения. Пусть $U$~--- произвольное
векторное пространство над $k$, и
$\psi\colon V_1\times\dots\times V_s\to U$~--- полилинейное
отображение. Покажем, что оно представляется в виде композиции $\ph$ и
некоторого линейного отображения $\tld\psi$.
Для задания $\tld\psi\colon W\to U$ достаточно задать его
(произвольным образом) на базисе, то есть, на элементах вида
$e^1_{i_1}\otimes\dots\otimes e^s_{i_s}$. Это можно сделать
единственным образом:
положим $\tld\psi(e^1_{i_1}\otimes\dots\otimes e^s_{i_s})
= \psi(e^1_{i_1},\dots, e^s_{i_s})$. Композиция $\tld\psi\circ\ph$,
разумеется, является полилинейным отображением и
совпадает с $\psi$ на наборах вида $(e^1_{i_1},\dots,e^s_{i_s})$, и
цепочка равенств выше показывает, что значение полилинейного
отображения на произвольном наборе $(v_1,\dots,v_s)$ выражается через
его значения на наборах такого вида. Поэтому $\tld\psi\circ\ph$
совпадает с $\psi$. 
\end{proof}

\subsection{Двойственное пространство}

\literature{[vdW], гл. IV, \S~21; [KM], ч. 1, \S~1, п. 9.}

Пусть $V$~--- векторное пространство над полем $k$. Рассмотрим $k$ как
[одномерное] векторное пространство над $k$. Тогда множество
$\Hom(V,k)$ линейных отображений из $V$ в $k$ ({\it линейных функций}
на $V$) само является векторным пространством над $k$
(см. раздел~\ref{subsect:hom_space}). Операции на нем вполне
естественны: сложение функций и умножение функций на скаляры. Это
пространство мы будем обозначать через $V^* = \Hom(V,k)$ и называть
\dfn{пространством, двойственным к $V$}\index{векторное пространство!двойственное}

Пусть теперь $V$~--- {\it конечномерное} векторное пространство над
$k$ и $\mc B = (e_1,\dots,e_n)$~--- базис $V$. По универсальному
свойству базиса (теорема~\ref{thm:universal-basis-property}) для
задания элемента $\ph\in V^* = \Hom(V,k)$ достаточно задать
(произвольным образом) элементы $\ph(e_1),\dots,\ph(e_n)\in k$.

\begin{proposition}
Пусть $V$~--- векторное пространство над $k$ с базисом
$\mc B = (e_1,\dots,e_n)$.
Обозначим через $e_i^*$ функцию $V\to k$, равную $1$ на
базисном векторе $e_i$ и $0$ на всех остальных базисных
векторах. Таким образом, $e_i^*(e_i) = 1$ и $e_i^*(e_j) = 0$ при всех
$j\neq i$.
Тогда $(e^*_1,\dots,e^*_n)$~--- базис пространства $V^*$.
\end{proposition}
\begin{proof}
Пусть $\ph\colon V\to k$~--- произвольный элемент пространства
$V^*$. Мы знаем (теорема~\ref{thm:universal-basis-property}), что
задать $\ph$~--- это то же самое, что задать значения
$\ph(e_1),\dots,\ph(e_n)\in k$. Рассмотрим функцию
$\ph(e_1)e^*_1 + \dots + \ph(e_n)e^*_n$. Покажем, что она совпадает с
$\ph$.
Действительно, для базисного вектора $e_i$ получаем
$(\ph(e_1)e^*_1 + \dots + \ph(e_n)e^*_n)(e_i)
= \ph(e_1)e^*_1(e_i) + \dots + \ph(e_1)e^*_n(e_i)
= \ph(e_i)e^*_i(e_i) = \ph(e_i)$.
Значит, функции $\ph(e_1)e^*_1 + \dots + \ph(e_n)e^*_n$ и $\ph$
совпадают на базисных векторах, а потому совпадают везде. Значит, мы
представили функцию $\ph$ как линейную комбинацию функций
$e^*_i$. Осталось показать, что функции $e^*_i$ линейно независимы.

Действительно, предположим, что $c_1 e^*_1 + \dots + c_n e^*_n =
0$~--- нетривиальная линейная комбинация. Это означает, что
$c_i\neq 0$ при некотором $i$. Но тогда
и $(c_1 e^*_1 + \dots + c_n e^*_n)(e_i) = 0$, а левая часть
равна $c_1 e^*_1(e_i) + \dots + c_n e^*_n(e_i) = c_i\neq 0$~---
противоречие.
\end{proof}

Таким образом, в конечномерном случае пространства $V$ и $V^*$ имеют
одинаковую размерность. Из этого следует, что они изоморфны
(теорема~\ref{thm:isomorphic-iff-equidimensional}). Например, имеется
изоморфизм $V\to V^*$, отправляющий $e_i$ в $\ph_i$ при $i=1,\dots,n$,
если $e_1,\dots,e_n$~--- базис $V$. Однако, этот изоморфизм не
является каноническим, то есть, существенно зависит от выбора базиса.
В то же время, {\it дважды двойственное} пространство
$V^{**} = \Hom(V^*,k)$ {\it канонически} изоморфно $V$.

\begin{proposition}
Рассмотрим отображение $V\to V^{**}$, сопоставляющее вектору $v\in V$
функцию $v^{**}\colon V^*\to k$, заданную равенством $v^{**}(\ph) =
\ph(v)$ для всех $\ph\in V^*$. Если пространство $V$ конечномерно, то
указанное отображение является изоморфизмом.
\end{proposition}
\begin{proof}
Нетрудно проверить, что $v^{**}$ является линейным
отображением $V^*\to k$. Действительно, если $\ph,\psi\in V^*$,
$\lambda\in k$, то
$v^{**}(\ph+\psi) = (\ph+\psi)(v) = \ph(v) + \psi(v) = v^{**}(\ph) +
v^{**}(\psi)$ и $v^{**}(\lambda\ph) = (\lambda\ph)(v) = \lambda\cdot\ph(v)
= \lambda\cdot v^{**}(\ph)$.

Таким образом, $v^{**}\in V^{**}$ для всех $v\in V$. Покажем, что
сопоставление $v\mapsto v^{**}$ линейно зависит от $v$. Необходимо
проверить, что $(v+w)^{**} = v^{**} + w^{**}$ и $(\lambda v)^{**} =
\lambda v^{**}$. Чтобы проверить совпадение двух отображений $V^*\to
k$, достаточно проверить, что результаты их применения к произвольному
элементу $\ph\in V^*$ совпадают:
$(v+w)^{**}(\ph) = \ph(v+w) = \ph(v)+\ph(w) = v^{**}(\ph) +
w^{**}(\ph)$, $(\lambda v)^{**}(\ph) = \ph(\lambda v) =
\lambda\cdot\ph(v) = \lambda\cdot v^{**}(\ph)$.

Мы получили линейное отображение $V\to V^{**}$. Покажем, что оно
инъективно. Для этого достаточно проверить, что его ядро
тривиально. Пусть вектор $v\in V$ таков, что $v^{**}=0$. Это означает,
что $v^{**}(\ph) = 0$ для всех $\ph\in V^*$, то есть, что $\ph(v)=0$
для всех $\ph\colon V\to k$. Покажем, что из этого следует, что
$v=0$. Действительно, если $v\neq 0$, то вектор $v$ можно дополнить до
базиса $(v,e_1,e_2,\dots)$ пространства $V$. Определим функцию
$\ph_v\in V^*$ равенствами $\ph_v(v)=1$, $\ph_v(e_i)=0$ для всех
$i$. По универсальному свойству базиса этого достаточно для
корректного определения линейного отображения $\ph_v\colon V\to k$. По
предположению $\ph_v(v) = 0$, в то время как мы положили
$\ph_v(v) = 1$~--- противоречие.

Наконец, воспользуемся конечномерностью: мы знаем, что $\dim(V^{**}) =
\dim(V^*) = \dim(V)$, и у нас есть инъективное отображение $V\to
V^{**}$. По теореме о гомоморфизме~\ref{thm:homomorphism-linear}
из этого следует, что наше отображение сюръективно
и, стало быть, является изоморфизмом векторных пространств.
\end{proof}

\subsection{Канонические изоморфизмы}

\literature{[KM], ч. 4, \S~2, пп. 4--6.}

\begin{theorem}[Выражение $\Hom$ через $\otimes$]\label{thm:hom_and_otimes}
Для любых конечномерных векторных пространств $U,V$ над $k$ имеет
место канонический изоморфизм
$$
U\otimes V\isom\Hom(U^*,V).
$$ 
\end{theorem}
\begin{proof}
Определим отображение $\eta\colon U\otimes V\to\Hom(U^*,V)$, отправив
разложимый тензор $u\otimes v\in U\otimes V$ в
отображение $U^*\to V$, $\ph\mapsto\ph(u)v$. Написанная формула
билинейно зависит от $u$ и от $v$, поэтому корректно определяет
линейное отображение из тензорного произведения $U\otimes V$.

Покажем, что $\eta$~--- изоморфизм. Для этого выберем базис
$(f_1,\dots,f_m)$ в $U$ и базис $(e_1,\dots,e_n)$ в $V$.
При этом $\{f_j\otimes e_i\}$~--- базис в $U\otimes V$
(предложение~\ref{prop:tensor_product_basis}).
Вспомним, как строится базис пространства $\Hom(U^*,V)$.
Заметим, что в пространстве $U^*$ у нас есть базис
$(\ph_1,\dots,\ph_m)$, двойственный базису $(f_1,\dots,f_m)$.
Как мы знаем из теоремы~\ref{thm:hom-isomorphic-to-m},
после выбора базисов в $U^*$ и $V$ пространство $\Hom(U^*,V)$
оказывается изоморфно пространству матриц $M(n,m,k)$,
а в этом пространстве имеется стандартный базис из матричных
единиц. Матричная единица $E_{ij}$ соответствует отображению
$U^*\to V$, которое $\ph_j$ переводит в $e_i$, а все остальные
базисные векторы $\ph_h$, $h\neq j$, отправляет в $0$. Обозначим это
отображение через $a_{ij}$.

Мы утверждаем, что отображение $\eta$ переводит $f_j\otimes e_i$ в
$a_{ij}$.
Действительно, по нашему определению $f_j\otimes e_i$ переводится
в отображение $U^*\to V$, $\ph\mapsto\ph(f_j)e_i$. Проверим, что это и
есть $a_{ij}$. Действительно, $\ph_j\mapsto\ph_j(f_j)e_i = e_i$
и $\ph_h\mapsto\ph_h(f_j)e_i = 0$ при $h\neq j$.

Таким образом, отображение $\eta$ переводит базис пространства
$U\otimes V$ в базис пространства $\Hom(U^*,V)$, а потому биективно.
\end{proof}

\begin{corollary}\label{cor:hom_and_otimes_2}
Для любых конечномерных векторных пространств $U,V$ над $k$ имеет
место канонический изоморфизм
$$
U^*\otimes V\isom\Hom(U,V).
$$
\end{corollary}
\begin{proof}
Применим предыдущую теорему к $U^*$ и $V$:
$U^*\otimes V \isom \Hom((U^*)^*,V) \isom \Hom(U,V)$.
\end{proof}

\begin{corollary}\label{cor:u_otimes_k}
Для любого конечномерного векторного пространства $U$ над $k$ имеет
место канонический изоморфизм
$U\otimes k\isom U$.
\end{corollary}
\begin{proof}
По теореме~\ref{thm:hom_and_otimes} есть канонический изоморфизм
$U\otimes k\isom\Hom(U^*,k)$; правая часть по определению равна
$(U^*)^*\isom U$.
\end{proof}

\begin{theorem}[Двойственность и $\otimes$]\label{thm:duality_and_otimes}
Для любых конечномерных векторных пространств $U,V$ над $k$ имеет
место канонический изоморфизм
$$
(U\otimes V)^*\isom U^*\otimes V^*.
$$
\end{theorem}
\begin{proof}
Зададим отображение $U^*\otimes V^*\to (U\otimes V)^*$. Как всегда,
достаточно определить его на разложимых тензорах
$\ph\otimes\psi\in U^*\otimes V^*$. Образом этого тензора должен быть
элемент пространства $(U\otimes V)^*$, то есть, линейное отображение
$U\otimes V\to k$, которое достаточно задать на разложимых тензорах
$u\otimes v\in U\otimes V$. Отправим такой тензор в
$\ph(u)\psi(v)\in k$.
Очевидно, что написанное выражение билинейно зависит от $(u,v)$,
потому определяет элемент пространства $(U\otimes V)^*$. С другой
стороны, этот элемент билинейно зависит от $(\ph,\psi)$.
Итак, мы построили линейное отображение
$\eta\colon U^*\otimes V^*\to (U\otimes V)^*$:
отправляющее $\ph\otimes\psi$ в линейное отображение
$u\otimes v\mapsto \ph(u)\psi(v)$.

Покажем, что построенное отображение является изоморфизмом. Для этого
выберем базис $(f_1,\dots,f_m)$ в пространстве $U$ и базис
$(e_1,\dots,e_n)$ в пространстве $V$. Тогда в пространствах $U^*$ и
$V^*$ возникают двойственные базисы: $(f_1^*,\dots,f_m^*)$ и
$(e_1^*,\dots,e_n^*)$, соответственно. Поэтому в пространстве
$U^*\otimes V^*$ естественно взять тензорное произведение этих
двойственных базисов $(f_j^*\otimes e_i^*)$. С другой стороны, в
пространстве $(U\otimes V)^*$ естественно выбрать базис, двойственный
к тензорному произведению исходных базисов $U$ и $V$:
$(f_j\otimes e_i)^*$.

Покажем, что при нашем линейном отображении
$\eta$ базисный элемент $f_j^*\otimes e_i^*$ переходит в базисный
элемент $(f_j\otimes e_i)^*$. Действительно,
по определению $\eta(f_j^*\otimes e_i^*)$~--- это линейное
отображение, отправляющее $u\otimes v$ в $f_j^*(u)e_i^*(v)$. Если мы
подставим в него $u=f_j$ и $v=e_i$, то получим $f_j^*(f_j)e_i^*(e_i) =
1$; если же подставим любую другую пару $u=f_k$, $v=e_h$ (где $k\neq
j$ или $h\neq i$), то получим $f_j^*(f_k)e_i^*(e_h) = 0$, поскольку
хотя бы один сомножитель равен нулю. Значит, $\eta(f_j^*\otimes
e_i^*)$ переводит базисный элемент $f_j\otimes e_i\in U\otimes V$ в
$1$, а все остальные базисные элементы в $0$. Но $(f_j\otimes e_i)^*$
действует ровно так же на базисных элементах, поэтому
$\eta(f_j^*\otimes e_i^*) = (f_j\otimes e_i)^*$, что и требовалось.
Таким образом, $\eta$ переводит базис в базис, и потому является
изоморфизмом.
\end{proof}

\begin{corollary}
Для любых конечномерных векторных пространств $U_1,\dots,U_s$ над $k$
имеет место канонический изоморфизм
$$
(U_1\otimes\dots\otimes U_s)^*\isom U_1^*\otimes\dots\otimes U_s^*.
$$
\end{corollary}
\begin{proof}
По индукции из теоремы~\ref{thm:duality_and_otimes} и
предложения~\ref{prop:tensor_assoc_and_comm}.
\end{proof}

\begin{theorem}[Сопряженность $\otimes$ и $\Hom$]\label{thm:otimes_hom_adjoint}
Для любых конечномерных векторных пространств $U,V,W$ над $k$ имеет
место канонический изоморфизм
$$
\Hom(U\otimes V,W)\isom\Hom(U,\Hom(V,W)).
$$
\end{theorem}
\begin{proof}
Заметим сначала, что размерности обеих частей равны
$\dim(U)\cdot\dim(V)\cdot\dim(W)$. Рассмотрим произвольный элемент
$\ph\in\Hom(U,\Hom(V,W))$. Он сопоставляет (линейным образом)
каждому элементу $u\in U$ некоторое линейное отображение
$\ph_u\colon V\to W$, $v\mapsto\ph_u(v)$. Построим теперь по этому
элементу $\ph$ линейное отображение из $U\otimes V$ в $W$ следующим
образом: разложимый тензор $u\otimes v\in U\otimes V$ отправим в
$\ph_u(v)\in W$. Это сопоставление билинейно зависит от $u$ и от $v$,
(поскольку $\ph$ и $\ph_u$ линейны), и потому мы получили однозначно
определенное линейное отображение $\eta(\ph)\colon U\otimes V\to W$,
то есть, элемент $\Hom(U\otimes V, W)$. При этом сопоставление
$\ph\mapsto\eta(\ph)$ является, очевидно, линейным.
Наконец, покажем, что $\eta$ является инъекцией. Предположим, что
$\eta(\ph)=0$, то есть, $\eta(\ph)(u\otimes v)=0$ для всех $u\in U$,
$v\in V$. Но по нашему определению $\eta(\ph)(u\otimes v) = \ph_u(v)$;
поэтому $\ph_u(v)=0$ при всех $u\in U$, $v\in V$, откуда $\ph_u=0$ при
всех $u\in U$, откуда $\ph=0$.
Теперь из инъективности $\eta$ и совпадения размерностей следует, что
$\eta$ и сюръективно, а потому является изоморфизмом.
\end{proof}

На самом деле в доказательстве этой теоремы можно было, как и раньше,
выбрать базисы в $U,V,W$, получить базисы во всех фигурирующих в
формулировке пространствах, и честно проверить, что построенное
отображение $\eta$ переводит базис в базис. Еще один вариант
доказательства теоремы~\ref{thm:otimes_hom_adjoint}~---
воспользоваться уже доказанными изоморфизмами:
$\Hom(U\otimes V,W)\isom (U\otimes V)^*\otimes W\isom
(U^*\otimes V^*)\otimes W\isom U^*\otimes(V^*\otimes W)
\isom U^*\otimes\Hom(V,W) \isom\Hom(U,\Hom(V,W))$

\subsection{Тензорное произведение линейных отображений}

\literature{[K2], гл. 6, \S~1, пп. 2, 5; [KM], ч. 4, \S~2, п. 7.}

Пусть $\ph\colon U\to V$, $\psi\colon W\to Z$~--- линейные
отображения. Сейчас мы определим их \dfn{тензорное
  произведение}\index{тензорное произведение!линейных отображений}
$\ph\otimes\psi$, которое будет линейным отображением из $U\otimes W$
в $V\otimes Z$.
Сопоставим разложимому тензору $u\otimes w\in U\otimes W$
разложимый тензор $\ph(u)\otimes\psi(w)\in V\otimes Z$. Нетрудно
видеть, что это сопоставление ведет себя билинейно по $u$ и по $w$, и
потому задает корректно определенное линейное отображение
$$\ph\otimes\psi\colon U\otimes W\to V\otimes Z.$$
Покажем, что это определение обладает естественными свойствами.

\begin{theorem}\label{thm:tensor_product_maps}
Тензорное произведение линейных отображений обладает следующими
свойствами:
\begin{enumerate}
\item $(\ph'\ph)\otimes(\psi'\psi) =
  (\ph'\otimes\psi')(\ph\otimes\psi)$;
\item $\id_U\otimes\id_V = \id_{U\otimes V}$;
\item $(\ph+\ph')\otimes\psi = \ph\otimes\psi + \ph'\otimes\psi$;
\item $\ph\otimes(\psi+\psi') = \ph\otimes\psi + \ph\otimes\psi'$;
\item $(\lambda\ph)\otimes\psi = \lambda(\ph\otimes\psi) = \ph\otimes(\lambda\psi)$.
\end{enumerate}
\end{theorem}
\begin{proof}
Мы проверим самое сложное свойство~--- первое.
Пусть $U\stackrel{\ph}{\to} V \stackrel{\ph'}{\to} V'$,
$W\stackrel{\psi}{\to} Z \stackrel{\psi'}{\to} Z'$~--- линейные
отображения.
Выберем векторы $u\in U$, $w\in W$ и применим
$(\ph'\ph)\otimes(\psi'\psi)$ к разложимому тензору $u\otimes w$. По
определению получаем
$$
((\ph'\ph)\otimes(\psi'\psi))(u\otimes w) =
(\ph'\ph)(u)\otimes(\psi'\psi)(w) =
\ph'(\ph(u))\otimes\psi'(\psi(w)).
$$
С другой стороны,
$$
(\ph'\otimes\psi')(\ph\otimes\psi)(u\otimes w) =
(\ph'\otimes\psi')(\ph(u)\otimes\psi(w)) =
\ph'(\ph(u))\otimes\psi'(\psi(w)).
$$
Значит, два указанных отображения совпадают на всех разложимых
тензорах, а потому равны.
\end{proof}

\begin{theorem}
Для любых конечномерных векторных пространств $U,V,W,Z$ над $k$ имеет
место канонический изоморфизм
$$\Hom(U\otimes W,V\otimes Z) \isom \Hom(U,V)\otimes\Hom(W,Z).$$
\end{theorem}
\begin{proof}
Мы построили отображение
$\Hom(U,V)\times\Hom(W,Z)\to\Hom(U\otimes W,V\otimes Z)$,
$(\ph,\psi)\mapsto\ph\otimes\psi$.
По теореме~\ref{thm:tensor_product_maps} это сопоставление билинейно,
поэтому определяет линейное отображение
$\Hom(U,V)\otimes\Hom(W,Z) \to \Hom(U\otimes W,V\otimes Z)$, и обычные
рассуждения (например, выбор базисов во всех указанных пространствах)
убеждают нас, что получился изоморфизм.
Еще один способ доказательства~--- воспользоваться уже доказанными
изоморфизмами:
$$\Hom(U\otimes W,V\otimes Z) \isom (U\otimes W)^*\otimes (V\otimes Z)
\isom (U^*\otimes V)\otimes (W^*\otimes Z) \isom
\Hom(U,V)\otimes\Hom(W,Z).$$
\end{proof}

Выясним, как выглядит матрица тензорного произведения линейных
отображений.
Пусть вообще $x\in M(l,m,k)$, $y\in M(n,p,k)$~--- две произвольные
матрицы над полем $k$. Определим \dfn{кронекерово
  произведение}\index{кронекерово произведение} матриц
$x$ и $y$ как матрицу $x\otimes y\in M(lm,np,k)$, которую проще всего
представлять себе блочной матрицей
$$
x\otimes y = \begin{pmatrix}x_{11}y & \dots & x_{1m}y\\
\vdots & \ddots & \vdots\\
x_{l1}y & \dots & x_{lm}y\end{pmatrix}.
$$
Обратите внимание, что кронекерово произведение матриц мы обозначаем
тем же значком $\otimes$, что и тензорное произведение. Это не
случайно: заметим пока, что кронекерово произведение обладает многими
обычными свойствами тензорного произведения.

\begin{proposition}[Свойства кронекерова
  произведения]\label{prop:kronecker_product}
\hspace{1em}
\begin{enumerate}
\item {\em Ассоциативность}: $(x\otimes y)\otimes z = x\otimes
  (y\otimes z)$ (после забывания блочных структур).
\item {\em Дистрибутивность относительно сложения}: $(x+y)\otimes z =
  x\otimes z + y\otimes z$, $x\otimes (y+z) = x\otimes y + x\otimes
  z$.
\item {\em Однородность}: $(\alpha x)\otimes y = \alpha (x\otimes y) =
  x\otimes (\alpha y)$.
\item {\em Взаимная дистрибутивность кронекерова произведения и
    умножения}: $(xy)\otimes (uv) = (x\otimes u)(y\otimes v)$.
\end{enumerate}
\end{proposition}
\begin{proof}
Все эти свойства легко проверяются прямым вычислением.
\end{proof}

Наконец, мы готовы показать, что матрица тензорного произведения
линейных отображений является кронекеровым произведением матриц этих
отображений. Для простоты мы ограничимся случаем линейных операторов
(то есть, квадратных матриц). Рассмотрим линейные операторы
$\ph\colon U\to U$, $\psi\colon V\to V$ на конечномерных пространствах
$U$, $V$. Как обычно, после выбора базисов $(e_1,\dots,e_m)$ в $U$ и
$(f_1,\dots,f_n)$ в $V$ мы можем считать, что $U = k^m$, $V=k^n$~---
пространства столбцов. В этом случае векторы $u\in U$, $v\in V$
истолковываются как столбцы высоты $m$ и $n$, соответственно, а
линейный оператор~--- как умножение на квадратную матрицу: если
$a,b$~--- матрицы операторов $\ph$, $\psi$ в выбранных базисах,
получаем линейные отображения
$$
\ph\colon U\to U, u\mapsto au,
$$
где $a\in M(m,k)$, и
$$
\psi\colon V\to V, v\mapsto bv,
$$
где $b\in M(n,k)$.

В пространстве $U\otimes V$ имеется тензорный базис $(e_i\otimes
f_j)$, в котором $mn$ элементов. Он позволяет отождествить $U\otimes
V$ с $k^{mn}$. При нашем упорядочивании тензорного базиса
(см. определение~\ref{dfn:tensor_basis}) это отождествление выглядит
следующим образом. Пусть $u = \sum_i u_i e_i$, $v = \sum_j v_j f_j$.
Тогда $u\otimes v = (\sum_i u_ie_i)\otimes (\sum_j v_jf_j)
 = \sum_{i,j}u_iv_j(e_i\otimes f_j)$. Это означает, что
$$
\begin{pmatrix}u_1\\ \dots \\ u_m\end{pmatrix}
\otimes
\begin{pmatrix}v_1\\ \dots \\ v_n\end{pmatrix}
=
\begin{pmatrix}u_1v_1\\ \dots \\ u_1v_n \\ u_2v_1 \\ \dots \\ u_mv_1
  \\ \dots \\ u_mv_n\end{pmatrix}.
$$

\begin{theorem}
Если матрица оператора $\ph$ в базисе $(e_i)$ равна $a$, а матрица
оператора $\psi$ в базисе $(f_j)$ равна $b$, то матрица оператора
$\ph\otimes\psi$ в тензорном базисе $(e_i\otimes f_j)$ равна
кронекеровому произведениею $a\otimes b$.
\end{theorem}
\begin{proof}
Пусть $u\in U$, $v\in V$~--- произвольные векторы. По определению
тензорное произведение отображений $\ph$ и $\psi$ действует на
разложимый тензор $u\otimes v\in U\otimes V$ следующим образом:
$(\ph\otimes\psi)(u\otimes v) = \ph(u)\otimes\psi(v)$.
С другой стороны, кронекерово произведение $a\otimes b$ умножается на
столбец $u\otimes v$ следующим образом:
$(a\otimes b)(u\otimes v) = (au\otimes bv)$~--- здесь мы
воспользовались свойством~4 из
предложения~\ref{prop:kronecker_product}.
Но при наших отождествлениях $au = \ph(u)$, $bv = \psi(v)$. Поэтому
отображение $\ph\otimes\psi$ совпадает с умножением на матрицу
$a\otimes b$ на разложимых тензорах, а значит и везде.
\end{proof}

\subsection{Тензорные пространства}

\literature{[F], гл. XIV, \S~4, п. 4; [K2], гл. 6, \S~1, п. 1; [vdW],
  гл. IV, \S~24; [KM], ч. 4, \S~3, пп. 1--2.}

Пусть $V$~--- конечномерное векторное пространство над полем $k$, и
$V^* = \Hom(V,k)$~--- двойственное к нему. В ближайших
параграфах мы будем изучать векторные пространства
$$
T^p_q(V) = \underbrace{V\otimes\dots\otimes V}_{p\mbox{ раз}} \otimes
\underbrace{V^*\otimes\dots\otimes V^*}_{q\mbox{ раз}}.
$$
Пространство $T^p_q(V)$ традиционно называется пространством $q$ раз
ковариантных и $p$ раз контравариантных тензоров, или просто
\dfn{тензорным пространством}\index{тензорное пространство} (если из
контекста понятно, о каких значениях $p$, $q$ идет речь). Элементы
тензорных пространств называются \dfn{тензорами}\index{тензор} над
$V$. Если $x\in T^p_q(V)$, то пара $(p,q)$ называется
\dfn{типом}\index{тип тензора} тензора $x$, $p$ называется его
\dfn{контравариантной
  валентностью}\index{валентность!контравариантная}, а 
$q$~--- его \dfn{ковариантной
  валентностью}\index{валентность!ковариантная}. Сумма $p+q$
называется \dfn{полной валентностью}\index{валентность!полная}. Если
$p=0$, тензор $x$ называется \dfn{чисто
  ковариантным}\index{тензор!чисто ковариантный}, а если $q=0$~---
\dfn{чисто контравариантным}\index{тензор!чисто контравариантный}.

На самом деле, нам уже встречались тензоры небольшой валентности:
\begin{itemize}
\item При $p=q=0$ удобно считать, что $T^0_0(V) = k$; тензоры типа
  $(0,0)$~--- это просто скаляры.
\item $T^1_0(V)=V$~--- векторы;
\item $T^0_1(V)=V^*$~--- ковекторы;
\item $T^2_0(V) = V\otimes V = (V^*\otimes V^*)^* = \Hom(V^*\otimes
  V^*,k)$. Напомним, что (по определению тензорного произведения)
  линейные отображения из $V^*\otimes V^*$ в $k$~--- это то же самое, что
  {\em билинейные} отображения из $V^*\times V^*$ в $k$. Поэтому тензоры
  типа $(2,0)$ можно интерпретировать как билинейные формы на $V^*$.
\item $T^1_1(V) = V\otimes V^* = \Hom(V,V)$~--- линейные операторы на
  $V$.
\item $T^0_2(V) = V^*\otimes V^* = (V\otimes V)^* = \Hom(V\otimes
  V,k)$. Как и в случае тензоров типа $(2,0)$, заметим, что линейные
  отображения из $V\otimes V$ в $k$~--- это в точности билинейные
  отображения из $V\times V$ в $k$. Поэтому тензоры типа $(0,2)$ можно
  интерпретировать как билинейные формы на $V$.
\item $T^1_2(V) = V\otimes V^*\otimes V^* = (V\otimes V)^*\otimes V =
  \Hom(V\otimes V,V)$; то есть, тензоры типа $(1,2)$~--- это
  билинейные отображения из $V\times V$ в $V$; при желании можно это
  интерпретировать как задание умножения на векторах,
  дистрибутивного относительно суммы.
\end{itemize}

\subsection{Тензоры в классических обозначениях}

\literature{[F], гл. XIV, \S~1; [K2], гл. 6, \S~1, пп. 3, 4; [KM],
  ч. 4, \S~4, пп. 1--4.}

В прикладной математике и инженерных науках все встречающиеся тензоры
(тензор деформации, тензор электромагнитного поля, тензор инерции,
тензор Эйнштейна\dots) возникают почти исключительно в координатной
записи.
Напомним, что если в пространстве $V$ выбран базис $\mc E=(e_1,\dots,e_n)$,
то в двойственном пространстве возникает двойственный базис
$(e_1^*,\dots,e_n^*)$. Для того, чтобы приблизить наши обозначения к
традиционным, мы будем обозначать двойственный базис через
$(e^1,\dots,e^n)$.
Каждый вектор $v\in V$ можно разложить по базису $\mc E$:
$$
v = \sum e_i v^i = \begin{pmatrix}e_1 & \dots & e_n\end{pmatrix}
\begin{pmatrix}v^1\\\vdots\\ v^n\end{pmatrix},
$$
а каждый ковектор $\ph\in V^*$~--- по двойственному базису:
$$
\ph = \sum \ph_i e^i = \begin{pmatrix}\ph_1 & \dots &
  \ph_n\end{pmatrix}
\begin{pmatrix}e^1\\\vdots\\ e^n\end{pmatrix}.
$$

При этом в тензорном пространстве $T^p_q$ (для произвольных $p,q$)
возникает тензорный базис, состоящий из векторов вида
$e_{i_1}\otimes\dots\otimes e_{i_p}\otimes
e^{j_1}\otimes\dots\otimes e{j_q}$, где
$1\leq i_1,\dots,i_p,j_1,\dots,j_q\leq n$.
Таким образом, каждый тензор $x\in T^p_q(V)$ можно единственным
образом записать в виде
$$
x = \sum_{\substack{i_1,\dots,i_p \\ j_1,\dots,j_q}}
x^{i_1\dots i_p}_{j_1\dots j_q} e_{i_1}\otimes\dots\otimes
e_{i_p}\otimes e^{j_1}\otimes\dots\otimes e^{j_q},
$$
где $x^{i_1\dots i_p}_{j_1\dots j_q}\in k$~--- координаты тензора в
этом базисе.

Традиционно тензор задавался явным перечислением своих координат. При
этом, поскольку этот набор зависит от выбора базиса, приходится
указывать, как же преобразуются координаты тензора при другом выборе
базиса.

Для этого выберем в $V$ другой базис $\mc F = (f_1,\dots,f_n)$,
который будет называться {\em новым} (в отличие от {\em старого}
базиса $\mc E = (e_1,\dots,e_n)$). Напомним, что мы изучали, как
связаны координаты векторов в этих базисах, с помощью [обратимой]
матрицы перехода
$C = (\mc E\rsa\mc F)$
(см. определение~\ref{def:change_of_basis_matrix}):
$$
\begin{pmatrix} f_1 & \dots & f_n\end{pmatrix} =
\begin{pmatrix} e_1 & \dots & e_n\end{pmatrix}\cdot C.
$$
Вспомним, как преобразуются координаты вектора $v = \sum_i e_iv^i$ при
замене базиса:
$$
v = \begin{pmatrix}e_1 & \dots & e_n\end{pmatrix}
\begin{pmatrix}v^1\\\vdots\\ v^n\end{pmatrix} =
\begin{pmatrix}e_1 & \dots & e_n\end{pmatrix}\cdot C\cdot C^{-1}\cdot
\begin{pmatrix}v^1\\\vdots\\ v^n\end{pmatrix} =
\begin{pmatrix}f_1 & \dots & f_n\end{pmatrix}\cdot
C^{-1}\begin{pmatrix}v^1\\\vdots\\ v^n\end{pmatrix}.
$$
Таким образом, при переходе в новый базис столбец координат вектора
умножается на $C^{-1}$. Это означает
(см. замечание~\ref{rem:contravariant_change}), что координаты вектора
преобразуются {\em контравариантным образом}; именно поэтому число $p$
в определении тензорного пространства $T^p_q(V)$ называется
контравариантной валентностью.
В то же время координаты {\em ковектора} преобразуются
{\em ковариантным образом}. Действительно, по определению
двойственного базиса
$$
e^i(e_j)= \begin{cases}1,&i=j\\ 0,&i\neq j\end{cases}.
$$
Это означает, что
$$
\begin{pmatrix}e^1\\ \vdots \\ e^n\end{pmatrix}
\cdot
\begin{pmatrix}e_1 & \dots & e_n\end{pmatrix} =
\begin{pmatrix} 1 & \dots & 0\\\vdots & \ddots & \vdots\\0 & \dots &
  1\end{pmatrix} = E.
$$
и аналогично для базиса $\mc F$.
Домножим последнее равенство на $C^{-1}$ слева и на $C$ справа:
$$
C^{-1}\begin{pmatrix}e^1\\ \vdots \\ e^n\end{pmatrix}
\cdot
\begin{pmatrix}e_1 & \dots & e_n\end{pmatrix}C =
C^{-1}EC = E.
$$
В левой части стоит
$C^{-1}\begin{pmatrix}e^1\\ \vdots \\ e^n\end{pmatrix}
\cdot
\begin{pmatrix}f_1 & \dots & f_n\end{pmatrix}$,
поэтому
$$
C^{-1}\begin{pmatrix}e^1\\ \vdots \\ e^n\end{pmatrix} = 
\begin{pmatrix}f^1\\ \vdots \\ f^n\end{pmatrix}.
$$
Это и означает, что двойственный базис преобразуется с помощью матрицы
$C^{-1}$, а потому координаты ковекторов преобразуются с помощью
матрицы $(C^{-1})^{-1} = C$. Это несложно проверить и непосредственно:
если $\ph = \sum \ph_i e^i$, то
$$
\ph =
\begin{pmatrix}\ph_1 & \dots & \ph_n\end{pmatrix}
\begin{pmatrix}e^1\\\vdots\\ e^n\end{pmatrix} =
\begin{pmatrix}\ph_1 & \dots & \ph_n\end{pmatrix}\cdot C\cdot C^{-1}\cdot
\begin{pmatrix}e^1\\\vdots\\ e^n\end{pmatrix} =
\begin{pmatrix}\ph_1 & \dots & \ph_n\end{pmatrix}C\cdot
\begin{pmatrix}f^1\\\vdots\\ f^n\end{pmatrix}.
$$

У нас все готово к тому, чтобы выяснить, как меняются координаты
произвольного тензора при замене базиса. Пусть
$$
x = \sum_{\substack{i_1,\dots,i_p\\j_1,\dots,j_q}}
y^{i_1\dots i_p}_{j_1\dots j_q}f_{i_1}\otimes\dots\otimes
f_{i_p}\otimes f^{j_1}\otimes\dots\otimes f^{j_q}
$$
--- выражение того
же тензора $x$ в новом тензорном базисе. Мы хотим выразить
$\left( y^{i_1\dots i_p}_{j_1\dots j_q}\right)$ через
$\left( x^{i_1\dots i_p}_{j_1\dots j_q}\right)$. В следующей теореме
удобно элемент матрицы $C$, стоящий на пересечении $i$-й строки и
$j$-го столбца записывать как $C^i_j$, а не $C_{ij}$.

\begin{theorem}
Пусть $C = (C^i_j)$~--- матрица перехода от старого базиса к новому,
$\tld{C} = (\tld{C}^i_j) = C^{-1}$~--- обратная к ней. Тогда
координаты тензора $x\in T^p_q(V)$ в новом тензорном базисе следующим
образом выражаются через его координаты в старом тензорном базисе:
$$
y^{i_1\dots i_p}_{j_1\dots j_q} =
\sum_{\substack{h_1,\dots,h_p\\k_1,\dots,k_q}}
\tld{C}^{i_1}_{h_1}\dots\tld{C}^{i_p}_{h_p}C^{k_1}_{j_1}\dots C^{k_q}_{j_q}
x^{h_1\dots h_p}_{k_1\dots k_q}
$$
\end{theorem}
\begin{proof}
Достаточно доказать эту формулу для разложимых тензоров, а в этом
случае нужно применить формулы преобразования координат векторов и
ковекторов в каждом из сомножителей.
\end{proof}
Иными словами, координаты тензора преобразуются контравариантно (при
помощи матрицы $C^{-1}$) по контравариантным сомножителям, и
ковариантно (при помощи матрицы $C$) по ковариантным сомножителям.
