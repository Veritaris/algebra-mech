\documentclass[12pt]{article}
\usepackage[T2A]{fontenc}
\usepackage[utf8]{inputenc}
\usepackage[russian]{babel}
\usepackage{mdwlist}
\usepackage{amsfonts}
\usepackage{amssymb}
\usepackage{amsmath}
\usepackage{amsthm}
\usepackage{paralist}
\usepackage{ccfonts,eulervm,euler}
\renewcommand{\bfdefault}{sbc}
%\usepackage{ulem}
\usepackage[margin=0.7in,bmargin=0.7in]{geometry}


\pagestyle{empty}

\DeclareMathOperator{\Hom}{Hom}

\newcommand\glava[1]{{\bf\hfill #1}}

\begin{document}

\begin{center}
{\Large Вопросы экзамена по алгебре}
\par\medskip
Группы 251, 253 (лектор Александр Лузгарев)
\par\medskip
Четвертый семестр, весна 2016
\end{center}

\glava{Эвклидовы и унитарные пространства}
\begin{compactenum}
\item Эвклидовы и унитарные пространства: определения и первые
  примеры.
\item Норма и угол.
\item Матрица Грама.
\item Поведение матрицы Грама при замене базиса, ее обратимость.
\item Ортогонализация Грама--Шмидта.
\item Ортогональные и унитарные матрицы, равносильные определения.
\item Ортонормированные базисы. Теорема Риса.
\item Ортогональное дополнение, его свойства.
\item Ортогональная проекция.
\item Сопряженное отображение: существование и единственность.
\item Свойства сопряжения. Матрица сопряженного отображения.
\item Самосопряженные операторы. Критерий равенства нулю самосопряженного оператора.
\item Нормальные операторы, их собственные числа и векторы.
\item Спектральная теорема для нормальных операторов в унитарном пространстве.
\item Спектральная теорема для самосопряженных операторов в эвклидовом пространстве.
\item Спектральная теорема для нормальных операторов в эвклидовом пространстве.
\item Самосопряженные, кососимметрические, унитарные операторы в
  унитарных и эвклидовых пространствах.
\item Изометрии.
\item Теорема Эйлера о вращениях, приведение квадратичной формы к
  диагональному виду, разложение пространства в ортогональную прямую
  сумму собственных подпространств.
\item Положительно определенные операторы.
\item Извлечение квадратного корня из положительно определенного
  оператора.
\item Полярное разложение.
\suspend{compactenum}
\glava{Теория групп}
\resume{compactenum}
\item Группы: определение, примеры.
\item Подгруппы: определение, примеры. Подгруппы аддитивной группы.
\item Подгруппа, порожденная множеством: две конструкции
\item Классы смежности, разбиение на классы и соответствующие
  отношения эквивалентности.
\item Нормальные подгруппы: определение и равносильные
  переформулировки.
\item Гомоморфизмы групп: определение, примеры.
\item Ядро и образ гомоморфизма.
\item Ядро и инъективность, изоморфизм и биективность.
\item Конструкция фактор-группы.
\item Теорема о гомоморфизме.
\item Циклические группы: определение и классификация. Порядок
  элемента.
\item Равномощность множеств левых и правых смежных классов. Теорема
  Лагранжа.
\item Следствия теоремы Лагранжа.
\item Прямое произведение групп
\item Критерий разложения группы в прямое произведение подгрупп.
\item Разложение перестановки в произведение независимых циклов.
\item Описание классов сопряженности в симметрической группе.
\item Теорема Кэли.
\item Диэдральная группа.
\suspend{compactenum}
\glava{Полилинейная алгебра}
\resume{compactenum}
\item Тензорное произведение двух пространств.
\item Тензорное произведение нескольких пространств. Ассоциативность,
  коммутативность. Тензорный базис.
\item Двойственное пространство. Изоморфизм пространства с дважды
  двойственным.
\item Выражение $\Hom$ через $\otimes$.
\item Двойственность и $\otimes$.
\item Сопряженность $\otimes$ и $\Hom$.
\item Тензорное произведение линейных отображений. Его свойства,
  тензорное произведение $\Hom$-пространств.
\item Кронекерово произведение матриц. Матрица тензорного
  произведения.
\item Тензорные пространства. Координаты тензора.
\item Преобразование координат тензора при замене базиса.
\end{compactenum}

\end{document}

